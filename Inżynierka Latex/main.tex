%%%%%%%%%%%%%%%%%%%%%%%%%%%%%%%%%%%%%%%%%%%%%%%%%%%%%%%%%%%%%%%%%%%%%%%%%%%
%% Bachelor's & Master's Thesis Template                                 %%
%% Copyleft by Dawid Weiss & Marta Szachniuk                             %%
%% Faculty of Automatic Control, Robotics, and Electrical Engineering    %%
%% Poznan University of Technology, 2023                                 %%
%%%%%%%%%%%%%%%%%%%%%%%%%%%%%%%%%%%%%%%%%%%%%%%%%%%%%%%%%%%%%%%%%%%%%%%%%%%


% Szkielet dla pracy inżynierskiej/magisterskiej pisanej w języku polskim.
\documentclass[polish,bachelor,a4paper,oneside]{ppcreefthesis}

% Template for an engineering/master's thesis written in English.
% \documentclass[english,bachelor,a4paper,oneside]{ppcreefthesis}


\usepackage[utf8]{inputenc}
\usepackage[OT4]{fontenc}
\usepackage{xcolor}   % Do kolorowania tekstu
\usepackage{soul}      % Do podkreślania i podświetlania tekstu
\usepackage[usenames,dvipsnames]{xcolor}
\usepackage{graphicx}
\usepackage{tabularx}
\usepackage{multirow}
\usepackage{multicol}
\usepackage{array}
\usepackage{float}
\usepackage{changepage}
\usepackage{tikz}
\usetikzlibrary{shapes, arrows, positioning,shapes.geometric, calc}
\usepackage{adjustbox}
\usepackage{hyperref}
\usepackage{subcaption}
\usepackage{enumitem}
\usepackage{listings}
\usepackage{enumitem}
\usepackage{longtable}
\usepackage{listings} 
\usepackage{textcomp}



\newcommand{\customnote}[1]{\sethlcolor{yellow}\hl{#1}}



\newcolumntype{W}{>{\centering\arraybackslash}p{5cm}}
\newcolumntype{w}{>{\centering\arraybackslash}m{1.2cm}}
% Config listingu:

% Definicje kolorów
\definecolor{functioncolor}{rgb}{0.16,0.37,0.64}  % rgb(41,94,163)
\definecolor{collectioncolor}{rgb}{0.08,0.53,0.56} % rgb(20,134,143)
\definecolor{numbercolor}{rgb}{0.71,0.29,0.0}      % rgb(182,73,0)
\definecolor{stringcolor}{rgb}{0.64,0.09,0.09}     % rgb(162,23,23)
\definecolor{commentcolor}{rgb}{0.0,0.45,0.0}      % rgb(0,114,0)
\definecolor{boolcolor}{rgb}{0.47,0.33,0.28}       % rgb(121,85,72)

%CFG do listingu javascript
\lstdefinelanguage[ECMAScript2015]{JavaScript}[]{JavaScript}{
  morekeywords=[1]{await, async, case, catch, class, const, default, do,
    enum, export, extends, finally, from, implements, import, instanceof,
    let, static, super, switch, throw, try},
  morestring=[b]` % Interpolation strings.
}


%
% JavaScript version 1.1 by Gary Hammock
%
% Reference:
%   B. Eich and C. Rand Mckinney, "JavaScript Language Specification
%     (Preliminary Draft)", JavaScript 1.1.  1996-11-18.  [Online]
%     http://hepunx.rl.ac.uk/~adye/jsspec11/titlepg2.htm
%

\lstdefinelanguage{JavaScript}{
  morekeywords=[1]{break, continue, delete, else, for, function, if, in,
    new, return, this, typeof, var, void, while, with},
  % Literals, primitive types, and reference types.
  morekeywords=[2]{false, null, true, boolean, number, undefined,
    Array, Boolean, Date, Math, Number, String, Object},
  % Built-ins.
  morekeywords=[3]{eval, parseInt, parseFloat, escape, unescape},
  sensitive,
  morecomment=[s]{/*}{*/},
  morecomment=[l]//,
  morecomment=[s]{/**}{*/}, % JavaDoc style comments
  morestring=[b]',
  morestring=[b]"
}[keywords, comments, strings]



\lstalias[]{ES6}[ECMAScript2015]{JavaScript}

% Requires package: color.
\definecolor{mediumgray}{rgb}{0.3, 0.4, 0.4}
\definecolor{mediumblue}{rgb}{0.0, 0.0, 0.8}
\definecolor{forestgreen}{rgb}{0.13, 0.55, 0.13}
\definecolor{darkviolet}{rgb}{0.58, 0.0, 0.83}
\definecolor{royalblue}{rgb}{0.25, 0.41, 0.88}
\definecolor{crimson}{rgb}{0.86, 0.8, 0.24}

\lstdefinestyle{JSES6Base}{
  backgroundcolor=\color{white},
  basicstyle=\ttfamily,
  breakatwhitespace=false,
  breaklines=false,
  captionpos=b,
  columns=fullflexible,
  commentstyle=\color{mediumgray}\upshape,
  emph={},
  emphstyle=\color{crimson},
  extendedchars=true,  % requires inputenc
  fontadjust=true,
  frame=single,
  identifierstyle=\color{black},
  keepspaces=true,
  keywordstyle=\color{mediumblue},
  keywordstyle={[2]\color{darkviolet}},
  keywordstyle={[3]\color{royalblue}},
  numbers=left,
  numbersep=5pt,
  numberstyle=\tiny\color{black},
  rulecolor=\color{black},
  showlines=true,
  showspaces=false,
  showstringspaces=false,
  showtabs=false,
  stringstyle=\color{forestgreen},
  tabsize=2,
  title=\lstname,
  upquote=true  % requires textcomp
}

\lstdefinestyle{JavaScript}{
  language=JavaScript,
  style=JSES6Base
}
\lstdefinestyle{ES6}{
  language=ES6,
  style=JSES6Base
}

% Definicja stylu dla Power FX
\lstdefinelanguage{PowerFx}{
    morekeywords={Set, ClearCollect, LookUp, AddColumns, Filter, Max, Text, Now, If, And, Or,MyProfile,Run,Substitute,Concat},
    sensitive=true,
    keywordstyle=\color{functioncolor}\bfseries,
    commentstyle=\color{commentcolor}\itshape,
    stringstyle=\color{stringcolor},
    identifierstyle=\color{black},   % Standardowy kolor dla identyfikatorów
    numbers=left,
    numberstyle=\tiny\color{black},
    morestring=[b]",
    morestring=[s]{\%}{\%},
    morekeywords=[2]{true,false},  % Kolor dla true/false
    keywordstyle=[2]\color{boolcolor},
    morekeywords=[3]{1,2,3,4,5},   % Kolor dla liczb (możesz dodać inne liczby)
    keywordstyle=[3]\color{numbercolor},
    % Określenie, które identyfikatory traktować jako kolekcje
    morekeywords=[4]{Lista_Uslug, Lista_Kwot, Lista_Indykacji,CombinedData},  % Przykład kolekcji
    keywordstyle=[4]\color{collectioncolor},
}

\lstset{
literate=%
    *{0}{{{\color{numbercolor}0}}}1
     {1}{{{\color{numbercolor}1}}}1
     {2}{{{\color{numbercolor}2}}}1
     {3}{{{\color{numbercolor}3}}}1
     {4}{{{\color{numbercolor}4}}}1
     {5}{{{\color{numbercolor}5}}}1
     {6}{{{\color{numbercolor}6}}}1
     {7}{{{\color{numbercolor}7}}}1
     {8}{{{\color{numbercolor}8}}}1
     {9}{{{\color{numbercolor}9}}}1
     {ą}{{\k a}}1
  	    {Ą}{{\k A}}1
           {ż}{{\. z}}1
           {Ż}{{\. Z}}1
           {ź}{{\' z}}1
           {Ź}{{\' Z}}1
           {ć}{{\' c}}1
           {Ć}{{\' C}}1
           {ę}{{\k e}}1
           {Ę}{{\k E}}1
           {ó}{{\' o}}1
           {Ó}{{\' O}}1
           {ń}{{\' n}}1
           {Ń}{{\' N}}1
           {ś}{{\' s}}1
           {Ś}{{\' S}}1
           {ł}{{\l}}1
           {Ł}{{\L}}1
     {true}{{{\color{boolcolor}true}}}5
     {false}{{{\color{boolcolor}false}}}5,
    backgroundcolor=\color{white},   % kolor tła
    basicstyle=\ttfamily\small,       % styl tekstu
    breaklines=true,                 % łamanie długich linii
    showstringspaces=false,          % nie pokazuj spacji w stringach
    captionpos=b,                    % pozycja podpisu
    escapeinside={\%*}{*)},          % umożliwia wstawienie LaTeX w kodzie
    morekeywords=[5]{UżytkownicyusługiOffice365},  % Aby liczba 365 nie była pomarańczowa
    keywordstyle=[5]\color{black},
}

\lstset{language=PowerFx}

%--------------------------------------
% Strona tytułowa \ Front page
%--------------------------------------

% Autorzy pracy, jeśli jest ich więcej niż jeden
% wstaw między nimi separator \and

% Authors of the thesis, if there is more than one
% insert a separator between them \and

\author{%
   Remigiusz Wolniak \album{151192} \and 
   Michał Gajdzis \album{151066} }
\authortitle{}                                % Do not change.

\title{Automatyzacja procesu planowania wydatków na~usługi IT w VW Poznań z wykorzystaniem Microsoft~Power~Platform}

% Promotor pracy
% Your supervisor comes here.

\ppsupervisor{dr hab.~inż.~Piotr~Kaczmarek} 
\ppinstitute{Instytut Robotyki i Inteligencji Maszynowej \\ }
 %            Zakład Sterowania i Elektroniki Przemysłowej

% Rok złożenia pracy
% Year of final submission (not graduation!)
\ppyear{2025}                                 


\begin{document}

% Pierwsza strona zaczyna się tutaj



% Front matter starts here
\frontmatter\pagestyle{empty}%
\maketitle\cleardoublepage%

% Ponizej tresc na pierwsza strone pracy dyplomowej z VW (od Kroteckiego)

\begin{center}
    \Large \textbf{Zastrzeżenie dotyczące treści pracy dyplomowej}
\end{center}

\vspace{1cm}

%\onehalfspacing % Ustawienie interlinii 1.5

Niniejsza praca dyplomowa/magisterska/inżynierska/licencjacka/zaliczeniowa*
zawiera treści, informacje itp. udostępnione przez spółkę Volkswagen Poznań Sp.
z o.o.
z siedzibą przy ulicy Warszawskiej 349, 61-060 w Poznaniu, mogące stanowić
tajemnice przedsiębiorstwa tej spółki i mogące być wykorzystane wyłącznie dla
potrzeb napisania niniejszej pracy. Wobec powyższego niedozwolone jest
wykorzystywanie całości lub części niniejszej pracy, a także udostępnianie całości
lub części pracy komukolwiek jak również kopiowanie, powielanie, publikowanie itp.
bez pisemnej zgody spółki Volkswagen Poznań Sp. z o.o. – zastrzeżenie to nie ma
zastosowania do przypadku udostępnienia niniejszej pracy nauczycielom
akademickim w celu oceny, recenzji
i obrony ww. pracy. Podmioty, które naruszą powyższy zakaz ponoszą
odpowiedzialność odszkodowawczą wobec spółki Volkswagen Poznań Sp. z o.o.

\vspace{1cm}

\textit{*Podkreśl odpowiedni rodzaj pracy.}

%--------------------------------------
% Miejsce na kartę pracy dyplomowej - opcjonalnie, nie jest wymagane
%--------------------------------------
\thispagestyle{empty}\vspace*{\fill}%
\begin{center}Tutaj będzie skan karty pracy dyplomowej. \end{center}%
\vfill\cleardoublepage%
%--------------------------------------


%--------------------------------------
% Spis treści
%--------------------------------------

\pagenumbering{Roman}
\pagestyle{ppfcmthesis}
\tableofcontents*
\cleardoublepage % Zaczynamy od nieparzystej strony

%--------------------------------------
% Rozdziały
%--------------------------------------

%Najwygodniej jeśli każdy rozdział znajduje się w oddzielnym pliku
\mainmatter%
\chapter{Wstęp}
Współczesny świat biznesu stawia coraz większe wymagania wobec przedsiębiorstw, zarówno w zakresie wydajności procesów, jak i precyzji podejmowanych decyzji. Tradycyjne metody zarządzania i przetwarzania danych, oparte na pracy manualnej i mało efektywnych narzędziach, stają się niewystarczające w obliczu rosnącej skali operacji oraz konieczności szybkiego i niezawodnego podejmowania decyzji. W odpowiedzi na te wyzwania coraz większą rolę odgrywają rozwiązania z zakresu automatyzacji biurowej, które pozwalają na oszczędność czasu i zasobów, usprawnienie kluczowych procesów organizacyjnych oraz minimalizację ryzyka błędów ludzkich.
\par Jednym z obszarów, w którym automatyzacja znajduje zastosowanie, jest zarządzanie usługami IT i powiązanymi kosztami. W dużych organizacjach o rozbudowanej strukturze, konieczność gromadzenia, analizy oraz weryfikacji danych finansowych stanowi poważne wyzwanie. Dzięki wdrożeniu odpowiednich narzędzi, procesy te mogą być prowadzone w sposób uporządkowany i efektywny, umożliwiając jednocześnie bieżącą kontrolę nad wydatkami oraz lepsze planowanie budżetowe.
\par   Ustandaryzowany i zautomatyzowany przepływ informacji ogranicza ryzyko powielania błędów i pozwala na skrócenie czasu potrzebnego na wykonanie poszczególnych zadań. Dodatkowo, wdrożenie automatyzacji zapewnia większą przejrzystość i ułatwia dostęp do informacji każdemu uczestnikowi procesu.
\par W dobie intensywnej cyfryzacji przedsiębiorstw oraz dynamicznego rozwoju technologii, automatyzacja biurowa staje się konieczna, aby sprostać wymaganiom współczesnego rynku. Odpowiednio zaprojektowane systemy i narzędzia wspierają nie tylko wydajność operacyjną, ale także strategiczne zarządzanie zasobami, umożliwiając rozwój w innych obszarach swojej działalności.
\vspace{1cm}
\par Celem pracy jest opracowanie aplikacji usprawniającej proces podejmowania decyzji dotyczących zakupu \emph{usług IT}\footnote{\emph{Usługi IT}  należy rozumieć jako licencje oraz klucze dostępu do używanych systemów informatycznych.} na najbliższy rok kalendarzowy. Praca została wykonana z wykorzystaniem \emph{Power Platform} oraz \emph{SharePoint}, które są integralną cześcią pakietu \emph{Microsoft 365}. Zdecydowano się na wybór tego rozwiązania, ponieważ pozwala ono na prostą integrację między programami wchodzącymi w skład pakietu. Ponadto, każdy z uczestników procesu ma dostęp do wspomnianych serwisów, co pozwala uniknąć dodatkowych kosztów.

\par\textbf{\textcolor{red}{DOPISAĆ:} \newline
    \begin{quote}
        \color{red}
        Struktura pracy jest następująca. W rozdziale 2.  \newline przedstawiono powód wykonywanego zadania, wraz z jego wyjaśnieniem oraz opisem wykorzystanych kompontenów   \newline
        Rozdział 3 jest poświęcony założeniom projektowym i architekturze \ldots (kilka zdań).  \newline
        Rozdział 4 zawiera implementację \ldots (kilka zdań) \ldots itd.  \newline
        Rozdział X stanowi podsumowanie pracy.
    \end{quote}}
\textbf{\textcolor{red}{
        W przypadku prac inżynierskich zespołowych lub magisterskich 2-osobowych, po tych dwóch w/w akapitach
        musi w pracy znaleźć się akapit, w którym będzie opisany udział w pracy poszczególnych członków zespołu. Na przykład:}
    \begin{quote}
        \color{red}
        Jan Kowalski w ramach niniejszej pracy wykonał projekt tego i tego, opracował \ldots
        Grzegorz Brzęczyszczykiewicz wykonał \ldots, itd.
    \end{quote}}



\chapter{Stan wiedzy}

Praca koncentruje się na automatyzacji biurowej procesu wymiany wycen usług oraz kosztów IT, który do tej pory opierał się na ręcznym zarządzaniu za pomocą plików programu Excel. Tego rodzaju podejście było czasochłonne, mało efektywne i podatne na błędy, szczególnie w przypadku dużych zbiorów danych oraz konieczności współpracy między działami firmy.

W celu usprawnienia tego procesu zaprojektowano oraz wdrożono aplikację opartą na plat ormie Microsoft Power Apps. Nowe rozwiązanie oferuje intuicyjny interfejs, automatyzację kluczowych etapów oraz możliwość pełnego śledzenia przepływu danych. Dzięki temu aplikacja znacząco przyczynia się do zwiększenia efektywności pracy biurowej, minimalizując ryzyko błędów i usprawniając współpracę w organizacji.


%\chapter{Podstawy teoretyczne}

Rozdział teoretyczny --- przegląd literatury naświetlający stan wiedzy na dany temat. 

Przegląd literatury naświetlający stan wiedzy na dany temat obejmuje rozdziały pisane na podstawie
literatury, której wykaz zamieszczany jest w części pracy pt.~\emph{Literatura} (lub inaczej \emph{Bibliografia},
\emph{Piśmiennictwo}). W tekście pracy muszą wystąpić odwołania do wszystkich pozycji zamieszczonych w
wykazie literatury. \textbf{Nie należy odnośników do literatury umieszczać w stopce strony.} Student jest
bezwzględnie zobowiązany do wskazywania źródeł pochodzenia informacji przedstawianych w pracy,
dotyczy to również rysunków, tabel, fragmentów kodu źródłowego programów itd. Należy także podać
adresy stron internetowych w przypadku źródeł pochodzących z Internetu.



\chapter{Architektura rozwiązania}

Niniejszy rozdział przedstawia architekturę rozwiązania, obejmującą zarówno założenia projektowe, jak i koncepcję opracowywanego systemu. Założenia projektowe określają podstawowe wymagania oraz wytyczne, stanowiąc punkt wyjścia dla opracowywanego rozwiązania. Natomiast koncepcja rozwiązania, uwzględniająca zasadnicze założenia, strukturę i logikę działania, stanowi podstawę implementacji rozwiązania.

\section{Założenia projektowe}
Założenia projektowe stanowią zbór wytycznych, które określają funkcjonalność oraz wymagania techniczne, tworzonego rozwiązania.




\subsection{Systematyzacja danych}
Jedną z kluczowych funkcji omawianej aplikacji jest systematyzacja danych. Arkusze kalkulacyjne przesyłane przez oddział w Wolfsburgu często nie posiadają ustalonej struktury, co znacząco utrudniało ich czytelność i wymagało od użytkowników dodatkowego czasu na analizę zawartych informacji.

Brak jednolitego formatu danych uniemożliwiał również stworzenie spójnej bazy, co ograniczało możliwość ich wykorzystania w systemach automatyzacji procesów biznesowych. Dzięki wdrożonemu rozwiązaniu możliwe jest ujednolicenie danych, co pozwala na ich efektywne zarządzanie i automatyczne przetwarzanie.
\subsection{Archiwizacja danych}
Utworzenie bazy danych gromadzącej informacje o wcześniejszych działaniach realizowanych w ramach projektowanego systemu stanowi istotny element zapewniający ciągłość procesów decyzyjnych. Dzięki systematycznej archiwizacji nowi użytkownicy mogą szybko zapoznać się z przebiegiem procedur i lepiej zrozumieć kontekst dotychczas podejmowanych decyzji. Dostęp do zasobów historycznych nie tylko skraca czas potrzebny na pełne wdrożenie w funkcjonowanie systemu, lecz także usprawnia przetwarzanie danych bieżących.

\begin{comment}
Utworzenie bazy danych zawierającej dane historyczne jest ważne w kontekście projektowanego systemu.

Ma ona za zadanie zapewnić ciągłość procesów decyzyjnych. Archiwizacja danych umożliwia nowym użytkownikom szybkie zapoznanie się z procesem, jego historią oraz podejmowanymi wcześniej decyzjami. Dzięki dostępowi do danych historycznych możliwe jest zrozumienie kontekstu wcześniejszych działań, co znacząco skraca czas potrzebny na wdrożenie się do pracy z systemem.

Zarchiwizowane dane stanowią podstawę do tworzenia raportów dotyczących budżetu oraz kosztów usług IT na przestrzeni lat. Analiza trendów, porównanie wyników oraz identyfikacja obszarów wymagających optymalizacji stają się możliwe dzięki dostępowi do historycznych informacji.

Dane archiwalne pozwalają na obiektywne podejmowanie decyzji, opartych na analizie wcześniejszych wyników. Dzięki temu użytkownicy mogą unikać powtarzania błędów oraz skutecznie przewidywać potencjalne konsekwencje podejmowanych działań.
\end{comment}
\subsection{Interfejs przyjazny dla użytkownika}
\begin{comment}Dedykowane narzędzie z prostym i intuicyjnym interfejsem znacząco ułatwia nawigację po bazie danych, eliminując problemy związane z tradycyjnymi rozwiązaniami, takimi jak arkusze kalkulacyjne. \end{comment} 

Dzięki dedykowanemu narzędziu z prostym i intuicyjnym interfejsem, nawigacja po bazie danych jest znacznie łatwiejsza, a problemy związane z używaniem arkuszy kalkulacyjnych zostają wyeliminowane. Przyjazny dla użytkownika interfejs oznacza:
\begin{itemize}
    \item \textbf{Prostotę:} nieskomplikowany układ umożliwia szybkie odnalezienie potrzebnych informacji.
    \item \textbf{Przejrzystość:} dane są zaprezentowane w sposób czytelny, z jasno określonymi polami i etykietami.
    \item \textbf{Przydatne funkcje:} filtrowanie i wyszukiwanie danych wspierają efektywność pracy.
\end{itemize}
Klarowny układ i czytelność interfejsu pozwalają użytkownikowi skupić się na konkretnej usłudze, co minimalizuje ryzyko pomyłek, takich jak błędne interpretowanie danych lub wybór niewłaściwego wiersza.

Takie podejście nie tylko zwiększa efektywność pracy, ale również poprawia komfort użytkowników, dzięki czemu procesy związane z analizą i zarządzaniem danymi stają się bardziej zrozumiałe i mniej podatne na błędy.
\subsection{Użycie pakietu Microsoft 365}
Wykorzystanie platformy Power\footnote{Platforma Power (\english{Power Platform}) -- Składowa pakietu Microsoft 365. Zawiera ona takie programy jak Power Apps, Power Automate czy Power BI.} w połączeniu z Sharepoint, pozwala na utworzenie w pełni funkcjonalnego rozwiązania, zachowując spójność danych dzięki integracji poszczególnych składników pakietu.

Aby korzystanie z aplikacji było możliwe, użytkownicy muszą mieć dostęp do potrzebnych usług oraz licencje. W przypadku omawianego pakietu, każdy z pracowników, ma do niego dostęp. Pozwala to na uniknięcie dodatkowych kosztów. 

Niestety użyty pakiet, nie jest dostępny w najbardziej rozbudowanym wariancie. Wprowadza to pewne ograniczenia, ponieważ brakuje w nim oprogramowania do tworzenia i zarządzania rozbudowanymi bazami danych o złożonej strukturze (takie możliwości daje między innymi \emph{Microsoft Azure}).
Sharepoint pozwala jedynie na utworzenie prostej bazy danych opierającej się o wcześniej opisane listy.  Głównym problemem było ograniczenie związane z brakiem możliwości tworzenia relacji między kilkoma listami, co znacząco utrudniało zarządzanie danymi o złożonej strukturze.

\subsection{Optymalizacja}
Priorytetem implementowanego rozwiązania jest optymalizacja procesu. Oprócz oszczędności czasu poprzez wprowadzenie automatyzacji, ważne jest usprawnienie analizy i przetwarzania danych poprzez użytkowników.
\section{Koncepcja rozwiązania}

Niniejszy rozdział koncentruje się na omówieniu koncepcji rozwiązania problemu. Przed przystąpieniem do prac zdefiniowano plan postępowania, którego celem jest osiągnięcie zamierzonego rezultatu.
Pracę podzielono na cztery główne etapy:
\begin{itemize}
    \item utworzenie dedykowanej bazy danych,
    \item zrealizowanie mechanizmu do wgrywania załączników oraz ich przetwarzania,
    \item przygotowanie formularza do wypełniania danych na temat bieżącej indykacji,
    \item automatyzacja procesu generowania raportów.
\end{itemize}
\subsection{Baza danych}

Zdecydowano się na wykorzystanie list programu Sharepoint do przechowywania danych. Pomimo tego, że nie jest to dedykowane rozwiązanie bazodanowe, wybór ten wynika z wymogu integracji z istniejącą infrastrukturą.

\subsubsection*{Struktura bazy danych}
\label{Subsec: StrukturaBazyDanych}
\label{instruction-link}
W wyniku analizy danych historycznych zidentyfikowano elementy kluczowe dla procesu indykacji. Na tej podstawie zaprojektowano strukturę składającą się z trzech powiązanych ze sobą list:

\begin{itemize}
  \item \textbf{Lista usług} -- zawierająca podstawowe, niezmienne informacje o serwisach,
  \item \textbf{Lista kwot} -- przechowująca dane odnośnie cen i liczbie licencji, które zmieniają się raz do roku,
  \item \textbf{Lista indykacji} -- gromadząca informacje w obrębie jednej indykacji.
\end{itemize}
\subsubsection*{Atrybuty danych}
Na podstawie analizy wymagań oraz dotychczasowego procesu, zdefiniowano następujący zestaw atrybutów, które powinna zawierać baza danych:

\begin{multicols}{3}
  \begin{itemize}
    \item Service group
    \item Service main group
    \item Service sub group
    \item Business Service
    \item Instruction link
    \item ID
    \item Business Service Manager
    \item Unit Of Measurement
    \item Settlement Type
    \item Current Year Plan EUR
    \item Quantity Current Year
    \item Next Year Plan EUR
    \item Quantity Next Year
    \item Year
    \item MPK
    \item Difference
    \item Indication Number
    \item Comment Intern
    \item Comment Date
    \item Comment Author
    \item Comment PZ to WOB
    \item Comment BSM
    \item Comment K-DES
    \item Decision
    \item Final comment
  \end{itemize}
\end{multicols}
Powyższy zestaw atrybutów został opracowany na podstawie analizy danych historycznych z poprzednich lat (przedstawionych w Tabeli \ref{HeaderComparison}). Wybrane pola reprezentują najczęściej występujące informacje w procesie indykacji, uzupełnione o dodatkowe atrybuty niezbędne do efektywnego funkcjonowania procesu, takie jak pola komentarzy czy decyzji.

\subsubsection*{Model powiązań}

\begin{figure}[h]
  \centering
  \includegraphics[width=0.9\textwidth]{figures/Diagram.png}
  \caption{Schemat relacji między listami}
  \label{SchematList}
\end{figure}

% \begin{figure}[h]
%   \makebox[0.925\textwidth][c]{
%     \begin{tikzpicture}

%       \tikzstyle{ListBlock} = [
%       rectangle,
%       rounded corners,
%       minimum width=3cm,
%       minimum height=1cm,
%       text centered,
%       draw=black,
%       fill=white,
%       anchor=north
%       ]

%       \node (ListaUslug) [ListBlock, text width=3cm] at (0,0) {
%         \textbf{Lista Usług}\\[4pt]
%         \emph{Service ID}\\[2pt]
%         Service Name\\[2pt]
%         Service Group\\[2pt]
%         Service Sub Group\\[2pt]
%         Service Main Group\\[2pt]
%         Instruction Link\\[2pt]
%       };

%       \node (ListaKwot) [ListBlock, text width=5cm]
%       at ([xshift=5cm] ListaUslug.north east) {
%         \textbf{Lista Kwot}\\[4pt]
%         \emph{Service ID}\\[2pt]
%         \emph{Year}\\[2pt]
%         Unit Of Measurement\\[2pt]
%         Settlement Type\\[2pt]
%         Business Service Manager\\[2pt]
%         Current Year Plan EUR\\[2pt]
%         QTY Current Year\\[2pt]
%         Next Year Plan EUR\\[2pt]
%         QTY Next Year\\[2pt]
%         Difference\\[2pt]
%         MPK\\[2pt]
%       };

%       \node (ListaIndykacji) [ListBlock, text width=3cm]
%       at ([xshift=4cm] ListaKwot.north east) {
%         \textbf{Lista Indykacji}\\[4pt]
%         \emph{Service ID}\\[2pt]
%         \emph{Year}\\[2pt]
%         Indication No.\\[2pt]
%         Comment Date\\[2pt]
%         Comment Author\\[2pt]
%         Comment PZ to WOB\\[2pt]
%         Comment BSM\\[2pt]
%         Comment K-DES\\[2pt]
%         Comment Intern\\[2pt]
%         Decision\\[2pt]
%         Final comment\\[2pt]
%       };

%       \draw [<->] ([yshift=-2em-4pt]ListaUslug.north east) -- node[anchor=south]{\emph{Service ID}} ([yshift=-2em-4pt]ListaKwot.north west);
%       \draw [<->] ([yshift=-2em-4pt]ListaKwot.north east) -- node[anchor=south]{\emph{Service ID}} ([yshift=-2em-4pt]ListaIndykacji.north west);
%       \draw [<->] ([yshift=-3.5em-4pt]ListaKwot.north east) -- node[anchor=north]{\emph{Year}} ([yshift=-3.5em-4pt]ListaIndykacji.north west);

%     \end{tikzpicture}}
%   \caption{Schemat relacji między listami.}
%   \label{SchematList}
% \end{figure}

Model danych przedstawiony na Rysunku \ref{SchematList} został zaprojektowany z uwzględnieniem następujących założeń:

\begin{itemize}
  \item \emph{Lista usług} będzie pełnić rolę centralnego rejestru serwisów, zawierając ich podstawową charakterystykę,
  \item \emph{Lista kwot} umożliwi śledzenie zmian w wymiarze finansowym na przestrzeni lat,
  \item \emph{Lista indykacji} ma za zadanie przechowywać historię procesu decyzyjnego wraz z towarzyszącymi komentarzami i ustaleniami.
\end{itemize}




\subsection{Dodawanie informacji do bazy danych}

\begin{comment}

% likwidacja czasu przeszlego

Po ustaleniu struktury danych wykorzystywanych przez system, kolejnym etapem jest określenie sposobu importu informacji z arkuszy kalkulacyjnych do bazy danych.

Głównym wyzwaniem okazał się brak systematycznej organizacji danych w arkuszach programu Excel, co skutkowało niekompatybilnością z zaprojektowaną bazą danych.
W celu rozwiązania tego problemu, opracowano dedykowany interfejs w aplikacji, który wspomaga użytkownika w procesie przetwarzania danych, minimalizując ryzyko wystąpienia błędów.
Proces rozpoczyna się od tymczasowego umieszczenia pliku Excel w folderze współdzielonym na platformie SharePoint. Podczas implementacji napotkano problem z widocznością danych -- większość systemów może odczytać jedynie informacje zorganizowane w \emph{tabele programu Excel}\footnote{Tabela w programie Excel wymaga osobnej deklaracji poprzez zaznaczenie zakresu komórek i wybór opcji \emph{Narzędzia główne}$\to$\emph{Formatuj jako tabelę}}.

Rozwiązaniem okazało się zastosowanie skryptu pakietu Office, działającego bezpośrednio w arkuszu. Skrypt ten automatycznie tworzy tabelę o dynamicznym rozmiarze oraz usuwa puste kolumny. Zaimplementowany algorytm skutecznie identyfikuje początek tabeli oraz określa jej wymiary na podstawie liczby wierszy i kolumn, pomijając nieistotne dane.

W celu dostosowania danych do struktury bazy, zaimplementowano formularz walidacyjny dla nazw kolumn. System pobiera nazwy istniejących kolumn z arkusza i umożliwia ich mapowanie na predefiniowaną listę nagłówków z list SharePoint. Proces ten wykorzystuje wspomniany wcześniej skrypt do pobrania aktualnych nazw.

Po uporządkowaniu struktury, użytkownik określa rok oraz numer indykacji dla importowanego arkusza. Następnie dane są przekazywane do \emph{flow} w programie \emph{Power Automate}, który przypisuje je do odpowiednich list w bazie danych, jednocześnie zapobiegając duplikacji rekordów.

Interfejs został dodatkowo wyposażony w formularz służący do przypisywania numerów \emph{MPK} nowym serwisom. Jest to kluczowy element, gdyż numer \emph{MPK} determinuje obszar odpowiedzialny za obsługę danej usługi. Dla serwisów występujących w poprzednich latach, system automatycznie przepisuje istniejące numery \emph{MPK}, redukując ilość danych do wprowadzenia. Jednocześnie zachowano możliwość modyfikacji wcześniej przypisanych numerów w razie potrzeby.

\end{comment}


Po ustaleniu struktury danych wykorzystywanych przez system, kolejnym etapem jest określenie sposobu importu informacji z arkuszy kalkulacyjnych do bazy danych. Postanowiono wykorzystać program Power Automate w celu automatyzacji tego procesu. Jednakże z uwagi na dużą rozbierzność danych wymaga on asysty użytkownika. 
\begin{comment}
\customnote{
    TO POLECI DO IMPLEMENTACJI

    Głównym wyzwaniem okazał się brak systematycznej organizacji danych w arkuszach programu Excel, co skutkowało niekompatybilnością z zaprojektowaną bazą danych.
    W celu rozwiązania tego problemu, opracowano dedykowany interfejs w aplikacji, który wspomaga użytkownika w procesie przetwarzania danych, minimalizując ryzyko wystąpienia błędów.
    Proces rozpoczyna się od tymczasowego umieszczenia pliku Excel w folderze współdzielonym na platformie SharePoint. Podczas implementacji napotkano problem z widocznością danych -- większość systemów może odczytać jedynie informacje zorganizowane w \emph{tabele programu Excel}\footnote{Tabela w programie Excel wymaga osobnej deklaracji poprzez zaznaczenie zakresu komórek i wybór opcji \emph{Narzędzia główne}$\to$\emph{Formatuj jako tabelę}}.

    Rozwiązaniem okazało się zastosowanie skryptu pakietu Office, działającego bezpośrednio w arkuszu. Skrypt ten automatycznie tworzy tabelę o dynamicznym rozmiarze oraz usuwa puste kolumny. Zaimplementowany algorytm skutecznie identyfikuje początek tabeli oraz określa jej wymiary na podstawie liczby wierszy i kolumn, pomijając nieistotne dane.}
\end{comment}

W celu dostosowania danych do struktury bazy, zaplanowano zaimplementowanie formularza walidacyjnego dla nazw kolumn. System pobiera nazwy istniejących kolumn z arkusza i umożliwia ich mapowanie z wykorzystaniem predefiniowanej listy nagłówków z list SharePoint.

Po uporządkowaniu struktury, użytkownik określa rok oraz numer indykacji dla importowanego arkusza. Następnie dane przekazywane są do \emph{flow} w programie \emph{Power Automate}, który przypisze je do odpowiednich list w bazie danych, jednocześnie zapobiegając duplikacji rekordów.

Interfejs jest dodatkowo wyposażony w formularz służący do przypisywania numerów \emph{MPK} nowym serwisom. Jest to kluczowy element, ponieważ numer \emph{MPK} determinuje obszar odpowiedzialny za obsługę danej usługi. Dla serwisów występujących w poprzednich latach, system automatycznie przypisuje istniejące numery \emph{MPK}, redukując ilość danych do wprowadzenia. Jednocześnie zachowana będzie możliwość modyfikacji wcześniej przypisanych numerów.


\subsection{Interfejs procesu decyzyjnego}

Interfejs obsługi procesu decyzyjnego został podzielony na dwa współpracujące ze sobą ekrany. Takie rozwiązanie pozwala na zachowanie przejrzystości prezentowanych informacji przy jednoczesnym zapewnieniu dostępu do wszystkich niezbędnych funkcjonalności.

\subsubsection{Ekran listy serwisów}
Pierwszy ekran pełni funkcję panelu nawigacyjnego, prezentując podstawowe informacje o serwisach:
\begin{itemize}
    \item Service Name -- nazwa serwisu,
    \item Service ID -- unikalny identyfikator,
    \item MPK -- numer księgowy obszaru odpowiedzialnego,
    \item Decision -- aktualny status decyzji.
\end{itemize}

Zaimplementowany system filtrowania umożliwia:
\begin{itemize}
    \item wyszukiwanie serwisów po ID lub nazwie,
    \item filtrowanie według przypisanych numerów MPK,
    \item segregację według statusu decyzji (\emph{Accepted}, \emph{Not Accepted}, \emph{No Status}).
\end{itemize}

\subsubsection{Ekran szczegółowy serwisu}
Po wybraniu serwisu z listy, użytkownik zostaje przekierowany do ekranu szczegółowego, który składa się z trzech głównych sekcji:

\paragraph{Sekcja historyczna}
Zawiera dwie uzupełniające się tabele:
\begin{itemize}
    \item tabela przeglądowa -- prezentująca historię decyzji z poprzednich lat,
    \item tabela szczegółowa -- wyświetlająca pełne informacje dla wybranego roku (domyślnie rok poprzedni).
\end{itemize}

\paragraph{Formularz decyzyjny}
Zestaw pól do wprowadzenia informacji o bieżącej indykacji:
\begin{itemize}
    \item rok (wartość domyślna: bieżący),
    \item numer indykacji (wartość domyślna: kolejny wolny numer),
    \item autor (wartość domyślna: zalogowany użytkownik),
    \item komentarze (wewnętrzny, BSM, K-DES),
    \item decyzja (wartość domyślna: poprzednia decyzja).
\end{itemize}

\paragraph{Panel podglądu}
W celu zwiększenia czytelności, długie komentarze (powyżej 30 znaków) są początkowo prezentowane w formie skróconej. Pełna treść komentarza jest dostępna po jego wybraniu i wyświetlana w dedykowanym panelu podglądu.

System automatycznego uzupełniania wartości domyślnych przyspiesza proces decyzyjny, jednocześnie zachowując możliwość modyfikacji każdego z parametrów w razie potrzeby.

\subsection{Generowanie raportu}

Ostatnim etapem cyklu obsługi procesu jest generowanie raportu, który będzie gotowy do odesłania w celu kontynuacji konsultacji z zakładem w Wolfsburgu. W ramach tego etapu zaplanowano rozwiązanie wykorzystujące dane przechowywane na listach sharepointowych, co umożliwi wydajne zarządzanie informacjami niezbędnymi do stworzenia raportu.

W dedykowanym oknie aplikacji użytkownik będzie miał możliwość wyboru odpowiedniego roku oraz etapu (numer indykacji). Po dokonaniu tych wyborów, system przetworzy wybrane kryteria, aby zgromadzić odpowiednie dane z różnych źródeł.

Kolekcja\footnote{Kolekcja (Power Apps) -- tymczasowy zbiór danych, przechowywanych lokalnie w aplikacji, umożliwiający zarządzanie rekordami podczas jej działania.} danych, która zostanie utworzona na podstawie wybranych kryteriów, będzie łączyć informacje z trzech różnych list sharepointowych. Dzięki temu możliwe będzie skonsolidowanie danych w jedną, spójną strukturę, która zawierać będzie wszystkie niezbędne informacje do sporządzenia raportu. Mechanizmy wyszukiwania umożliwią powiązanie identyfikatorów usług z odpowiadającymi im rekordami, co zapewni dokładność i kompletność zgromadzonych danych.

Dodatkowo, w tym samym oknie aplikacji użytkownik będzie miał dostęp do podglądu zgromadzonych danych w formie tabeli. Umożliwi to weryfikację poprawności i kompletności informacji przed finalnym wygenerowaniem raportu. Po zatwierdzeniu danych, system przekaże zgromadzoną kolekcję do Power Automate, gdzie zostanie przeprowadzona dalsza obróbka, niezbędna do przygotowania raportu w odpowiednim formacie (tj. w formacie arkusza kalkulacyjnego \textit{Excel}) do odesłania.


\chapter{Implementacja}

W niniejszym rozdziale przedstawiono szczegóły techniczne zaimplementowanego rozwiązania. Omówione zostaną kluczowe aspekty
implementacyjne systemu, obejmujące wykorzystanie platformy Microsoft Power Platform - w szczególności
Power Apps do budowy interfejsu użytkownika oraz Power Automate do automatyzacji procesów biznesowych.
Ponadto, przedstawiona zostanie integracja z platformą SharePoint oraz implementacja skryptów
usprawniających pracę z pakietem Microsoft Office. Rozdział stanowi techniczne rozwinięcie przyjętych
założeń projektowych, prezentując metodykę realizacji poszczególnych komponentów systemu.

W ramach analizy technicznej zostaną szczegółowo omówione poszczególne komponenty systemu oraz sposób
ich integracji. Szczególna uwaga zostanie poświęcona mechanizmom przepływu danych, automatyzacji
procesów oraz implementacji logiki biznesowej w środowisku low-code. Istotnym elementem będzie również
prezentacja zastosowanych rozwiązań w zakresie bezpieczeństwa danych oraz optymalizacji wydajności
w kontekście platformy Microsoft 365.

\section{Ekran dodawania danych}

Zdecydowano, że pierwszym ekranem aplikacji będzie ekran zapisu danych. Decyzja ta wynika z faktu, że bez przetworzonych danych, utworzenie innych ekranów byłoby zdecydowanie trudniejsze. Ekran ten składa się z elementów, które zostaną omówione poniżej.

\subsection{Zapis pliku w chmurze}
Pierwszym etapem procesu jest zapis pliku w chmurze, co umożliwia jego udostępnienie innym systemom. Do realizacji tego zadania wykorzystano kontrolkę\footnote{Kontrolka -- element służący do nawigacji, wyświetlania danych i obsługi aplikacji.} \emph{Attachment Control}. Pozwala ona na zapisanie pliku w pamięci aplikacji. Odbywa się to przez naciśnięcie przycisku \emph{"Dołącz plik"} lub przy użyciu mechaniki \emph{przeciągnij i upuść} (\english{Drag And Drop}). 

Aby przekazać plik oraz jego zawartość należy nacisnąć przycisk opisany jako \emph{Save attachments} znajdujący się pod wcześniej omawianym elementem. Naciśnięcie go skutkuje wywołaniem szeregu funkcji opisanych we właściwości \emph{OnSelect}. W pierwszej kolejności sprawdzane jest, czy plik został załadowany. Jeśli tak, to wywoływany jest przepływ \emph{SaveFileAndRunScript}. Wynik przepływu jest zapisywany w zmiennej tablicowej, która w Power Apps określana jest jako \definicja{kolekcja}, o nazwie \emph{FlowOutput}. Po wykonaniu się przepływu, pliki zapisane w pamięci aplikacji zostają usunięte.



\subsubsection{Przepływ SaveFileAndRunScript}
Rysunek \ref{fig:savefileandrunscript}, przedstawia edytor programu Power Automate. Widoczny w nim przepływ nazwany \emph{SaveFileAndRunScript} jest odpowiedzialny za zapisanie pliku w chmurze oraz wstępne przetworzenie. W momencie wywołania przepływu, plik jest przekazany jako parametr wejściowy. Przepływ ten składa się z kilku kroków, które zostaną omówione w kolejności ich wykonywania.

\begin{figure}[t]
    \centering
    \includegraphics[width=0.9\textwidth]{figures/SaveFileAndRunScript.png}
    \caption{Widok przepływu SaveFileAndRunScript}
    \label{fig:savefileandrunscript}
\end{figure}

\begin{enumerate}
    \item \textbf{Funkcja: Power Apps (V2)} \\
    Przepływ rozpoczyna się od funkcji wywoływanej bezpośrednio z aplikacji Power Apps. Jako parametry wejściowe przyjmuje:
    \begin{itemize}
        \item nazwę pliku (\textit{File Name}),
        \item zawartość pliku (\textit{File Content}) w formacie binarnym.
    \end{itemize}

    \item \textbf{Zainicjalizowanie zmiennej} \\
    Element \textit{Initialize variable} tworzy zmienną o nazwie \textit{FileExists}, która przechowuje informację, czy plik o podanej nazwie znajduje się już na SharePoint.

    \item \textbf{Sprawdzenie istniejących plików} \\
    Blok \textit{Get files} pobiera listę wszystkich plików z wybranego folderu SharePoint wraz z ich metadanymi, takimi jak nazwa, ścieżka czy data modyfikacji. Wynik zostaje zapisany w zmiennej \textit{FileExists}, która przyjmuje wartość \textit{true}, jeśli plik został znaleziony, lub \textit{false}, jeśli plik nie istnieje.

    \item \textbf{Instrukcja warunkowa \emph{If}} \\
    Element \textit{Condition} sprawdza wartość zmiennej \textit{FileExists}. W zależności od wyniku:
    \begin{itemize}
        \item jeśli zmienna ma wartość \textit{true} -- przepływ kończy działanie,
        \item jeśli zmienna ma wartość \textit{false} -- przepływ kontynuuje proces zapisu.
    \end{itemize}

    \item \textbf{Utworzenie pliku} \\
    Blok \textit{Create file} tworzy nowy plik w SharePoint, wykorzystując parametry:
    \begin{itemize}
        \item adres witryny SharePoint,
        \item ścieżkę do folderu docelowego,
        \item nazwę pliku,
        \item zawartość pliku.
    \end{itemize}

    \item \textbf{Uruchomienie przepływu podrzędnego} \\
    Po pomyślnym zapisaniu pliku przepływ wywołuje tzw. \textit{child flow}, który inicjuje działanie skryptu Office. Skrypt ten odpowiada za przetworzenie pliku w sposób zgodny z założeniami aplikacji. Jego wynik w formacie JSON jest zwracany do przepływu nadrzędnego. 

    \item \textbf{Odpowiedź do aplikacji} \\
    Blok \textit{Respond to Power Apps} kończy przepływ, zwracając do aplikacji dane w formacie JSON, przetworzone przez wspomniany skrypt.
\end{enumerate}

\subsection{Skrypt pakietu Office}
Po utworzeniu pliku w SharePoint, w ramach przepływu następuje jego przetworzenie przez skrypt. Jego zadaniem jest dostosowanie pliku do wymagań systemu. Poniżej przedstawiono kroki działania skryptu:

\begin{enumerate}
    \item \textbf{Wybór arkusza roboczego} \\
    Skrypt identyfikuje arkusz zawierający dane, analizuje zakres używanych komórek i usuwa ochronę hasłem, jeśli jest aktywna -- krok ten jest wymagany, aby wprowadzanie zmian w arkuszu było możliwe.

    \item \textbf{Analiza danych} \\
    Skrypt rozpoczyna analizę od wyszukiwania początku tabeli w arkuszu. Następnie:
    \begin{itemize}
        \item usuwa puste kolumny, które nie zawierają żadnych danych,
        \item tworzy tabelę o dynamicznym rozmiarze, uwzględniając zakres danych znajdujących się w arkuszu,
        \item uzupełnia brakujące komórki w kluczowych kolumnach, korzystając z danych w poprzednich wierszach.
    \end{itemize}
    Takie podejście pozwala na uporządkowanie danych i przygotowanie ich do dalszego przetwarzania.

    \item \textbf{Dopasowanie nazw kolumn} \\
    Skrypt porównuje istniejące nazwy kolumn z listą standardowych nagłówków, korzystając z algorytmu \textit{Jaro-Winkler}. Algorytm ten:
    \begin{itemize}
        \item analizuje podobieństwo tekstów, porównując wspólne znaki oraz ich kolejność,
        \item przyznaje dodatkowe punkty za zgodność początkowych znaków (prefiksu),
        \item zwraca wynik jako wartość z przedziału od 0 do 1, gdzie wartości bliższe 1 oznaczają większe podobieństwo.
    \end{itemize}
    Wynik tego procesu jest wykorzystywany w dalszych etapach aplikacji, m.in. do walidacji struktury danych. Jeśli podobieństwo jest mniejsze niż 90\%, skrypt sugeruje ręczne dopasowanie nazwy kolumny.

    \item \textbf{Zwrócenie wyników} \\
    Skrypt generuje JSON zawierający mapowanie oryginalnych nazw kolumn z najlepszymi dopasowaniami z listy standardowych nagłówków.
\end{enumerate}

Wprowadzenie przepływu podrzędnego było konieczne z uwagi na sposób, w jaki Power Automate obsługuje operacje na plikach w SharePoint. Gdy plik zostaje zapisany w folderze SharePoint, system przypisuje mu status wskazujący, czy jest gotowy do przetworzenia. W przypadku realizacji obu operacji (zapisu i przetwarzania pliku) w ramach jednego przepływu pojawiał się problem. Wynikał on z faktu, że przepływ pobierał dane z SharePoint już na etapie wstępnego sprawdzenia, czy plik o określonej nazwie istnieje. Informacja ta była przechowywana w pamięci przepływu i nie była aktualizowana w trakcie jego dalszego wykonywania.

W efekcie, po utworzeniu nowego pliku, przepływ nie miał możliwości odświeżenia informacji o jego istnieniu i statusie. To powodowało błąd uniemożliwiający uruchomienie skryptu, gdyż system informował, że plik, dla którego miał być wykonany, nie istnieje.

Rozwiązaniem tego problemu było wyodrębnienie etapu przetwarzania pliku do osobnego przepływu podrzędnego. Przepływ podrzędny, uruchamiany po zakończeniu procesu zapisu pliku, działał niezależnie i pobierał aktualne dane z SharePoint w momencie swojego wywołania. Dzięki temu możliwe było wyeliminowanie problemu braku odświeżonych informacji o statusie pliku, co pozwoliło na poprawne uruchomienie skryptu Office Script.

\textcolor{red}{LINK DO TEGO JARO\_WINKLERA: https://crucialbits.com/blog/a-comprehensive-list-of-similarity-search-algorithms/}

\vspace{1cm}

\begin{figure}[H]
    \centering
    \includegraphics[width=\textwidth]
    {figures/SaveAttachmentsForm.png}
    \caption{Formularz zapisu pliku w chmurze}      
    \label{fig:saveattachmentsform}
\end{figure}

Rysunek \ref{fig:saveattachmentsform} ilustruje opisane wcześniej elementy ekranu, na którym użytkownik może dodawać załączniki do procesu, podpisane jako \emph{"Add attachments to process"}. Obok znajduje się lista zapisanych plików, umożliwiająca wybór pliku do dalszego przetwarzania. Poniżej umieszczono przycisk \emph{"Click to open:..."}, który pozwala na otwarcie wybranego pliku w nowym oknie przeglądarki, co ułatwia jego weryfikację i podgląd.

\subsection{Walidacja nazw kolumn} Kolejnym etapem przed zapisaniem danych do bazy jest walidacja nazw kolumn. W tym celu zaimplementowano formularz, którego układ przedstawiono na rysunku \ref{fig:columnmappingform}. Formularz zawiera \emph{galerię} – element umożliwiający wyświetlanie wielu rekordów danych o różnych typach. Pola wyświetlające dane w galerii mogą być dostosowywane w dowolny sposób w zależności od potrzeb użytkownika.

\noindent Galeria składa się z dwóch kolumn: \begin{itemize} \item Lewa kolumna prezentuje obecne nazwy kolumn, które są wyświetlane za pomocą kontrolki \emph{Label}\footnote{\emph{Label} -- kontrolka tekstowa umożliwiająca wyświetlanie statycznych wartości.}. \item Prawa kolumna zawiera kontrolkę \emph{ComboBox}\footnote{\emph{ComboBox} -- rozwijana lista z możliwością wprowadzania tekstu.}, umożliwiającą wybór nazwy z listy standardowych nagłówków. Lista wartości w kontrolce \emph{ComboBox} jest generowana przez skrypt opisany w poprzednich sekcjach. \end{itemize}

Po prawej stronie formularza znajduje się instrukcja użytkownika, zawierająca wskazówki dotyczące prawidłowego uzupełniania nazw kolumn. Poniżej instrukcji umieszczono przycisk \emph{Update column names}, który umożliwia wprowadzenie zmian w strukturze danych.

\noindent Działanie tego mechanizmu opiera się na zastosowaniu skryptu pakietu Office. Skrypt jako parametr wejściowy przyjmuje zmienną tablicową w formacie JSON, zawierającą mapowanie oryginalnych nazw kolumn z poprawionymi wartościami wybranymi przez użytkownika. Następnie skrypt iteruje po wierszu zawierającym nagłówki kolumn i dokonuje ich zamiany zgodnie z mapowaniem. Po zakończeniu działania skrypt zwraca nową strukturę nazw kolumn.

Rysunek \ref{fig:columnmappingform} prezentuje wszystkie elementy formularza, w tym kontrolki umożliwiające wybór roku i numeru indykacji, które zostały umieszczone pod przyciskiem \emph{Update column names}. Kontrolki te, wraz z przyciskiem \emph{Upload data}, są kluczowe dla kolejnego etapu przetwarzania danych, obejmującego ich integrację z listami SharePoint. 


\begin{figure}[t]
    \centering
    \includegraphics[width=\textwidth]{figures/ColumnMappingForm.png}
    \caption{Formularz walidacji nazw kolumn}
    \label{fig:columnmappingform}
\end{figure}

\subsection{Integracja z listami SharePoint} Po zakończeniu walidacji nazw kolumn, kolejnym etapem jest integracja przetworzonych danych z listami SharePoint. Proces ten rozpoczyna się od wyboru roku i numeru indykacji przy użyciu dedykowanych kontrolek \emph{Dropdown}\footnote{\emph{Dropdown} -- kontrolka umożliwiająca wybór jednej z dostępnych wartości z rozwijanej listy, bez możliwości edycji.}. Wybrane wartości są następnie wykorzystywane podczas importu danych do odpowiednich list, co odbywa się za pomocą przycisku \emph{Upload data}. 

Skutki kliknięcia przycisku mogą się różnić w zależności od wybranych wartości i tego czy nazwy kolumny zostały zmienione. Rysunki \ref{fig:CorrectHeadersPopup} oraz \ref{fig:DoYouWantToOverwrite} przedstawiają dwa możliwe scenariusze, które mogą wystąpić po naciśnięciu przycisku. Pierwszy z nich występuje kiedy uprzednio nie został wciśnięty przycisk \emph{Update column names}. System upewnia się że użytkownik nie wgra przypadkowo danych z niepoprawnymi nagłówkami. Drugi natomiast pojawia sięw przypadku, gdy wybrane przez użytkownika rok oraz numer indykacji, istnieją w bazie danych. System pyta czy użytkownik chce nadpisać dane, które już tam się znajdują czy anulować operacje. W momencie kiedy nazwy kolumn nie zostaną zmienione oraz dane z wybranym rokiem i numerem indykacji istnieją w bazie danych, pojawiają się oba okna z informacjami.

\begin{figure}[htbp]
    \centering
    % Pierwszy obrazek
    \begin{minipage}{0.48\textwidth}
        \centering
        \includegraphics[width=\linewidth]{figures/CorrectHeadersPopup.png}
        \caption{Zapytanie o poprawność nazw kolumn}
        \label{fig:CorrectHeadersPopup}
    \end{minipage}\hfill
    % Drugi obrazek
    \begin{minipage}{0.48\textwidth}
        \centering
        \includegraphics[width=\linewidth]{figures/DoUWantToOverwrite.png}
        \caption{Zapytanie o nadpisanie danych}
        \label{fig:DoYouWantToOverwrite}
    \end{minipage}
    \label{fig:obrazki}
\end{figure}

Kiedy użytkownik upewni się, że wszystkie dane są poprawne i zatwierdzi operację, system przystępuje do importu danych. W tym celu wywołuje kolejny przepływ w programie Power Automate, który przypisuje informacji do odpowiednich list w bazie danych upewniając się jednocześnie, że nie zostaną dodane duplikaty rekordów. 

\noindent Przepływ ten jest bardzo rozbudowany, dlatego zamiast widoku edytora Power Automate, pokazany zostanie jego schemat blokowy na rysunku


% Zmodyfikowane style z dodaną czcionką \large
\tikzstyle{startstop} = [rectangle, rounded corners, minimum width=3cm, minimum height=1cm, text centered, line width=2pt, draw={rgb,255:red,116; green,39; blue,116}, fill={rgb,255:red,234; green,223; blue,234}]

\tikzstyle{processExcel} = [rectangle, minimum width=3cm, minimum height=1cm, text centered, line width=2pt, draw={rgb,255:red,16; green,124; blue,65}, fill=ForestGreen!80]

\tikzstyle{Variable} = [rectangle, minimum width=3cm, minimum height=1cm, text centered, line width=2pt, draw={rgb,255:red,119; green,11; blue,214}, fill={rgb,255:red,171; green,104; blue,230}]

\tikzstyle{Data} = [rectangle, minimum width=3cm, minimum height=1cm, text centered, line width=2pt, draw={rgb,255:red,140; green,108; blue,255}, fill={rgb,255:red,166; green,141; blue,255}]

\tikzstyle{SP} = [rectangle, minimum width=3cm, minimum height=1cm, text centered, line width=2pt, draw={rgb,255:red,3 ; green,108; blue,112}, fill={rgb,255:red,39; green,181; blue,194}, align=center]

\tikzstyle{decision} = [diamond, minimum width=3cm, minimum height=1cm, text centered, draw=black, fill=green!30]

\tikzstyle{decision} = [diamond, minimum width=3cm, minimum height=1cm, text centered, draw=black, fill=green!30]
\tikzstyle{arrow} = [thick,->,>=stealth]
\tikzstyle{data} = [parallelogram, minimum width=3cm, minimum height=1cm, text centered, draw=black, fill=yellow!30]

\resizebox{0.9\textwidth}{!}{%
\begin{tikzpicture}[node distance=3cm]
   
% Start node
\node (start) [startstop, text width=8cm, align=center] {\textbf{Trigger: Power Apps} \\[4pt]
\begin{tabular}{@{}cc@{}}
    FileName (string) & Year (number) \\
    IndicationNo (number) & Overwrite (bool) \\
\end{tabular}};

% Process nodes
\node (getTables) [processExcel, below of=start, fill=ForestGreen!80, text width=6cm, align=center, yshift=0.5cm] {\textbf{Get Tables} \\[4pt]
(from Excel with name \textit{FileName})};

\node (initializeVars) [Variable, align=center, text width=12cm, below of=getTables, yshift=-0.5cm] {
\textbf{Initialize Variables:}\\[4pt]
\begin{tabular}{@{}ll@{}}
- Excel Table (string) & - ItemsAddedToListaUslug (string) \\
- ItemsAddedToListaKwot (string) & - ItemsAddedToListaIndykacji (string) \\
- BatchRequestHeader (string) & - EndOfBatchRequest (string) \\
- Errors (string) \\
\end{tabular}
};


% Loop
% Twój niestandardowy bloczek jako węzeł
\node (applyToEach) [draw=none, inner sep=0pt, below of=initializeVars, below of=initializeVars, yshift=1.5cm] {%
    \begin{tikzpicture}[baseline]
        \draw [fill={rgb,255:red,152; green,171; blue,193}, draw={rgb,255:red,72; green,105; blue,145}, line width=2pt, dashed] (0,0) rectangle (5.5,-3.5);
        \draw [Variable, text width=4.5cm] (0.25,-1.5) rectangle node {\normalsize \textbf{Set Variable} \\[4pt]
        ExcelFile = TableContent} (5.25,-2.75);
        \node [font=\normalsize] at (2.75,-0.5) {\textbf{Apply to each}};
        \draw [processExcel] (0.25,-0.75) rectangle node {\normalsize \textbf{List rows present in table}} (5.25,-1.5);
    \end{tikzpicture}
};

% Select Columns
\node (selectCols) [Data, below of=applyToEach, text width=6cm, align=center] {\textbf{Select Columns from Excel}};

\node (Condition1) [rectangle, minimum width=3cm, minimum height=1cm, text centered, line width=2pt, draw={rgb,255:red,72; green,79; blue,88}, fill={rgb,255:red,149; green,153; blue,158},below of=selectCols, yshift=1cm, text width = 4cm] {\textbf{Condition 1}\\[4pt] \textit{Overwrite} == false};

\coordinate (belowCondition1) at ($(Condition1.south) - (0,2cm)$);


\node (ComplexNode) [draw=none, inner sep=0pt, anchor=north] at (Condition1.south) {%
    \begin{tikzpicture}[baseline]
\draw [ color={rgb,255:red,251; green,137; blue,129} , fill={rgb,255:red,254; green,237; blue,236}, line width=1pt , dashed] (2,23.75) rectangle  node {}  (14,-3); %false
\draw [ color={rgb,255:red,136; green,218; blue,141} , fill={rgb,255:red,237; green,249; blue,238}, line width=1pt , dashed] (-10.25,23.75) rectangle  node {}  (2,5.5); %true
\node (true1) [SP, dashed, minimum width=6cm, text width=5cm, draw=RedOrange] at (-4,22.5) {\normalsize \emph{*Only for "Lista Kwot"*}\\ \textbf{Get Previous Year MPK}};
\node (true2) [Data, minimum width=6cm, text width=5cm,dashed, draw=RedOrange] at (-4,21) {\normalsize \textbf{Select "MPK" column}};
\node (true3) [Data, minimum width=6cm, text width=5cm,dashed, draw=RedOrange] at (-4,19.5) {\normalsize \textbf{MPK Object}};
\node (true4) [SP, minimum width=6cm, text width=5cm, text width=5cm] at (-4,18) {\normalsize \textbf{Get "Lista Usług/Kwot/Indykacji"}};
\node (true5) [Data, minimum width=6cm, text width=5cm] at (-4,16.5) {\normalsize \textbf{Create Filter Key}};
\node (true6) [Data, minimum width=6cm, text width=5cm] at (-4,15) {\normalsize \textbf{Filter array}};

\draw [ color={rgb,255:red,136; green,218; blue,141} , fill={rgb,255:red,237; green,249; blue,238}, line width=1pt , dashed] (-9.75,12.5) rectangle  node {}  (-4.28,6);
\node (true11) [Data, minimum width=4cm, text width=3cm] at (-7,11) {\normalsize \textbf{Select columns to add to list}};
\node  (true12) [Data, minimum width=4cm] at (-7,9.5) {\normalsize \textbf{Join headers}};
\node  (true13) [SP, minimum width=4cm, text width=4cm] at (-7,8) {\normalsize \textbf{Send an HTTP request to Sharepoint}};
\draw [ color={rgb,255:red,251; green,137; blue,129} , fill={rgb,255:red,254; green,237; blue,236}, line width=1pt , dashed] (-4.24,12.5) rectangle  node {}  (1.5,6);
\node (Condition2) [rectangle, minimum width=3cm, minimum height=1cm, text centered, line width=2pt, draw={rgb,255:red,72; green,79; blue,88}, fill={rgb,255:red,149; green,153; blue,158}, minimum width=6cm, text width=5cm] at (-4,13.5) {\large \textbf{Condition 2}};
\node [font=\large] at (-4.25,23.25) {\textbf{If true}};
\node [font=\large] at (8,23.25) {\textbf{If false}};
\node [font=\large] at (-7,12) {\textbf{If true}};
\node [font=\large, text width=4cm, align=center] at (-1.25,12) {\textbf{If false}\\-Do nothing-};



\node (false1) [Data, minimum width=6cm, text width=5cm] at (8,22.5) { \textbf{batchTemplate}};
\node (false2) [SP, minimum width=6cm, text width=5cm] at (8,21) { \textbf{Get backend list name}};
\node (false3) [SP, minimum width=6cm, text width=5cm] at (8,19.5) { \textbf{Get list}};
\node (false4) [SP, minimum width=6cm, text width=5cm,dashed, draw=RedOrange] at (8,18) { \textbf{*only for lista kwot* \\ get prev year mpk}};
\node (false5) [Data, minimum width=6cm, text width=5cm,dashed, draw=RedOrange] at (8,16.5) { \textbf{Select MPK column}};
\node (false6) [Data, minimum width=6cm, text width=5cm,dashed, draw=RedOrange] at (8,15) { \textbf{MPK Object}};
\node (false7) [Data, minimum width=6cm, text width=5cm] at (8,13.5) {\textbf{Select SP ID}};
\node (false8) [Data, minimum width=6cm, text width=5cm] at (8,12) { \textbf{SP List create object}};
\node (false9) [Data, minimum width=6cm, text width=5cm] at (8,10.5) { \textbf{Select columns to update}};
\node (false10) [Data, minimum width=6cm, text width=5cm] at (8,9) { \textbf{Filter out null IDs}};
\node (false11) [Data, minimum width=6cm, text width=5cm] at (8,7.5) { \textbf{Replace Template Data}};
\node (false12) [Data, minimum width=6cm, text width=5cm] at (8,6) { \textbf{Batch Data}};
\node (false13) [SP, minimum width=6cm, text width=5cm] at (8,4.5) { \textbf{Send Batch Request}\\ Update Items};
\node (false14) [Variable, minimum width=6cm, text width=5cm] at (8,3) { \textbf{Assign result of send batch data to Error var}};

\draw [ color={rgb,255:red,251; green,137; blue,129} , fill={rgb,255:red,254; green,237; blue,236}, line width=1pt , dashed] (8,0.5) rectangle  node {}  (13.5,-2.5);
\draw [ color={rgb,255:red,136; green,218; blue,141} , fill={rgb,255:red,237; green,249; blue,238}, line width=1pt , dashed] (2.5,0.5) rectangle  node {}  (8,-2.5);
\node[Data, minimum width=4cm, draw={rgb,255:red,244; green,23; blue,0} , fill={rgb,255:red,253; green,220; blue,217}] at (5.25,-1) {\normalsize \textbf{Terminate}};
\node (Condition3) [rectangle, minimum width=3cm, minimum height=1cm, text centered, line width=2pt, draw={rgb,255:red,72; green,79; blue,88}, fill={rgb,255:red,149; green,153; blue,158}, minimum width=6cm, text width=5cm] at (8,1.5) {\large \textbf{Condition 3}};
    \node [font=\large] at (5.25,0) {\textbf{If true}};
\node [font=\large, text width=4cm, align=center] at (11.25,0) {\textbf{If false}\\-Do nothing-};

\node (stop) [startstop, minimum width=6cm, text width=5cm] at (2,-4.5) { \textbf{Send Batch Request}\\ Update Items};

%Arrows
\draw [arrow] (true1) -- (true2);
\draw [arrow] (true2) -- (true3);
\draw [arrow] (true3) -- (true4);
\draw [arrow] (true4) -- (true5);
\draw [arrow] (true5) -- (true6);
\draw [arrow] (true6) -- (Condition2);

\draw [arrow] (true11) -- (true12);
\draw [arrow] (true12) -- (true13);

\draw [arrow] (false1) -- (false2);
\draw [arrow] (false2) -- (false3);
\draw [arrow] (false3) -- (false4);
\draw [arrow] (false4) -- (false5);
\draw [arrow] (false5) -- (false6);
\draw [arrow] (false6) -- (false7);
\draw [arrow] (false7) -- (false8);
\draw [arrow] (false8) -- (false9);
\draw [arrow] (false9) -- (false10);
\draw [arrow] (false10) -- (false11);
\draw [arrow] (false11) -- (false12);
\draw [arrow] (false12) -- (false13);
\draw [arrow] (false13) -- (false14);
\draw [arrow] (false14) -- (Condition3);


\draw [arrow] (-4.125,5.5) -- +(0,-9) -- +(6.125,-9) --(stop);
\draw [arrow] (8,-3) -- +(0,-0.5) -- +(-6,-0.5) --(stop);

\end{tikzpicture}

};
\draw [arrow] (start) -- (getTables);
\draw [arrow] (getTables) -- (initializeVars);
\draw [arrow] (initializeVars) -- (applyToEach);
\draw [arrow] (applyToEach) -- (selectCols);
\draw [arrow] (selectCols) -- (Condition1);
\draw [arrow] (Condition1) -- (ComplexNode);

\end{tikzpicture}
}
\input{chapters/03-praca-wlasna}
\input{chapters/04-zakonczenie}


%--------------------------------------
% Literatura
%--------------------------------------

\bibliographystyle{plain}{\raggedright\sloppy\small\bibliography{bibliografia}}

%--------------------------------------
% Dodatki
%--------------------------------------

\cleardoublepage\appendix%
\newpage
\input{chapters/zalacznik.tex}

%--------------------------------------
% Informacja o prawach autorskich
%--------------------------------------

\ppcolophon

\end{document}
