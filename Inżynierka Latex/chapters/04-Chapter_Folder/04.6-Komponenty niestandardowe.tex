\section{Składniki}
Składniki to niestandardowe elementy, które można implementować w środowisku Power Apps. Są one wykorzystywane do tworzenia złożonych konfiguracji, które nie są dostępne w standardowych kontrolkach. Przydatne są szczególnie w momencie kiedy chcemy stworzyć element, który będzie wykorzystywany w wielu miejscach aplikacji. W zależności od potrzeb, można stworzyć zarówno proste elementy, jak i bardziej skomplikowane, które będą spełniały określone wymagania. Ponieżej omówiono dwa składniki stworzone w ramach aplikacji.

\subsection{Nagłówek ekranu}
Pierwszym niestandardowym komponentem jest wstążka znajdująca się na górze ekranów. Składa się ona z następujących elementów:
\begin{itemize}
    \item logo firmy -- jest to element dekoracyjny,
    \item przycisk powrotu do poprzedniego ekranu -- oznaczony strzałką w lewo, dzięki funkcji \textit{Back()} przenosi użytkownika do poprzedniego ekranu,
    \item przycisk powrotu do ekranu głównego -- reprezentowany poprzez ikonę domu, przenosi użytkownika do ekranu głównego aplikacji (\emph{Navigate(HomeScreen, ScreenTransition.Cover)}),
    \item dane użytkownika -- wyświetlane są dane zalogowanego użytkownika, takie jak imię, nazwisko oraz adres email. Pobierane są one przy pomocy \emph{Office365Users.MyProfile()}\footnote{\emph{Office365Users} -- konektor pozwalający na dostęp do listy zawierającej informacje na temat użytkowników takich jak imie, nazwisko, dane kontaktowe lub dział. Atrybut \emph{MyProfile} pozwala na dostęp do informacji o bieżącym użytkowniku.}.
    \item awatar użytkownika -- również pobierany z użyciem \emph{Office365Users.MyProfile()}. W przypadku braku zdjęcia, wyświetlane są inicjały użytkownika.
\end{itemize}



\begin{figure}[H]
    \centering
    \includegraphics[width=0.9\textwidth]{figures/HeaderComponent.png}
    \caption{Nagłówek ekranu}
    \label{fig:headercomponent}
\end{figure}


Warto zauważyć, że tytuły widoczne na ekranach, nie są integralną częścią wstążki. Wynika to z faktu, że po dodaniu kontrolki \emph{label}, pozwalającej na wyświetlenie tekstu, nie jest możliwe zmienienie jej  tekstu we właściwościach utworzonego składnika. W związku z tym, tytuły są dodawane na każdym ekranie osobno.


\subsection{Indykator ładowania}
Drugim niestandardowym składnikiem jest kontrolka infromaująca o ładowaniu się aplikacji. Składa się ona z trzech elementów:
\begin{itemize}
    \item koło ładowania -- animowana grafika w formacie \emph{SVG}(\definicja{Scalable Vector Graphics}).
    \item tekst -- informacja dla użytkownika, że aplikacja ładuje dane,
    \item tło -- półprzezroczyste tło z filtrem powodującym rozmycie elementów ekranu. Tło wykonane przy użyciu \emph{tekst HTML}.
\end{itemize}

\begin{figure}[H]
    \centering
    \includegraphics[width=0.9\textwidth]{figures/SpinnerComponent.png}
    \caption{Kontrolka ładowania}
    \label{fig:spinnercomponent}
\end{figure}

