\section{Ekran startowy aplikacji i przygotowanie danych}
\label{relacje-list}
\sectionauthor{R. Wolniak}

Kiedy dane zostały dostosowane do działania systemu, przystąpiono do implementacji pozostałych części rozwiązania.
Kolejnym elementem jest ekran startowy aplikacji, zawierający przyciski, które przekierowują użytkownika do odpowiednich sekcji aplikacji.

Dodatkowo, podczas uruchomienia aplikacji, pobierane są dane z list SharePoint a następnie odpowiednio przetwarzane w celu płynnego wyświetlania ich w aplikacji. \\

W pierwszym kroku zmiennej \emph{varDownloadingData} przypisywana jest wartość \emph{true} za pomocą funkcji \emph{Set()}. Zmienna ta pełni kluczową rolę w zarządzaniu interfejsem użytkownika podczas procesu ładowania danych – aktywuje wskaźnik ładowania oraz blokuje możliwość wprowadzania zmian przez użytkownika, co zapobiega ewentualnym błędom wynikającym z prób modyfikacji danych w trakcie ich pobierania.

Następnie funkcja \emph{ClearCollect()} tworzy kolekcję \emph{colYears}, która zawiera pięć elementów reprezentujących zakres lat: od dwóch lat wstecz do dwóch lat naprzód. Analogicznie, tworzona jest kolekcja \emph{colNumbers}, zawierająca numery indykacji, które mogą być wykorzystywane w polach typu \emph{Dropdown}. Kolekcje te są niezbędne do budowy dynamicznego i responsywnego interfejsu użytkownika, umożliwiając łatwe zarządzanie danymi w aplikacji.

W kolejnym kroku, za pomocą funkcji \emph{Set()}, pobierane są informacje o aktualnie zalogowanym użytkowniku i przypisywane do zmiennej \emph{UserVar}. Informacje te mogą być wykorzystywane do personalizacji interfejsu użytkownika lub kontroli dostępu do poszczególnych funkcji aplikacji, w~zależności od uprawnień użytkownika.

Aplikacja tworzy lokalne kopie trzech list danych: \emph{Lista\_Usług}, \emph{Lista\_Kwot} oraz \emph{Lista\_Indykacji}, wykorzystując funkcję \emph{ClearCollect()}. Dane są pobierane w~partiach po 2000 rekordów, aby zoptymalizować wydajność i uniknąć przekroczenia limitów pamięciowych. Proces ten polega na iteracyjnym przetwarzaniu danych, gdzie dla każdej listy określany jest zakres identyfikatorów (ID) pochodzących z listy Sharepoint dla kolejnych partii, zaczynając od najmniejszego ID i zwiększając go o 2000. Każda partia jest filtrowana według tego zakresu i dodawana do odpowiedniej kolekcji lokalnej. Takie podejście pozwala na szybsze przetwarzanie danych i uniknięcie problemów przy pobieraniu dużych zestawów danych. Listing \ref{lst:Pobieranie_list} przedstawia kod odpowiedzialny za pobranie listy usług w~partiach. Proces wygląda podobnie dla pozostałych list.

\lstset{language=C,caption={Frogament kodu odpowiedzialny za pobieranie listy w partiach},label=lst:Pobieranie_list}
\begin{lstlisting}[language=PowerFx]
// 1. Pobieranie danych z Lista_Uslug w partiach
Clear(LocalServiceData); // Wyczyszczenie kolekcji LocalServiceData przed rozpoczęciem pobierania
ForAll(Sequence(Round((First(Sort(Lista_Uslug, 'Identyfikator (ID)', SortOrder.Descending)).'Identyfikator (ID)' - First(Sort(Lista_Uslug, 'Identyfikator (ID)', SortOrder.Ascending)).'Identyfikator (ID)') / 2000 + 1, 0), 1, 1),
    With({
        _firstID: First(Sort(Lista_Uslug, 'Identyfikator (ID)', SortOrder.Ascending)).'Identyfikator (ID)' + (ThisRecord.Value - 1) * 2000,
        _lastID: First(Sort(Lista_Uslug, 'Identyfikator (ID)', SortOrder.Ascending)).'Identyfikator (ID)' + ThisRecord.Value * 2000
    },
    Collect(LocalServiceData, Filter(Lista_Uslug, 'Identyfikator (ID)' >= _firstID && 'Identyfikator (ID)' < _lastID)))
);
\end{lstlisting}

Listing \ref{lst:MergedData} przedstawia kod odpowiedzialny za łączenie danych z trzech list w celu utworzenia kolekcji \emph{MergedData}. W tym celu zastosowano funkcję \emph{AddColumns()}, aby dodać dwie nowe kolumny: \emph{Kwoty} oraz \emph{Indykacje}. Funkcja \emph{With()} pozwala na uproszczenie złożonych obliczeń poprzez przypisanie wyników pośrednich do zmiennych, co zwiększa czytelność kodu. Dane w~kolumnach są wyodrębniane przy użyciu funkcji \emph{LookUp()} oraz \emph{Filter()}, które umożliwiają precyzyjne filtrowanie i wyszukiwanie rekordów na podstawie trzech kluczowych kryteriów: \emph{Service\_ID}, \emph{Year} oraz \emph{IndicationNo}.


\lstset{language=C,caption={Fragment kodu odpowiedzialny za łączenie trzech list},label=lst:MergedData}
\begin{lstlisting}[language=PowerFx]
// 4. Łączenie danych z Lista_Uslug, Lista_Kwot i Lista_Indykacji
ClearCollect(MergedData,
    AddColumns(LocalServiceData As ServiceRecord,
        Kwoty,
        With({
            MaxYearCostRecord: First(Sort(Filter(LocalCostData, Service_ID = ServiceRecord.Service_ID), Year, SortOrder.Descending))
        },
        If(IsBlank(MaxYearCostRecord), Blank(), LookUp(LocalCostData, Service_ID = ServiceRecord.Service_ID && Year = MaxYearCostRecord.Year))),
        Indykacje,
        With({
            MaxYearRecord: First(Sort(Filter(LocalIndicationsData, Service_ID = ServiceRecord.Service_ID), Year, SortOrder.Descending))
        },
        If(IsBlank(MaxYearRecord), Blank(), LookUp(LocalIndicationsData, Service_ID = ServiceRecord.Service_ID && Year = MaxYearRecord.Year && IndicationNo = MaxYearRecord.IndicationNo)))
    )
);

// Ustawienie zmiennej varDownloadingData na false, aby wskazać, że proces pobierania danych został zakończony
Set(varDownloadingData, false);
\end{lstlisting}

W efekcie, kolekcja \emph{MergedData} zawiera dane dla najnowszego roku i najwyższego numeru indykacji dla każdej usługi, co umożliwia prezentację aktualnych informacji w interfejsie użytkownika.

Na końcu procesu zmiennej \emph{varDownloadingData} przypisywana jest wartość \emph{false}, co sygnalizuje zakończenie pobierania danych i gotowość aplikacji do użytku. Lokalne kopie list (\emph{LocalServiceData}, \emph{LocalCostData}, \emph{LocalIndicationsData}) zostały utworzone w celu przyspieszenia działania mechanizmu filtrowania oraz zwiększenia efektywności podczas wyboru usług do edycji.




