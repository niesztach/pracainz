
\chapter{Podstawy teoretyczne}

%\newline\textcolor{orange}{
%Można coś dopisać/ poprawić to powyżej. Chciałem tutaj przedstawić zarys jak wygląda proces.
%myśle żeby poniżej w każdym akapicie opisać krok po kroku jak to wyglądało z użyciem exceli.
%Tylko trzeba to napisać sensownie i łądnie żeby nikt się nie zajebał w akcji i żeby uwzględnić
%wszystkie rzeczy. Tutaj chyba nie będziemy się rozwodzili na temat minusów tego rozwiązania.
%no i potem bym dał informacje na temat użytych technologii: typescript, sharepoint, power au-
%tomate i power apps. imo taka kolejność żeby odzwierciedlała trochę jaki jest proces w apce bo
%bedziesz mógł się odwołąć do poprzednich. np że automate zaciaga dane z SP i każdy wie czym
%jest już sharepoint i essa.}
\section{Struktura procesu}
Przedmiotem omawianego procesu jest podjęcie decyzji na tematu zakupu usług IT w zakładzie
Volkswagen Poznań. Polega on na wymianie uwag, dotyczących wcześniej używanego bądź nowego
oprogramowania, między oddziałem Volkswagen w Poznaniu a zakładem z siedzibą w Wolfsburgu.
W wyniku wymiany zdań zapada decyzja o zakupie lub rezygnacji z wybranego produktu. Pro-
cedura rozpoczyna się wraz z początkiemn czerwca i trwa do przełomu grudnia i stycznia. Podzielona zazwyczaj na cztery iteracje. Efektem
przedstawianych działań jest nabycie odpowiedniej ilości potrzebnych uprawnień licencyjnych. Przy
podejmowaniu decyzji kluczowymi aspektami są:
\begin{itemize}
    \item liczba użytkowników danego oprogramowania,
    \item cena zakupu w porównaniu z rokiem poprzednim,
    \item określenie czy dana usługa zostanie w pełni wykorzystana biorąc pod uwagę poprzednie
kryteria.
\end{itemize}
Dotychczas analiza i przetwarzanie danych odbywało się przy użyciu arkuszy kalkulacyjnych pro-
gramu Excel. Natomiast wymiana informacji pomiędzy jednostkami dokonywana była poprzez
wysyłanie wiadomości e-mail.
\subsection{Gromadzenie danych dotyczących ofert usługodawców}
Informacje na temat serwisów są zbierane na początku roku, przed rozpoczęciem cyklu procesu. W tym czasie, prowadzone są rozmowy między menadżerami odpowiedzialnymi za dane rozwiązanie (\akronim{BSM}, \english{Business Service Manager}) a firmami świadczącymi usługi, w celu otrzymania zaaktualizowanych wiadomości związanych z ich produktami. Na podstawie danych od usługodawców oraz menadżerów, powstaje arkusz, który jest przekazywany do zakładu w Poznaniu.
\subsection{Przygotowanie danych}
Otrzymany arkusz kalkulacyjny, zawiera tabelę o strukturze kolumn podobnej do tabeli \ref{Headers2022}. Brakuje w nim jednak informacji kluczowych do rozpoczęcia cyklu.
Dlatego pierwszym krokiem jest przygotowanie danych przez osobę nadzorującą proces ze strony oddziału w Poznaniu.
Jej zadaniem jest manualne przypisanie numeru określającego miejsce powstawania kosztów, wewnętrznie nazywanego \definicja{MPK}. Numer ten definiuje konkretną jednostkę należącą do obszaru IT, która decyduje o zakupie danego produktu. Ponadto, dodawana jest kolumna, w której znajduje się wyliczona różnica cen między rokiem obecnym a poprzednim, w celu określenia czy koszt wzrósł lub zmalał. Tak przetworzony plik zostaje umieszczony we wspólnej przestrzeni dyskowej, co umożliwia pozostałym uczestnikom procesu przystąpienie do analizy oraz dalszego przetwarzania zawartych w nim informacji.

\renewcommand{\arraystretch}{1.1} % Zwiększenie wysokości komórek
\begin{table}[H] % [H] - tabela dokładnie w tym miejscu
    \begin{adjustwidth}{-50pt}{-20pt}
        \centering
        \caption{Nagłówki kolumn z arkusza kalkulacyjnego z roku 2022}
        \label{Headers2022}
        \makebox[\textwidth][c]{%
            \begin{tabular}{*{3}{|m{1.1cm}}|w|m{0.4cm}|m{1.5cm}|m{1.75cm}|w|m{0.7cm}|w|m{0.7cm}|}
                \hline
                Service group & Service main group & Service sub group & Business Service & ID & Business Service Manager & Unit of Measurement & PL70 2022 PLAN EUR w KVA & QTY & PL71 2023 PLAN EUR w KVA & QTY \\ \hline
            \end{tabular}
        }
    \end{adjustwidth}
\end{table}

\subsection{Przebieg Iteracji}
W trakcie trwania iteracji rozpatrywane są kluczowe informacje takie jak:
\begin{itemize}
\item Nazwa usługi - \textcolor{orange}{nwm czy takie kluczowe},
\item ID - \textcolor{orange}{nwm czy takie kluczowe},
\item osoba zajmująca się daną usługą -- \emph{BSM} - \textcolor{orange}{nwm czy takie kluczowe},
\item \textcolor{red}{jakoś wyjaśnić Unit of Measurement xD},
\item decyzja podjęta w roku poprzednim.
\item cena oraz ilość użytkowników w roku obecnym,
\item cena oraz ilość użytkowników w roku przyszłym,
\end{itemize}
Po analizie i porównaniu danych z wcześniejszych lat, w arkuszu powstają kolejne kolumny. Ich struktura nie jest określona przez żaden standard, ale zazwyczaj zawierają one:
\begin{itemize}
\item Komentarz wewnętrzny,
\item Status,
\item Komentarz klienta.
\end{itemize}

\noindent\emph{Komentarz wewnętrzny} nie jest wymagany dla każdego serwisu. Jest on zapisywany w celu skonsultowania decyzji ze współpracownikami.\\ \emph{Status} określa wstępną, wymaganą decyzję (Zaakceptowany/Niezaakceptowany).\\ \emph{Komentarz klienta} zawiera uzasadnienie podjętej decyzji ze strony Volkswagen Poznań.\\Tak uzupełniony arkusz zostaje przekazany pośrednio przez zakład w Wolfsburgu, do zarządu firmy. \par
Kolejnym etapem jest analiza tych informacji przez wcześniej wymienione podmioty. Ich zadaniem jest konfrontacja podjętej decyzji. Dodawane są kolejne kolumny:
\begin{itemize}
    \item Komentarz BSM,
    \item Komentarz K-DES.
\end{itemize}

\noindent\emph{Komentarz BSM} jest to odpowiedź ze strony menadżera usługi.\\ \emph{Komentarz K-DES} \textcolor{red}{(tutaj by się przydało rozszyfrować co to K-DES z niemieckiego)} natomiast jest odpowiedzią międzynarodowego zarządu firmy.\par
Zaaktualizowany plik powraca do Volkswagen Poznań, rozpoczynając tym samym kolejną iterację procesu.







% Podsumowanie przebiegu proceso - nie tworzyłem nowego pliku na dwa zdania.
%\vspace{1cm}
%Jak wcześniej wspomniano, proces składa się zazwyczaj z czterech iteracji. Etapem kończącym cykl jest sporządzenie wymaganych dokumentów oraz faktur.
%To wydaje się już niepotrzebne. Duplikat tego co jest wyżej.

\section{Wykorzystane technologie}
Aby usprawnić przebieg procesu, zabiezpieczyć go przed błędami i usystematyzować, stworzona została aplikacja do jego obsługi. Głównym kryterium przy doborze technologii była powszechna dostępność do powstałego systemu wśród pracowników. Dlatego też zdecydowano się na wykorzystanie komponentów pakietu \emph{Office 365}. Pakiet ten jest bardzo rozbudowany i jest powszechnie używany w firmie Volkswagen. Zawiera on programy pozwalające na stworzenie kompletnego systemu bez konieczności dostępu do dodatkowych usług.
\subsection{Skrypty pakietu Office}
Skrypty pakietu Office umożliwiają automatyzację zadań w arkuszach kalkulacyjnych programu Excel. Jedną z dostępnych funkcji jest \emph{Action Recorder}, który pozwala na "nagranie" sekwencji kroków wykonanych przez użytkownika, a następnie przekształcenie ich na skrypt wielokrotnego użytku.\par
Skrypty pakietu Office są wyposażone w wbudowany \emph{edytor kodu} (\english{Code Editor}), oparty na języku \emph{TypeScript}, który jest rozszerzeniem \emph{JavaScript}. Pomimo tego, że edytor jest stosunkowo ograniczony, umożliwia stosowanie konstrukcji niedostępnych w \emph{Action Recorder}, takich jak instrukcje warunkowe czy pętle.\par
Dodatkowo, program Excel pozwala na zapis skryptu w skoroszycie. Oznacza to, że każdy użytkownik mający dostęp do pliku może również uruchomić kod powiązany z danym skoroszytem.
\subsection{SharePoint \texorpdfstring{\cite{maggierui_introduction_2024}}{}}
SharePoint to platforma należąca do pakietu Microsoft 365, umożliwiająca tworzenie aplikacji webowych, takich jak witryny i strony internetowe. Jej głównym celem jest usprawnienie współpracy zespołowej poprzez dostarczenie narzędzi do publikowania informacji i raportów, które mogą być skierowane do określonych grup odbiorców. \par
Jednym z kluczowych zastosowań SharePointa jest zarządzanie danymi. Platforma oferuje przestrzeń do przechowywania różnego rodzaju plików, dokumentów i informacji, pełniąc funkcję serwera danych. Dzięki dostępności wbudowanych konektorów\footnote{\emph{Konektor} (\english{connector}) -– moduł umożliwiający integrację aplikacji z usługami lub źródłami danych w celu wymiany informacji i synchronizacji systemów.} (\english{connectors}), umożliwia również wykorzystanie przechowywanych danych w procesie tworzenia aplikacji czy witryn. \par
Istotnym elementem środowiska SharePoint jest możliwość tworzenia list, często nazywanych \emph{listami sharepointowymi}. Listy te mogą być wykorzystywane jako proste bazy danych, które umożliwiają dynamiczne aktualizowanie i synchronizowanie danych w czasie rzeczywistym. \par
SharePoint oferuje zaawansowane zarządzanie uprawnieniami. Administratorzy mogą precyzyjnie definiować dostęp użytkowników do poszczególnych zasobów witryny co pozwala na skuteczne zabezpieczenie wrażliwych informacji. \par
Platforma jest silnie zintegrowana z innymi usługami pakietu Microsoft 365, takimi jak Teams, Outlook czy OneDrive. Dzięki temu użytkownicy mogą współdzielić dane, pracować nad nimi w czasie rzeczywistym i korzystać z jednego spójnego środowiska pracy. \par
Ważnym aspektem SharePointa jest możliwość dostosowania wyglądu i funkcjonalności witryn do potrzeb użytkowników. Personalizacja obejmuje m.in. konfigurację interfejsu, dodawanie aplikacji webowych czy tworzenie dedykowanych formularzy. \par
W kontekście współczesnych modeli pracy, takich jak praca hybrydowa czy zdalna, SharePoint oferuje wsparcie dla użytkowników korzystających z różnorodnych urządzeń. Dostęp do danych jest możliwy za pośrednictwem przeglądarki internetowej oraz aplikacji mobilnych.

\subsection{Power Automate}
Power Automate to narzędzie wchodzące w skład pakietu Microsoft 365, które umożliwia automatyzację procesów biznesowych (\akronim{RPA}, \english{Robotic Process Automation}). Pozwala ono na tworzenie przepływów pracy (\english{flows}), automatyzujących powtarzalne zadania i integrujących różne systemy, zwiększając efektywność procesów biznesowych.

Flow w Power Automate jest odpowiednikiem funkcji w standardowych językach programowania. Na przykład, przepływ może automatycznie wysyłać powiadomienia e-mail po aktualizacji rekordu w SharePoint. Różnica polega na tym, że jest ono tworzone w wizualnym środowisku Low-Code i działa na zasadzie logicznego ciągu akcji wyzwalanych po sobie przez określone instrukcje.

Za pomocą flow można tworzyć własne procesy, które przy odpowiedniej implementacji, dorównują tym znanym z pełnych środowisk kodowych pod względem logiki i efektywności. Do dyspozycji są instrukcje warunkowe, pętle, zmienne, operacje na danych czy integracje z API poprzez konektory.

\subsection{Power Apps}

Power Apps to środowisko Low-Code'owe, wchodzące w skład pakietu Office 365, które jest dedykowanym rozwiązaniem do tworzenia aplikacji biznesowych. Dzięki intuicyjnemu interfejsowi graficznemu daje możliwość prostej implementacji mechanizmu działania nawet przez osoby bez zaawansowanej wiedzy programistycznej. Jest ona zintegrowana z innymi usługami pakietu Office 365, takimi jak SharePoint czy Power Automate, co rozszerza możliwości stworzonych aplikacji.

Power Apps pozwala na stworzenie spersonalizowanej aplikacji, dostosowanej do motywu organizacji, a przy połączeniu z innymi serwisami daje możliwość tworzenia zaawansowanych rozwiązań, minimalizując przy tym czas potrzebny na ich zaimplementowanie.

Ekrany aplikacji, komponowane za pomocą tego rozwiązania, porównywalne są z tymi, które można stworzyć w standardowych środowiskach programistycznych (jak np. JavaScript czy .NET), jednak proces ich tworzenia jest prostszy, ze względu na obecność edytora wizualnego. Umożliwia on korzystanie z gotowych komponentów w aplikacji, takich jak przyciski, pola danych wejściowych, listy, tabele, grafiki etc.

Dodawanie elementów do ekranów aplikacji odbywa się poprzez przeciąganie ich z biblioteki i upuszczanie w wybranym miejscu. Każdy komponent, może zostać skonfigurowany według potrzeb użytkownika poprzez edycje \emph{właściwości}. Możemy określić między innymi wypełnienie czy pozycję \emph{X} i \emph{Y} na ekranie, ale niektóre obiekty mają też unikalne właściwości takie jak \emph{OnSelect} \footnote{OnSelect -- określa akcje, które zostaną wykonane po naciścięciu elementu} dla przycisku.


\newpage

\begin{figure}[h] % h - tu, t - góra, b - dół, p - strona dodatkowa
    \centering
    \includegraphics[width=\textwidth]{figures/PowerAppsOverview}
    \caption{Edytor Power Apps}
    \label{fig:PowerAppsEditorOverview}
\end{figure}
\textcolor{red}{LINK DO OBRAZKA I OPISU ELEMENTÓW: https://learn.microsoft.com/en-us/power-apps/maker/canvas-apps/power-apps-studio}

Rysunek \ref{fig:PowerAppsEditorOverview} przedstawia edytor programu. Zawiera on następujące elementy:
\begin{enumerate}
    \item \textbf{Pasek poleceń:} wyświetla inny zestaw poleceń w zależności od wybranego kontrolki.
    \item \textbf{Akcje aplikacji:} Opcje wyświetlania właściwości, dodawania komentarzy, sprawdzania błędów, udostępniania, podglądu, zapisu lub publikowania aplikacji.
    \item \textbf{Lista właściwości:} Lista właściwości wybranego obiektu.
    \item \textbf{Pasek formuł:} Tworzenie lub edycja formuły dla wybranej właściwości z użyciem jednej lub więcej funkcji.
    \item \textbf{Menu tworzenia aplikacji:} Panel wyboru umożliwiający przełączanie się między źródłami danych oraz wstawianie dodatkowych opcji.
    \item \textbf{Lista elementów aplikacji:} Pokazuje lementy obecne na ekranie w postaci drzewa.
    \item \textbf{Płótno/ekran:} Główne płótno do komponowania struktury aplikacji.
    \item \textbf{Panel właściwości:} Lista właściwości wybranego obiektu.
    \item \textbf{Ustawienia i wirtualny agent:} Ustawienia aplikacji lub uzyskanie pomocy od wirtualnego agenta.
    \item \textbf{Selektor ekranu:} Przełączanie się między różnymi ekranami w aplikacji.
    \item \textbf{Zmiana rozmiaru płótna:} Zmienianie rozmiaru wyświetlanego płótna podczas tworzenia aplikacji.
\end{enumerate}

















