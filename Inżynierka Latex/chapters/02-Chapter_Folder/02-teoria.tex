
\chapter{Podstawy teoretyczne -- Stan wiedzy}

Przedmiotem omawianego procesu jest podjęcie decyzji na tematu zakupu usług IT w zakładzie
Volkswagen Poznań. Polega on na wymianie uwag, dotyczących wcześniej używanego bądź nowego
oprogramowania, między oddziałem Volkswagen w Poznaniu a zakładem z siedzibą w Wolfsburgu.
W wyniku wymiany zdań zapada decyzja o zakupie lub rezygnacji z wybranego produktu. Pro-
cedura rozpoczyna się wraz z początkiemn czerwca i trwa do przełomu grudnia i stycznia. Podzielona zazwyczaj na cztery iteracje. Efektem
przedstawianych działań jest nabycie odpowiedniej ilości potrzebnych uprawnień licencyjnych. Przy
podejmowaniu decyzji kluczowymi aspektami są:
\begin{itemize}
    \item liczba użytkowników danego oprogramowania,
    \item cena zakupu w porównaniu z rokiem poprzednim,
    \item określenie czy dana usługa zostanie w pełni wykorzystana biorąc pod uwagę poprzednie
kryteria.
\end{itemize}
Dotychczas analiza i przetwarzanie danych odbywało się przy użyciu arkuszy kalkulacyjnych pro-
gramu Excel. Natomiast wymiana informacji pomiędzy jednostkami dokonywana była poprzez
wysyłanie wiadomości e-mail.
%\newline\textcolor{orange}{
%Można coś dopisać/ poprawić to powyżej. Chciałem tutaj przedstawić zarys jak wygląda proces.
%myśle żeby poniżej w każdym akapicie opisać krok po kroku jak to wyglądało z użyciem exceli.
%Tylko trzeba to napisać sensownie i łądnie żeby nikt się nie zajebał w akcji i żeby uwzględnić
%wszystkie rzeczy. Tutaj chyba nie będziemy się rozwodzili na temat minusów tego rozwiązania.
%no i potem bym dał informacje na temat użytych technologii: typescript, sharepoint, power au-
%tomate i power apps. imo taka kolejność żeby odzwierciedlała trochę jaki jest proces w apce bo
%bedziesz mógł się odwołąć do poprzednich. np że automate zaciaga dane z SP i każdy wie czym
%jest już sharepoint i essa.}
\section{Przebieg procesu}
Informacje na temat serwisów są zbierane na początku roku, przed rozpoczęciem cyklu procesu. W tym czasie, prowadzone są rozmowy między menadżerami odpowiedzialnymi za dane rozwiązanie (\akronim{BSM}, \english{Business Service Manager}) a firmami świadczącymi usługi, w celu otrzymania zaaktualizowanych wiadomości związanych z ich produktami. Na podstawie danych od usługodawców oraz menadżerów, powstaje arkusz, który jest przekazywany do zakładu w Poznaniu.
\subsection{Iteracja I}
Otrzymany arkusz kalkulacyjny, zawiera tabelę o strzukturze kolumn podobnej Tabeli \ref{HeaderComparison}. Pierwszym krokiem jest przygotowanie danych przez osobę nadzorującą proces ze strony odziały w Poznaniu. Jej zadaniem jest manualne przypisanie 

\renewcommand{\arraystretch}{1.1} % Zwiększenie wysokości komórek
\begin{table}[t] % [H] - tabela dokładnie w tym miejscu
    \begin{adjustwidth}{-50pt}{-20pt}
    \centering
    \caption{Nagłówki kolumn z arkusza kalkulacyjnego z roku}
    \label{HeaderComparison}
    \makebox[\textwidth][c]{%
        \begin{tabular}{*{3}{|m{1.1cm}}|w|m{0.4cm}|m{1.5cm}|m{1.75cm}|w|m{0.7cm}|w|m{0.7cm}|}
        \hline
        Service group & Service main group & Service sub group & Business Service & ID & Business Service Manager & Unit of Measurement & PL70 2022 PLAN EUR w KVA & QTY & PL71 2023 PLAN EUR w KVA & QTY \\ \hline
        \end{tabular}
    }
    \end{adjustwidth}
\end{table}

Najważniejsze informacje to:
\begin{itemize}
\item Nazwa usługi,
\item ID,
\item osoba zajmująca się saną usługą -- \emph{BSM},
\item \textcolor{red}{jakoś wyjaśnić Unit of Measurement xD},
\item cena oraz ilość użytkowników w roku obecnym,
\item cena oraz ilość użytkowników w roku przyszłym.
\end{itemize}







\renewcommand{\arraystretch}{1.5} % Zwiększenie wysokości komórek
\begin{table}[H] % [t] - tabela bliżej górnej krawędzi strony
    \centering
    \caption{}
    \label{HeaderComparison}
    \makebox[\textwidth][c]{%
        \begin{tabular}{|c|W|W|W|}
        \hline
         & \textbf{2022} & \textbf{2023} & \textbf{2024} \\ \hline
        \multirow{12}{*}{\rotatebox{90}{\parbox{4cm}{\centering \textbf{Nazwy kolumn na\\przestrzeni lat}}}}
        & Service group & Service group & Service group \\ \cline{2-4}
        & Service main group & Service main group & Service main group \\ \cline{2-4}
        & Service sub group & Service sub group & Service sub group \\ \cline{2-4}
        & Business Service & Business Service & Business Service \\ \cline{2-4}
        & ID & ID & ID \\ \cline{2-4}
        & Business Service Manager & Business Service Manager & Business Service Manager \\ \cline{2-4}
        & Unit of Measurement & Unit of Measurement & Resource Unit \\ \cline{2-4}
        &  & Settlementtype & Settlementtype \\ \cline{2-4}
        & PL71 2023 PLAN EUR w KVA & PL71 2023 PLAN EUR w KVA & PL72 2024 PLAN EUR w KVA \\ \cline{2-4}
        & QTY & QTY & QTY \\ \cline{2-4}
        & PL71 2023 PLAN EUR w KVA & PL72 2024 PLAN EUR w KVA & PL73 2025 PLAN EUR w KVA \\ \cline{2-4}
        & QTY & QTY & QTY \\ \hline
        \end{tabular}
    }
\end{table}


    
    
    
