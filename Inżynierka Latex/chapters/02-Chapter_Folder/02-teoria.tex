
\chapter{Podstawy teoretyczne -- Stan wiedzy}

Przedmiotem omawianego procesu jest podjęcie decyzji na tematu zakupu usług IT w zakładzie
Volkswagen Poznań. Polega on na wymianie uwag, dotyczących wcześniej używanego bądź nowego
oprogramowania, między oddziałem Volkswagen w Poznaniu a zakładem z siedzibą w Wolfsburgu.
W wyniku wymiany zdań zapada decyzja o zakupie lub rezygnacji z wybranego produktu. Pro-
cedura ta trwa przez cały rok kalendarzowy i podzielona zazwyczaj na cztery iteracje. Efektem
przedstawianych działań jest nabycie odpowiedniej ilości potrzebnych uprawnień licencyjnych. Przy
podejmowaniu decyzji kluczowymi aspektami są:
\begin{itemize}
    \item liczba użytkowników danego oprogramowania,
    \item cena zakupu w porównaniu z rokiem poprzednim,
    \item określenie czy dana usługa zostanie w pełni wykorzystana biorąc pod uwagę poprzednie
kryteria.
\end{itemize}
Dotychczas analiza i przetwarzanie danych odbywało się przy użyciu arkuszy kalkulacyjnych pro-
gramu Excel. Natomiast wymiana informacji pomiędzy jednostkami dokonywana była poprzez
wysyłanie wiadomości e-mail.
\newline\textcolor{orange}{
Można coś dopisać/ poprawić to powyżej. Chciałem tutaj przedstawić zarys jak wygląda proces.
myśle żeby poniżej w każdym akapicie opisać krok po kroku jak to wyglądało z użyciem exceli.
Tylko trzeba to napisać sensownie i łądnie żeby nikt się nie zajebał w akcji i żeby uwzględnić
wszystkie rzeczy. Tutaj chyba nie będziemy się rozwodzili na temat minusów tego rozwiązania.
no i potem bym dał informacje na temat użytych technologii: typescript, sharepoint, power au-
tomate i power apps. imo taka kolejność żeby odzwierciedlała trochę jaki jest proces w apce bo
bedziesz mógł się odwołąć do poprzednich. np że automate zaciaga dane z SP i każdy wie czym
jest już sharepoint i essa.}
\vspace{5cm}
\par Rozdział teoretyczny --- przegląd literatury naświetlający stan wiedzy na dany temat. 

Przegląd literatury naświetlający stan wiedzy na dany temat obejmuje rozdziały pisane na podstawie
literatury, której wykaz zamieszczany jest w części pracy pt.~\emph{Literatura} (lub inaczej \emph{Bibliografia},
\emph{Piśmiennictwo}). W tekście pracy muszą wystąpić odwołania do wszystkich pozycji zamieszczonych w
wykazie literatury. \textbf{Nie należy odnośników do literatury umieszczać w stopce strony.} Student jest
bezwzględnie zobowiązany do wskazywania źródeł pochodzenia informacji przedstawianych w pracy,
dotyczy to również rysunków, tabel, fragmentów kodu źródłowego programów itd. Należy także podać
adresy stron internetowych w przypadku źródeł pochodzących z Internetu.


