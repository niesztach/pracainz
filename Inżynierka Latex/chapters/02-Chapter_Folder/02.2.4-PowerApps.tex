\subsection{Power Apps}
Power Apps to środowisko typu Low-Code, wchodzące w skład pakietu Microsoft 365, które jest dedykowane do tworzenia aplikacji biznesowych. Dzięki intuicyjnemu interfejsowi graficznemu umożliwia łatwą implementację mechanizmów działania, nawet osobom bez zaawansowanej wiedzy programistycznej. Jest zintegrowane z innymi usługami pakietu Microsoft 365, takimi jak SharePoint czy Power Automate, co znacznie rozszerza możliwości tworzonych aplikacji \texorpdfstring{\cite{tapanm-msft_official_nodate}}{}.

Power Apps pozwala na stworzenie spersonalizowanej aplikacji, dostosowanej do motywu organizacji, a przy połączeniu z innymi serwisami daje możliwość tworzenia zaawansowanych rozwiązań, minimalizując przy tym czas potrzebny na ich zaimplementowanie.

Ekrany aplikacji, komponowane za pomocą tego rozwiązania, porównywalne są z tymi, które można stworzyć w standardowych środowiskach programistycznych (jak np. JavaScript czy .NET), jednak proces ich tworzenia jest prostszy, ze względu na obecność edytora wizualnego. Umożliwia on korzystanie z gotowych komponentów w aplikacji, takich jak przyciski, pola danych wejściowych, listy, tabele, grafiki etc.

Dodawanie elementów do ekranów aplikacji odbywa się poprzez przeciąganie ich z biblioteki i upuszczanie w wybranym miejscu. Każdy komponent może zostać skonfigurowany według potrzeb użytkownika poprzez edycje \emph{właściwości}. Możemy określić między innymi wypełnienie czy pozycję \emph{X} i \emph{Y} na ekranie, ale niektóre obiekty mają też unikalne właściwości takie jak \emph{OnSelect}\footnote{OnSelect -- określa akcje, które zostaną wykonane po naciścięciu elementu} dla przycisku.

\newpage

\begin{figure}[h]
    \centering
    \includegraphics[width=\textwidth]{figures/PowerAppsOverview.png}
    \caption{Edytor \cite{lancedmicrosoft_understand_2024} Power Apps}
    \label{fig:PowerAppsEditorOverview}
\end{figure}

Rysunek \ref{fig:PowerAppsEditorOverview} przedstawia edytor programu. Zawiera on następujące elementy:
\begin{enumerate}
    \item \textbf{Pasek poleceń:} wyświetla inny zestaw poleceń w zależności od wybranego kontrolki.
    \item \textbf{Akcje aplikacji:} Opcje wyświetlania właściwości, dodawania komentarzy, sprawdzania błędów, udostępniania, podglądu, zapisu lub publikowania aplikacji.
    \item \textbf{Lista właściwości:} Lista właściwości wybranego obiektu.
    \item \textbf{Pasek formuł:} Tworzenie lub edycja formuły dla wybranej właściwości z użyciem jednej lub więcej funkcji.
    \item \textbf{Menu tworzenia aplikacji:} Panel wyboru umożliwiający przełączanie się między źródłami danych oraz wstawianie dodatkowych opcji.
    \item \textbf{Lista elementów aplikacji:} Pokazuje elementy obecne na ekranie w postaci drzewa.
    \item \textbf{Płótno/ekran:} Główne płótno do komponowania struktury aplikacji.
    \item \textbf{Panel właściwości:} Lista właściwości wybranego obiektu.
    \item \textbf{Ustawienia i wirtualny agent:} Ustawienia aplikacji lub uzyskanie pomocy od wirtualnego agenta.
    \item \textbf{Selektor ekranu:} Przełączanie się między różnymi ekranami w aplikacji.
    \item \textbf{Zmiana rozmiaru płótna:} Zmienianie rozmiaru wyświetlanego płótna podczas tworzenia aplikacji.
\end{enumerate}
