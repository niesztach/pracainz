\section{Struktura procesu}
Przedmiotem omawianego procesu jest podjęcie decyzji na tematu zakupu usług IT w zakładzie
Volkswagen Poznań. Polega on na wymianie uwag, dotyczących wcześniej używanego bądź nowego
oprogramowania, między oddziałem Volkswagen w Poznaniu a zakładem z siedzibą w Wolfsburgu.
W wyniku wymiany zdań zapada decyzja o zakupie lub rezygnacji z wybranego produktu. Pro-
cedura rozpoczyna się wraz z początkiemn czerwca i trwa do przełomu grudnia i stycznia. Podzielona zazwyczaj na cztery iteracje. Efektem
przedstawianych działań jest nabycie odpowiedniej ilości potrzebnych uprawnień licencyjnych. Przy
podejmowaniu decyzji kluczowymi aspektami są:
\begin{itemize}
    \item liczba użytkowników danego oprogramowania,
    \item cena zakupu w porównaniu z rokiem poprzednim,
    \item określenie czy dana usługa zostanie w pełni wykorzystana biorąc pod uwagę poprzednie
kryteria.
\end{itemize}
Dotychczas analiza i przetwarzanie danych odbywało się przy użyciu arkuszy kalkulacyjnych pro-
gramu Excel. Natomiast wymiana informacji pomiędzy jednostkami dokonywana była poprzez
wysyłanie wiadomości e-mail.