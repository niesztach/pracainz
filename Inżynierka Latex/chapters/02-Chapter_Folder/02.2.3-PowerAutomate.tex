\subsection{Power Automate}

Power Automate to narzędzie wchodzące w skład pakietu Microsoft 365, które pozwala na automatyzację procesów biznesowych (tzw. \textit{RPA} -- z angielskiego robotic process automation), ograniczając potrzebę wykonywania powtarzalnych czynności. Umożliwia ono tworzenie przepływów pracy -- \emph{flow}, opartych na zdarzeniach, które mogą spajać ze sobą różne aplikacje i usługi, takie jak np. SharePoint, Outlook czy PowerApps.

\emph{Flow} w Power Automate można porównać do funkcji w standardowych językach programowania (jak np. w C czy Pythonie), przy czym jest ono tworzone w wizualnym środowisku Low-Code i działa na zasadzie logicznego ciągu akcji wyzwalanych po sobie przez określone zdarzenia.

Za pomocą \emph{flow} można tworzyć własne procesy, które pod względem logiki i efektywności dorównują tym znanym z pełnych środowisk kodowych. Do dyspozycji są instrukcje warunkowe, pętle, zmienne, operacje na danych czy integracje z API poprzez konektory\footnote{\emph{Konektor} (z ang. \emph{connector}) -- element umożliwiający integrację aplikacji z usługą, np. poprzez wymianę danych między nimi}.
