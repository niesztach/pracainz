\subsection{Przebieg Iteracji}
W trakcie trwania iteracji analizowane są kluczowe informacje, takie jak:
\begin{itemize}
\item jednostka miary (ang. \emph{Unit of Measurement}),
\item decyzja podjęta w roku poprzednim,
\item cena oraz liczba użytkowników w roku obecnym,
\item cena oraz liczba użytkowników w roku przyszłym.
\end{itemize}
Po analizie i porównaniu danych z wcześniejszych lat, w arkuszu powstają kolejne kolumny. Ich struktura nie jest określona przez żaden standard, ale zazwyczaj zawierają one:
\begin{itemize}
\item Komentarz wewnętrzny,
\item Status,
\item Komentarz klienta.
\end{itemize}

\noindent\emph{Komentarz wewnętrzny} nie jest wymagany dla każdego serwisu. Jest on zapisywany w celu skonsultowania decyzji ze współpracownikami.\\ \emph{Status} określa wstępną, wymaganą decyzję (Zaakceptowany/Niezaakceptowany).\\ \emph{Komentarz klienta} zawiera uzasadnienie podjętej decyzji ze strony Volkswagen Poznań.\\Tak uzupełniony arkusz zostaje przekazany pośrednio przez zakład w Wolfsburgu, do zarządu firmy. \par
Kolejnym etapem jest analiza tych informacji przez wcześniej wymienione podmioty. Ich zadaniem jest konfrontacja podjętej decyzji. Dodawane są kolejne kolumny:
\begin{itemize}
    \item Komentarz BSM,
    \item Komentarz K-DES.
\end{itemize}

\noindent\emph{Komentarz BSM} jest to odpowiedź ze strony menadżera usługi.\\ \emph{Komentarz K-DES} \textcolor{red}{(tutaj by się przydało rozszyfrować co to K-DES z niemieckiego)} natomiast jest odpowiedzią międzynarodowego zarządu firmy.\par
Zaaktualizowany plik powraca do Volkswagen Poznań, rozpoczynając tym samym kolejną iterację procesu.





