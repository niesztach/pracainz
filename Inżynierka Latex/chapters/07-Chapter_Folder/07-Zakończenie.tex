\chapter{Zakończenie}
\chapterauthor{R. Wolniak}

W niniejszej pracy omówiono projekt aplikacji usprawniającej proces decyzyjny w firmie Volkswagen Poznań. 

Celem pracy było stworzenie narzędzia, które pozwoli na efektywniejsze podejmowanie decyzji w oparciu o analizę danych. Wstępne założenia obejmowały zarówno analizę potrzeb firmy, jak i opracowanie odpowiednich algorytmów pozwalających na systematyzację danych, implementację aplikacji oraz zabezpieczenie jej w celu minimalizacji błędów ludzkich.

W trakcie realizacji projektu udało się osiągnąć wszystkie założone cele. Aplikacja jest w pełni funkcjonalna, jednakże jest ona aktualnie w fazie testów, przed wdrożeniem w środowisku produkcyjnym.

W pracy zawarto najważniejsze fragmenty kodu źródłowego i opisano wpływ poszczególnych elementów na poprawę efektywności poprzez automatyzację większości operacji. Dodatkowo zminimalizowano ilość informacji wprowadzanych przez użytkowników i usprawniono proces analizy danych poprzez implementacje czytelnego interfejsu.

Podsumowując, praca ta nie tylko spełniła wszystkie założone cele, ale również przyczyniła się do zwiększenia efektywności procesów decyzyjnych w firmie Volkswagen Poznań. 