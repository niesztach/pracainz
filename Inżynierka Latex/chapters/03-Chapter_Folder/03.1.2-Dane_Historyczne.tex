\subsection{Archiwizacja danych}
Utworzenie bazy danych zawierającej dane historyczne stanowi kluczowy element projektowanego systemu, mający na celu zapewnienie ciągłości procesów decyzyjnych oraz wsparcie użytkowników w analizie i podejmowaniu świadomych decyzji. 

Archiwizacja danych umożliwia nowym użytkownikom szybkie zapoznanie się z procesem, jego historią oraz podejmowanymi wcześniej decyzjami. Dzięki dostępowi do danych historycznych możliwe jest zrozumienie kontekstu wcześniejszych działań, co znacząco skraca czas potrzebny na wdrożenie się do pracy z systemem.

Zarchiwizowane dane stanowią podstawę do tworzenia raportów dotyczących budżetu oraz kosztów usług IT na przestrzeni lat. Analiza trendów, porównanie wyników oraz identyfikacja obszarów wymagających optymalizacji stają się możliwe dzięki dostępowi do historycznych informacji.

Dane archiwalne pozwalają na obiektywne podejmowanie decyzji, opartych na analizie wcześniejszych wyników. Dzięki temu użytkownicy mogą unikać powtarzania błędów oraz skutecznie przewidywać potencjalne konsekwencje podejmowanych działań.