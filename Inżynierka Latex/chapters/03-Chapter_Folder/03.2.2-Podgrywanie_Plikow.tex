\subsection{Dodawanie informacji do bazy danych}

\begin{comment}

% likwidacja czasu przeszlego

Po ustaleniu struktury danych wykorzystywanych przez system, kolejnym etapem jest określenie sposobu importu informacji z arkuszy kalkulacyjnych do bazy danych.

Głównym wyzwaniem okazał się brak systematycznej organizacji danych w arkuszach programu Excel, co skutkowało niekompatybilnością z zaprojektowaną bazą danych.
W celu rozwiązania tego problemu, opracowano dedykowany interfejs w aplikacji, który wspomaga użytkownika w procesie przetwarzania danych, minimalizując ryzyko wystąpienia błędów.
Proces rozpoczyna się od tymczasowego umieszczenia pliku Excel w folderze współdzielonym na platformie SharePoint. Podczas implementacji napotkano problem z widocznością danych -- większość systemów może odczytać jedynie informacje zorganizowane w \emph{tabele programu Excel}\footnote{Tabela w programie Excel wymaga osobnej deklaracji poprzez zaznaczenie zakresu komórek i wybór opcji \emph{Narzędzia główne}$\to$\emph{Formatuj jako tabelę}}.

Rozwiązaniem okazało się zastosowanie skryptu pakietu Office, działającego bezpośrednio w arkuszu. Skrypt ten automatycznie tworzy tabelę o dynamicznym rozmiarze oraz usuwa puste kolumny. Zaimplementowany algorytm skutecznie identyfikuje początek tabeli oraz określa jej wymiary na podstawie liczby wierszy i kolumn, pomijając nieistotne dane.

W celu dostosowania danych do struktury bazy, zaimplementowano formularz walidacyjny dla nazw kolumn. System pobiera nazwy istniejących kolumn z arkusza i umożliwia ich mapowanie na predefiniowaną listę nagłówków z list SharePoint. Proces ten wykorzystuje wspomniany wcześniej skrypt do pobrania aktualnych nazw.

Po uporządkowaniu struktury, użytkownik określa rok oraz numer indykacji dla importowanego arkusza. Następnie dane są przekazywane do \emph{flow} w programie \emph{Power Automate}, który przypisuje je do odpowiednich list w bazie danych, jednocześnie zapobiegając duplikacji rekordów.

Interfejs został dodatkowo wyposażony w formularz służący do przypisywania numerów \emph{MPK} nowym serwisom. Jest to kluczowy element, gdyż numer \emph{MPK} determinuje obszar odpowiedzialny za obsługę danej usługi. Dla serwisów występujących w poprzednich latach, system automatycznie przepisuje istniejące numery \emph{MPK}, redukując ilość danych do wprowadzenia. Jednocześnie zachowano możliwość modyfikacji wcześniej przypisanych numerów w razie potrzeby.

\end{comment}

\subsection{Dodawanie informacji do bazy danych}
Po ustaleniu struktury danych wykorzystywanych przez system, kolejnym etapem jest określenie sposobu importu informacji z arkuszy kalkulacyjnych do bazy danych.

Głównym wyzwaniem jest brak systematycznej organizacji danych w arkuszach programu Excel, co skutkuje niekompatybilnością z zaprojektowaną bazą danych. W celu rozwiązania tego problemu, planowane jest opracowanie dedykowanego interfejsu w aplikacji, który wspomoże użytkownika w procesie przetwarzania danych, minimalizując ryzyko wystąpienia błędów.

\customnote{
   TO POLECI DO IMPLEMENTACJI

   Głównym wyzwaniem okazał się brak systematycznej organizacji danych w arkuszach programu Excel, co skutkowało niekompatybilnością z zaprojektowaną bazą danych.
   W celu rozwiązania tego problemu, opracowano dedykowany interfejs w aplikacji, który wspomaga użytkownika w procesie przetwarzania danych, minimalizując ryzyko wystąpienia błędów.
   Proces rozpoczyna się od tymczasowego umieszczenia pliku Excel w folderze współdzielonym na platformie SharePoint. Podczas implementacji napotkano problem z widocznością danych -- większość systemów może odczytać jedynie informacje zorganizowane w \emph{tabele programu Excel}\footnote{Tabela w programie Excel wymaga osobnej deklaracji poprzez zaznaczenie zakresu komórek i wybór opcji \emph{Narzędzia główne}$\to$\emph{Formatuj jako tabelę}}.

   Rozwiązaniem okazało się zastosowanie skryptu pakietu Office, działającego bezpośrednio w arkuszu. Skrypt ten automatycznie tworzy tabelę o dynamicznym rozmiarze oraz usuwa puste kolumny. Zaimplementowany algorytm skutecznie identyfikuje początek tabeli oraz określa jej wymiary na podstawie liczby wierszy i kolumn, pomijając nieistotne dane.}


W celu dostosowania danych do struktury bazy, planowane jest zaimplementowanie formularza walidacyjnego dla nazw kolumn. System będzie pobierał nazwy istniejących kolumn z arkusza i umożliwiał ich mapowanie na predefiniowaną listę nagłówków z list SharePoint. Proces ten wykorzysta wspomniany wcześniej skrypt do pobrania aktualnych nazw.

Po uporządkowaniu struktury, użytkownik będzie określał rok oraz numer indykacji dla importowanego arkusza. Następnie dane będą przekazywane do \emph{flow} w programie \emph{Power Automate}, który przypisze je do odpowiednich list w bazie danych, jednocześnie zapobiegając duplikacji rekordów.

Interfejs zostanie dodatkowo wyposażony w formularz służący do przypisywania numerów \emph{MPK} nowym serwisom. Jest to kluczowy element, ponieważ numer \emph{MPK} determinuje obszar odpowiedzialny za obsługę danej usługi. Dla serwisów występujących w poprzednich latach, system automatycznie przepisze istniejące numery \emph{MPK}, redukując ilość danych do wprowadzenia. Jednocześnie zachowana będzie możliwość modyfikacji wcześniej przypisanych numerów w razie potrzeby.

