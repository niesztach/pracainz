\subsection{Dodawanie informacji do bazy danych}
Po ustaleniu struktury danych wykorzystywanych przez system, należało określić w jaki sposób informacje z arkuszy kalkulacyjnych będą umieszczane w bazie danych.

Głównym problemem na tym etapie jest brak systematycznej organizacji danych zawartych w arkuszach programu Excel co prowadzi do braku kompatybilności z zaprojektowaną bazą danych. 
Zdecydowano się na utworzenie dedykowanego ekranu w aplikacji, który odpowiada za poprawne przetworzenie danych przy asyście użytkownika w celu uniknięcia błędów.

Pierwszym krokiem jest tymczasowe umieszczenie pliku Excel w folderze znajdujacym się we wspólnej przestrzeni roboczej programu Sharepoint. Dzięki temu, dokument jest dostępny dla innych systemów wykorzystanych do jego przetwarzania. Niestety pomimo możliwości otwarcia pliku przez inne systemy, dane w nim zawarte nie były widoczne. Jak się okazało, większość systemów jest w stanie odczytać informacje pogrupowane w \emph{tabele programu Excel}\footnote{W programie Excel tabelę trzeba osobno zadeklarować np. ręcznie zaznaczając zakres komórek a następnie wybierając opcję \emph{Narzędzia główne}$\to$\emph{Fromatuj jako tabelę}}.

Do rozwiązania tego problemu użyto skryptu pakietu Office, który działa wewnątrz arkusza a co za tym idzie ma bezpośredni dostęp do wszystkich informacji w nim zawartych. Skrypt ten oprócz tworzenia tabeli o dynamicznym rozmiarze, usuwa puste kolumny, które czasem były obecne wśród danych. Utworzony algorytm jest w stanie określić początek tabeli oraz jej wielkość zależną od liczby wierszy oraz kolumn, bez uwzględniania zbędnych informacji.

W celu przystosowania informacji z pliku do bazy danych, zdecydowano się na zastosowanie formularza do walidacji nazw kolumn. Formularz ten pobiera nazwy istniejących kolumn w zapisanym arkuszu i pozwala przypisać do nich nazwy z predefiniowanej listy zawierającej nagłówki znajdujące się w listach Sharepoint. Aktualne nazwy również pobierane są przy pomocy wcześniej opisanego skryptu.

Po usystematyzowaniu struktury użytkownik wybiera rok oraz numer indykacji, której dotyczy wgrywany arkusz. Następnie wszystkie informacje są przekazywane do \emph{flow} programu \emph{Power Automate}, które przypisuje je do odpowiednich listach bazy danych upewniając się, że nie powstają duplikaty.

Zdecydowano, że opisywany ekran będzie posiadał dodatkowy formularz odpowiedzialny za przypisywanie numerów \emph{MPK} dla nowych serwisów. Jest to bardzo ważne ponieważ numer ten okeśla, który obszar zajmuje się rozpatrzeniem usługi. W przypadku usług rozpatrywanych w latach poprzednich numer ten jest przepisywany aby ograniczyć liczbę wypełnianych danych. Oczywiście jest możliwość edycji istniejącego wcześniej numery w razie potrzeby. 