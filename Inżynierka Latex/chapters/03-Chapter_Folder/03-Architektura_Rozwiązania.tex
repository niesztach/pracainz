\chapter{Architektura rozwiązania}

Niniejszy rozdział przedstawia architekturę rozwiązania, obejmującą zarówno założenia projektowe, jak i koncepcję opracowywanego systemu. Założenia projektowe określają podstawowe wymagania oraz wytyczne, stanowiąc punkt wyjścia dla opracowywanego rozwiązania. Natomiast koncepcja rozwiązania, uwzględniająca zasadnicze założenia, strukturę i logikę działania, stanowi podstawę implementacji rozwiązania.

\section{Założenia projektowe}
Założenia projektowe stanowią zbór wytycznych, które określają funkcjonalność oraz wymagania techniczne, tworzonego rozwiązania.




\subsection{Systematyzacja danych}
Jedną z kluczowych funkcji omawianej aplikacji jest systematyzacja danych. Arkusze kalkulacyjne przesyłane przez oddział w Wolfsburgu często nie posiadają ustalonej struktury, co znacząco utrudniało ich czytelność i wymagało od użytkowników dodatkowego czasu na analizę zawartych informacji.

Brak jednolitego formatu danych uniemożliwiał również stworzenie spójnej bazy, co ograniczało możliwość ich wykorzystania w systemach automatyzacji procesów biznesowych. Dzięki wdrożonemu rozwiązaniu możliwe jest ujednolicenie danych, co pozwala na ich efektywne zarządzanie i automatyczne przetwarzanie.
\subsection{Archiwizacja danych}
Utworzenie bazy danych gromadzącej informacje o wcześniejszych działaniach realizowanych w ramach projektowanego systemu stanowi istotny element zapewniający ciągłość procesów decyzyjnych. Dzięki systematycznej archiwizacji nowi użytkownicy mogą szybko zapoznać się z przebiegiem procedur i lepiej zrozumieć kontekst dotychczas podejmowanych decyzji. Dostęp do zasobów historycznych nie tylko skraca czas potrzebny na pełne wdrożenie w funkcjonowanie systemu, lecz także usprawnia przetwarzanie danych bieżących.

\begin{comment}
Utworzenie bazy danych zawierającej dane historyczne jest ważne w kontekście projektowanego systemu.

Ma ona za zadanie zapewnić ciągłość procesów decyzyjnych. Archiwizacja danych umożliwia nowym użytkownikom szybkie zapoznanie się z procesem, jego historią oraz podejmowanymi wcześniej decyzjami. Dzięki dostępowi do danych historycznych możliwe jest zrozumienie kontekstu wcześniejszych działań, co znacząco skraca czas potrzebny na wdrożenie się do pracy z systemem.

Zarchiwizowane dane stanowią podstawę do tworzenia raportów dotyczących budżetu oraz kosztów usług IT na przestrzeni lat. Analiza trendów, porównanie wyników oraz identyfikacja obszarów wymagających optymalizacji stają się możliwe dzięki dostępowi do historycznych informacji.

Dane archiwalne pozwalają na obiektywne podejmowanie decyzji, opartych na analizie wcześniejszych wyników. Dzięki temu użytkownicy mogą unikać powtarzania błędów oraz skutecznie przewidywać potencjalne konsekwencje podejmowanych działań.
\end{comment}
\subsection{Interfejs przyjazny dla użytkownika}
\begin{comment}Dedykowane narzędzie z prostym i intuicyjnym interfejsem znacząco ułatwia nawigację po bazie danych, eliminując problemy związane z tradycyjnymi rozwiązaniami, takimi jak arkusze kalkulacyjne. \end{comment} 

Dzięki dedykowanemu narzędziu z prostym i intuicyjnym interfejsem, nawigacja po bazie danych jest znacznie łatwiejsza, a problemy związane z używaniem arkuszy kalkulacyjnych zostają wyeliminowane. Przyjazny dla użytkownika interfejs oznacza:
\begin{itemize}
    \item \textbf{Prostotę:} nieskomplikowany układ umożliwia szybkie odnalezienie potrzebnych informacji.
    \item \textbf{Przejrzystość:} dane są zaprezentowane w sposób czytelny, z jasno określonymi polami i etykietami.
    \item \textbf{Przydatne funkcje:} filtrowanie i wyszukiwanie danych wspierają efektywność pracy.
\end{itemize}
Klarowny układ i czytelność interfejsu pozwalają użytkownikowi skupić się na konkretnej usłudze, co minimalizuje ryzyko pomyłek, takich jak błędne interpretowanie danych lub wybór niewłaściwego wiersza.

Takie podejście nie tylko zwiększa efektywność pracy, ale również poprawia komfort użytkowników, dzięki czemu procesy związane z analizą i zarządzaniem danymi stają się bardziej zrozumiałe i mniej podatne na błędy.
\subsection{Użycie pakietu Microsoft 365}
Wykorzystanie platformy Power\footnote{Platforma Power (\english{Power Platform}) -- Składowa pakietu Microsoft 365. Zawiera ona takie programy jak Power Apps, Power Automate czy Power BI.} w połączeniu z Sharepoint, pozwala na utworzenie w pełni funkcjonalnego rozwiązania, zachowując spójność danych dzięki integracji poszczególnych składników pakietu.

Aby korzystanie z aplikacji było możliwe, użytkownicy muszą mieć dostęp do potrzebnych usług oraz licencje. W przypadku omawianego pakietu, każdy z pracowników, ma do niego dostęp. Pozwala to na uniknięcie dodatkowych kosztów. 

Niestety użyty pakiet, nie jest dostępny w najbardziej rozbudowanym wariancie. Wprowadza to pewne ograniczenia, ponieważ brakuje w nim oprogramowania do tworzenia i zarządzania rozbudowanymi bazami danych o złożonej strukturze (takie możliwości daje między innymi \emph{Microsoft Azure}).
Sharepoint pozwala jedynie na utworzenie prostej bazy danych opierającej się o wcześniej opisane listy.  Głównym problemem było ograniczenie związane z brakiem możliwości tworzenia relacji między kilkoma listami, co znacząco utrudniało zarządzanie danymi o złożonej strukturze.

\subsection{Optymalizacja}
Priorytetem implementowanego rozwiązania jest optymalizacja procesu. Oprócz oszczędności czasu poprzez wprowadzenie automatyzacji, ważne jest usprawnienie analizy i przetwarzania danych poprzez użytkowników.
\section{Koncepcja rozwiązania}

Niniejszy rozdział koncentruje się na omówieniu koncepcji rozwiązania problemu. Przed przystąpieniem do prac zdefiniowano plan postępowania, którego celem jest osiągnięcie zamierzonego rezultatu.
Pracę podzielono na cztery główne etapy:
\begin{itemize}
    \item utworzenie dedykowanej bazy danych,
    \item zrealizowanie mechanizmu do wgrywania załączników oraz ich przetwarzania,
    \item przygotowanie formularza do wypełniania danych na temat bieżącej indykacji,
    \item automatyzacja procesu generowania raportów.
\end{itemize}
\subsection{Baza danych}

Zdecydowano się na wykorzystanie list programu Sharepoint do przechowywania danych. Pomimo tego, że nie jest to dedykowane rozwiązanie bazodanowe, wybór ten wynika z wymogu integracji z istniejącą infrastrukturą.

\subsubsection*{Struktura bazy danych}
\label{Subsec: StrukturaBazyDanych}
\label{instruction-link}
W wyniku analizy danych historycznych zidentyfikowano elementy kluczowe dla procesu indykacji. Na tej podstawie zaprojektowano strukturę składającą się z trzech powiązanych ze sobą list:

\begin{itemize}
  \item \textbf{Lista usług} -- zawierająca podstawowe, niezmienne informacje o serwisach,
  \item \textbf{Lista kwot} -- przechowująca dane odnośnie cen i liczbie licencji, które zmieniają się raz do roku,
  \item \textbf{Lista indykacji} -- gromadząca informacje w obrębie jednej indykacji.
\end{itemize}
\subsubsection*{Atrybuty danych}
Na podstawie analizy wymagań oraz dotychczasowego procesu, zdefiniowano następujący zestaw atrybutów, które powinna zawierać baza danych:

\begin{multicols}{3}
  \begin{itemize}
    \item Service group
    \item Service main group
    \item Service sub group
    \item Business Service
    \item Instruction link
    \item ID
    \item Business Service Manager
    \item Unit Of Measurement
    \item Settlement Type
    \item Current Year Plan EUR
    \item Quantity Current Year
    \item Next Year Plan EUR
    \item Quantity Next Year
    \item Year
    \item MPK
    \item Difference
    \item Indication Number
    \item Comment Intern
    \item Comment Date
    \item Comment Author
    \item Comment PZ to WOB
    \item Comment BSM
    \item Comment K-DES
    \item Decision
    \item Final comment
  \end{itemize}
\end{multicols}
Powyższy zestaw atrybutów został opracowany na podstawie analizy danych historycznych z poprzednich lat (przedstawionych w Tabeli \ref{HeaderComparison}). Wybrane pola reprezentują najczęściej występujące informacje w procesie indykacji, uzupełnione o dodatkowe atrybuty niezbędne do efektywnego funkcjonowania procesu, takie jak pola komentarzy czy decyzji.

\subsubsection*{Model powiązań}

\begin{figure}[h]
  \centering
  \includegraphics[width=0.9\textwidth]{figures/Diagram.png}
  \caption{Schemat relacji między listami}
  \label{SchematList}
\end{figure}

% \begin{figure}[h]
%   \makebox[0.925\textwidth][c]{
%     \begin{tikzpicture}

%       \tikzstyle{ListBlock} = [
%       rectangle,
%       rounded corners,
%       minimum width=3cm,
%       minimum height=1cm,
%       text centered,
%       draw=black,
%       fill=white,
%       anchor=north
%       ]

%       \node (ListaUslug) [ListBlock, text width=3cm] at (0,0) {
%         \textbf{Lista Usług}\\[4pt]
%         \emph{Service ID}\\[2pt]
%         Service Name\\[2pt]
%         Service Group\\[2pt]
%         Service Sub Group\\[2pt]
%         Service Main Group\\[2pt]
%         Instruction Link\\[2pt]
%       };

%       \node (ListaKwot) [ListBlock, text width=5cm]
%       at ([xshift=5cm] ListaUslug.north east) {
%         \textbf{Lista Kwot}\\[4pt]
%         \emph{Service ID}\\[2pt]
%         \emph{Year}\\[2pt]
%         Unit Of Measurement\\[2pt]
%         Settlement Type\\[2pt]
%         Business Service Manager\\[2pt]
%         Current Year Plan EUR\\[2pt]
%         QTY Current Year\\[2pt]
%         Next Year Plan EUR\\[2pt]
%         QTY Next Year\\[2pt]
%         Difference\\[2pt]
%         MPK\\[2pt]
%       };

%       \node (ListaIndykacji) [ListBlock, text width=3cm]
%       at ([xshift=4cm] ListaKwot.north east) {
%         \textbf{Lista Indykacji}\\[4pt]
%         \emph{Service ID}\\[2pt]
%         \emph{Year}\\[2pt]
%         Indication No.\\[2pt]
%         Comment Date\\[2pt]
%         Comment Author\\[2pt]
%         Comment PZ to WOB\\[2pt]
%         Comment BSM\\[2pt]
%         Comment K-DES\\[2pt]
%         Comment Intern\\[2pt]
%         Decision\\[2pt]
%         Final comment\\[2pt]
%       };

%       \draw [<->] ([yshift=-2em-4pt]ListaUslug.north east) -- node[anchor=south]{\emph{Service ID}} ([yshift=-2em-4pt]ListaKwot.north west);
%       \draw [<->] ([yshift=-2em-4pt]ListaKwot.north east) -- node[anchor=south]{\emph{Service ID}} ([yshift=-2em-4pt]ListaIndykacji.north west);
%       \draw [<->] ([yshift=-3.5em-4pt]ListaKwot.north east) -- node[anchor=north]{\emph{Year}} ([yshift=-3.5em-4pt]ListaIndykacji.north west);

%     \end{tikzpicture}}
%   \caption{Schemat relacji między listami.}
%   \label{SchematList}
% \end{figure}

Model danych przedstawiony na Rysunku \ref{SchematList} został zaprojektowany z uwzględnieniem następujących założeń:

\begin{itemize}
  \item \emph{Lista usług} będzie pełnić rolę centralnego rejestru serwisów, zawierając ich podstawową charakterystykę,
  \item \emph{Lista kwot} umożliwi śledzenie zmian w wymiarze finansowym na przestrzeni lat,
  \item \emph{Lista indykacji} ma za zadanie przechowywać historię procesu decyzyjnego wraz z towarzyszącymi komentarzami i ustaleniami.
\end{itemize}




\subsection{Dodawanie informacji do bazy danych}

\begin{comment}

% likwidacja czasu przeszlego

Po ustaleniu struktury danych wykorzystywanych przez system, kolejnym etapem jest określenie sposobu importu informacji z arkuszy kalkulacyjnych do bazy danych.

Głównym wyzwaniem okazał się brak systematycznej organizacji danych w arkuszach programu Excel, co skutkowało niekompatybilnością z zaprojektowaną bazą danych.
W celu rozwiązania tego problemu, opracowano dedykowany interfejs w aplikacji, który wspomaga użytkownika w procesie przetwarzania danych, minimalizując ryzyko wystąpienia błędów.
Proces rozpoczyna się od tymczasowego umieszczenia pliku Excel w folderze współdzielonym na platformie SharePoint. Podczas implementacji napotkano problem z widocznością danych -- większość systemów może odczytać jedynie informacje zorganizowane w \emph{tabele programu Excel}\footnote{Tabela w programie Excel wymaga osobnej deklaracji poprzez zaznaczenie zakresu komórek i wybór opcji \emph{Narzędzia główne}$\to$\emph{Formatuj jako tabelę}}.

Rozwiązaniem okazało się zastosowanie skryptu pakietu Office, działającego bezpośrednio w arkuszu. Skrypt ten automatycznie tworzy tabelę o dynamicznym rozmiarze oraz usuwa puste kolumny. Zaimplementowany algorytm skutecznie identyfikuje początek tabeli oraz określa jej wymiary na podstawie liczby wierszy i kolumn, pomijając nieistotne dane.

W celu dostosowania danych do struktury bazy, zaimplementowano formularz walidacyjny dla nazw kolumn. System pobiera nazwy istniejących kolumn z arkusza i umożliwia ich mapowanie na predefiniowaną listę nagłówków z list SharePoint. Proces ten wykorzystuje wspomniany wcześniej skrypt do pobrania aktualnych nazw.

Po uporządkowaniu struktury, użytkownik określa rok oraz numer indykacji dla importowanego arkusza. Następnie dane są przekazywane do \emph{flow} w programie \emph{Power Automate}, który przypisuje je do odpowiednich list w bazie danych, jednocześnie zapobiegając duplikacji rekordów.

Interfejs został dodatkowo wyposażony w formularz służący do przypisywania numerów \emph{MPK} nowym serwisom. Jest to kluczowy element, gdyż numer \emph{MPK} determinuje obszar odpowiedzialny za obsługę danej usługi. Dla serwisów występujących w poprzednich latach, system automatycznie przepisuje istniejące numery \emph{MPK}, redukując ilość danych do wprowadzenia. Jednocześnie zachowano możliwość modyfikacji wcześniej przypisanych numerów w razie potrzeby.

\end{comment}


Po ustaleniu struktury danych wykorzystywanych przez system, kolejnym etapem jest określenie sposobu importu informacji z arkuszy kalkulacyjnych do bazy danych. Postanowiono wykorzystać program Power Automate w celu automatyzacji tego procesu. Jednakże z uwagi na dużą rozbierzność danych wymaga on asysty użytkownika. 
\begin{comment}
\customnote{
    TO POLECI DO IMPLEMENTACJI

    Głównym wyzwaniem okazał się brak systematycznej organizacji danych w arkuszach programu Excel, co skutkowało niekompatybilnością z zaprojektowaną bazą danych.
    W celu rozwiązania tego problemu, opracowano dedykowany interfejs w aplikacji, który wspomaga użytkownika w procesie przetwarzania danych, minimalizując ryzyko wystąpienia błędów.
    Proces rozpoczyna się od tymczasowego umieszczenia pliku Excel w folderze współdzielonym na platformie SharePoint. Podczas implementacji napotkano problem z widocznością danych -- większość systemów może odczytać jedynie informacje zorganizowane w \emph{tabele programu Excel}\footnote{Tabela w programie Excel wymaga osobnej deklaracji poprzez zaznaczenie zakresu komórek i wybór opcji \emph{Narzędzia główne}$\to$\emph{Formatuj jako tabelę}}.

    Rozwiązaniem okazało się zastosowanie skryptu pakietu Office, działającego bezpośrednio w arkuszu. Skrypt ten automatycznie tworzy tabelę o dynamicznym rozmiarze oraz usuwa puste kolumny. Zaimplementowany algorytm skutecznie identyfikuje początek tabeli oraz określa jej wymiary na podstawie liczby wierszy i kolumn, pomijając nieistotne dane.}
\end{comment}

W celu dostosowania danych do struktury bazy, zaplanowano zaimplementowanie formularza walidacyjnego dla nazw kolumn. System pobiera nazwy istniejących kolumn z arkusza i umożliwia ich mapowanie z wykorzystaniem predefiniowanej listy nagłówków z list SharePoint.

Po uporządkowaniu struktury, użytkownik określa rok oraz numer indykacji dla importowanego arkusza. Następnie dane przekazywane są do \emph{flow} w programie \emph{Power Automate}, który przypisze je do odpowiednich list w bazie danych, jednocześnie zapobiegając duplikacji rekordów.

Interfejs jest dodatkowo wyposażony w formularz służący do przypisywania numerów \emph{MPK} nowym serwisom. Jest to kluczowy element, ponieważ numer \emph{MPK} determinuje obszar odpowiedzialny za obsługę danej usługi. Dla serwisów występujących w poprzednich latach, system automatycznie przypisuje istniejące numery \emph{MPK}, redukując ilość danych do wprowadzenia. Jednocześnie zachowana będzie możliwość modyfikacji wcześniej przypisanych numerów.


\subsection{Interfejs procesu decyzyjnego}

Interfejs obsługi procesu decyzyjnego został podzielony na dwa współpracujące ze sobą ekrany. Takie rozwiązanie pozwala na zachowanie przejrzystości prezentowanych informacji przy jednoczesnym zapewnieniu dostępu do wszystkich niezbędnych funkcjonalności.

\subsubsection{Ekran listy serwisów}
Pierwszy ekran pełni funkcję panelu nawigacyjnego, prezentując podstawowe informacje o serwisach:
\begin{itemize}
    \item Service Name -- nazwa serwisu,
    \item Service ID -- unikalny identyfikator,
    \item MPK -- numer księgowy obszaru odpowiedzialnego,
    \item Decision -- aktualny status decyzji.
\end{itemize}

Zaimplementowany system filtrowania umożliwia:
\begin{itemize}
    \item wyszukiwanie serwisów po ID lub nazwie,
    \item filtrowanie według przypisanych numerów MPK,
    \item segregację według statusu decyzji (\emph{Accepted}, \emph{Not Accepted}, \emph{No Status}).
\end{itemize}

\subsubsection{Ekran szczegółowy serwisu}
Po wybraniu serwisu z listy, użytkownik zostaje przekierowany do ekranu szczegółowego, który składa się z trzech głównych sekcji:

\paragraph{Sekcja historyczna}
Zawiera dwie uzupełniające się tabele:
\begin{itemize}
    \item tabela przeglądowa -- prezentująca historię decyzji z poprzednich lat,
    \item tabela szczegółowa -- wyświetlająca pełne informacje dla wybranego roku (domyślnie rok poprzedni).
\end{itemize}

\paragraph{Formularz decyzyjny}
Zestaw pól do wprowadzenia informacji o bieżącej indykacji:
\begin{itemize}
    \item rok (wartość domyślna: bieżący),
    \item numer indykacji (wartość domyślna: kolejny wolny numer),
    \item autor (wartość domyślna: zalogowany użytkownik),
    \item komentarze (wewnętrzny, BSM, K-DES),
    \item decyzja (wartość domyślna: poprzednia decyzja).
\end{itemize}

\paragraph{Panel podglądu}
W celu zwiększenia czytelności, długie komentarze (powyżej 30 znaków) są początkowo prezentowane w formie skróconej. Pełna treść komentarza jest dostępna po jego wybraniu i wyświetlana w dedykowanym panelu podglądu.

System automatycznego uzupełniania wartości domyślnych przyspiesza proces decyzyjny, jednocześnie zachowując możliwość modyfikacji każdego z parametrów w razie potrzeby.

\subsection{Generowanie raportu}

Ostatnim etapem cyklu obsługi procesu jest generowanie raportu, który będzie gotowy do odesłania w celu kontynuacji konsultacji z zakładem w Wolfsburgu. W ramach tego etapu zaplanowano rozwiązanie wykorzystujące dane przechowywane na listach sharepointowych, co umożliwi wydajne zarządzanie informacjami niezbędnymi do stworzenia raportu.

W dedykowanym oknie aplikacji użytkownik będzie miał możliwość wyboru odpowiedniego roku oraz etapu (numer indykacji). Po dokonaniu tych wyborów, system przetworzy wybrane kryteria, aby zgromadzić odpowiednie dane z różnych źródeł.

Kolekcja\footnote{Kolekcja (Power Apps) -- tymczasowy zbiór danych, przechowywanych lokalnie w aplikacji, umożliwiający zarządzanie rekordami podczas jej działania.} danych, która zostanie utworzona na podstawie wybranych kryteriów, będzie łączyć informacje z trzech różnych list sharepointowych. Dzięki temu możliwe będzie skonsolidowanie danych w jedną, spójną strukturę, która zawierać będzie wszystkie niezbędne informacje do sporządzenia raportu. Mechanizmy wyszukiwania umożliwią powiązanie identyfikatorów usług z odpowiadającymi im rekordami, co zapewni dokładność i kompletność zgromadzonych danych.

Dodatkowo, w tym samym oknie aplikacji użytkownik będzie miał dostęp do podglądu zgromadzonych danych w formie tabeli. Umożliwi to weryfikację poprawności i kompletności informacji przed finalnym wygenerowaniem raportu. Po zatwierdzeniu danych, system przekaże zgromadzoną kolekcję do Power Automate, gdzie zostanie przeprowadzona dalsza obróbka, niezbędna do przygotowania raportu w odpowiednim formacie (tj. w formacie arkusza kalkulacyjnego \textit{Excel}) do odesłania.

