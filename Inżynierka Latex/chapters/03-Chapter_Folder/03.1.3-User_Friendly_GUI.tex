\subsection{Interfejs przyjazny dla użytkownika}
\begin{comment}Dedykowane narzędzie z prostym i intuicyjnym interfejsem znacząco ułatwia nawigację po bazie danych, eliminując problemy związane z tradycyjnymi rozwiązaniami, takimi jak arkusze kalkulacyjne. \end{comment} 

Dzięki dedykowanemu narzędziu z prostym i intuicyjnym interfejsem, nawigacja po bazie danych jest znacznie łatwiejsza, a problemy związane z używaniem arkuszy kalkulacyjnych zostają wyeliminowane. Przyjazny dla użytkownika interfejs oznacza:
\begin{itemize}
    \item \textbf{Prostotę:} nieskomplikowany układ umożliwia szybkie odnalezienie potrzebnych informacji.
    \item \textbf{Przejrzystość:} dane są zaprezentowane w sposób czytelny, z jasno określonymi polami i etykietami.
    \item \textbf{Przydatne funkcje:} filtrowanie i wyszukiwanie danych wspierają efektywność pracy.
\end{itemize}
Klarowny układ i czytelność interfejsu pozwalają użytkownikowi skupić się na konkretnej usłudze, co minimalizuje ryzyko pomyłek, takich jak błędne interpretowanie danych lub wybór niewłaściwego wiersza.

Takie podejście nie tylko zwiększa efektywność pracy, ale również poprawia komfort użytkowników, dzięki czemu procesy związane z analizą i zarządzaniem danymi stają się bardziej zrozumiałe i mniej podatne na błędy.