\subsection{Generowanie raportu}

Ostatnim etapem cyklu obsługi procesu jest generowanie raportu, który będzie gotowy do odesłania w celu kontynuacji konsultacji z zakładem w Wolfsburgu. W ramach tego etapu zaplanowano rozwiązanie wykorzystujące dane przechowywane na listach sharepointowych, co umożliwi wydajne zarządzanie informacjami niezbędnymi do stworzenia raportu.

W dedykowanym oknie aplikacji użytkownik będzie miał możliwość wyboru odpowiedniego roku oraz etapu (numer indykacji). Po dokonaniu tych wyborów, system przetworzy wybrane kryteria, aby zgromadzić odpowiednie dane z różnych źródeł.

Kolekcja\footnote{Kolekcja (Power Apps) -- tymczasowy zbiór danych, przechowywanych lokalnie w aplikacji, umożliwiający zarządzanie rekordami podczas jej działania.} danych, która zostanie utworzona na podstawie wybranych kryteriów, będzie łączyć informacje z trzech różnych list sharepointowych. Dzięki temu możliwe będzie skonsolidowanie danych w jedną, spójną strukturę, która zawierać będzie wszystkie niezbędne informacje do sporządzenia raportu. Mechanizmy wyszukiwania umożliwią powiązanie identyfikatorów usług z odpowiadającymi im rekordami, co zapewni dokładność i kompletność zgromadzonych danych.

Dodatkowo, w tym samym oknie aplikacji użytkownik będzie miał dostęp do podglądu zgromadzonych danych w formie tabeli. Umożliwi to weryfikację poprawności i kompletności informacji przed finalnym wygenerowaniem raportu. Po zatwierdzeniu danych, system przekaże zgromadzoną kolekcję do Power Automate, gdzie zostanie przeprowadzona dalsza obróbka, niezbędna do przygotowania raportu w odpowiednim formacie (tj. w formacie arkusza kalkulacyjnego \textit{Excel}) do odesłania.
