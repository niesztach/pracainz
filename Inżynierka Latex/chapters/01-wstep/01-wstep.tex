\chapter{Wstęp}

    \textcolor{red}{Współczesny świat biznesu stawia coraz większe wymagania wobec przedsiębiorstw, zarówno w zakresie efektywności procesów, jak i precyzji podejmowanych działań. Proces ten, dotychczas realizowany manualnie z wykorzystaniem tradycyjnych narzędzi do zarządzania danymi, wiązał się z dużym nakładem pracy, czasochłonnością, a także znacznym ryzykiem błędów ludzkich. W obliczu tych wyzwań coraz większą rolę odgrywają rozwiązania z zakresu automatyzacji biurowej, które umożliwiają usprawnienie kluczowych procesów organizacyjnych, minimalizując ryzyko błędów ludzkich oraz oszczędzając czas i zasoby.} \par Jednym z najważniejszych obszarów, w którym automatyzacja znajduje zastosowanie, jest zarządzanie usługami IT i powiązanymi kosztami. W dużych organizacjach, takich jak korporacje o rozbudowanej strukturze, konieczność zbierania, analizy oraz weryfikacji danych finansowych związanych z usługami IT stanowi poważne wyzwanie. Dzięki wdrożeniu odpowiednich narzędzi, procesy te mogą być prowadzone w sposób bardziej przejrzysty, zorganizowany i efektywny, umożliwiając jednocześnie bieżącą kontrolę nad wydatkami oraz lepsze planowanie budżetowe. \par{
    Ustandaryzowany i zautomatyzowany przepływ informacji ogranicza ryzyko powielania błędów i pozwala na skrócenie czasu potrzebnego na wykonanie poszczególnych etapów procesów. Co więcej, wdrożenie automatyzacji zapewnia większą przejrzystość i umożliwia każdemu uczestnikowi procesu łatwy dostęp do potrzebnych informacji w odpowiednim czasie. \par W dobie intensywnej cyfryzacji przedsiębiorstw oraz dynamicznego rozwoju technologii, automatyzacja biurowa staje się nie tylko opcją, ale wręcz koniecznością, aby sprostać wymaganiom współczesnego rynku. Odpowiednio zaprojektowane systemy i narzędzia wspierają nie tylko wydajność operacyjną, ale także strategiczne zarządzanie zasobami, umożliwiając organizacjom rozwój i utrzymanie konkurencyjności.}



\chapter{Wstęp szablon}

Wstęp do pracy powinien zawierać następujące elementy:
\begin{itemize}
    \item krótkie uzasadnienie podjęcia tematu; 
    \item cel pracy (patrz niżej); 
    \item zakres (przedmiotowy, podmiotowy, czasowy) wyjaśniający, w jakim rozmiarze praca będzie realizowana; 
    \item ewentualne hipotezy, które autor zamierza sprawdzić lub udowodnić; 
    \item krótką charakterystykę źródeł, zwłaszcza literaturowych; 
    \item układ pracy (patrz niżej), czyli zwięzłą charakterystykę zawartości poszczególnych rozdziałów; 
    \item ewentualne uwagi dotyczące realizacji tematu pracy np.~trudności, które pojawiły się w trakcie 
    realizacji poszczególnych zadań, uwagi dotyczące wykorzystywanego sprzętu, współpraca z firmami zewnętrznymi. 
\end{itemize}

\noindent
\textbf{Wstęp do pracy musi się kończyć dwoma następującymi akapitami:}
\begin{quote}
Celem pracy jest opracowanie / wykonanie analizy / zaprojektowanie / ...........
\end{quote}
oraz:
\begin{quote}
Struktura pracy jest następująca. W rozdziale 2 przedstawiono przegląd literatury na temat ........ 
Rozdział 3 jest poświęcony ....... (kilka zdań). 
Rozdział 4 zawiera ..... (kilka zdań) ............ itd. 
Rozdział X stanowi podsumowanie pracy. 
\end{quote}

W przypadku prac inżynierskich zespołowych lub magisterskich 2-osobowych, po tych dwóch w/w akapitach 
musi w pracy znaleźć się akapit, w którym będzie opisany udział w pracy poszczególnych członków zespołu. Na przykład:

\begin{quote}
Jan Kowalski w ramach niniejszej pracy wykonał projekt tego i tego, opracował ......
Grzegorz Brzęczyszczykiewicz wykonał ......, itd. 
\end{quote}

