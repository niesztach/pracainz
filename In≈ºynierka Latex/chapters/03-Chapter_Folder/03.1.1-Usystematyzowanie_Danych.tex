\subsection{Systematyzacja danych}
Jedną z zasadniczych funkcji omawianej aplikacji jest systematyzacja danych. Arkusze kalkulacyjne przesyłane przez oddział w Wolfsburgu nie posiadają ustandaryzowanej struktury, co negatywnie wpływa na ich czytelność oraz czas potrzebny na analizę.

\noindent Tabela \ref{HeaderComparison} przedstawia różnice w nazwach kolumn na przestrzeni trzech lat.

\renewcommand{\arraystretch}{1.3} % Zwiększenie wysokości komórek
\begin{table}[H] % [t] - tabela bliżej górnej krawędzi strony
   \centering
   \caption{Zestawienie nagłówków kolumn w latach 2022-2024}
   \label{HeaderComparison}
   \makebox[0.925\textwidth][c]{%
      \begin{tabular}{|c|W|W|W|}
         \hline
          & \textbf{2022}            & \textbf{2023}            & \textbf{2024}            \\ \hline
         \multirow{12}{*}{\rotatebox{90}{\parbox{3cm}{\centering \textbf{Nazwy kolumn na   \\przestrzeni lat}}}}
          & Service group            & Service group            & Service group            \\ \cline{2-4}
          & Service main group       & Service main group       & Service main group       \\ \cline{2-4}
          & Service sub group        & Service sub group        & Service sub group        \\ \cline{2-4}
          & Business Service         & Business Service         & Business Service         \\ \cline{2-4}
          & ID                       & ID                       & ID                       \\ \cline{2-4}
          & Business Service Manager & Business Service Manager & Business Service Manager \\ \cline{2-4}
          & Unit of Measurement      & Unit of Measurement      & Resource Unit            \\ \cline{2-4}
          &                          & Settlementtype           & Settlementtype           \\ \cline{2-4}
          & PL70 2022 PLAN EUR w KVA & PL71 2023 PLAN EUR w KVA & PL72 2024 PLAN EUR w KVA \\ \cline{2-4}
          & QTY                      & QTY                      & QTY                      \\ \cline{2-4}
          & PL71 2023 PLAN EUR w KVA & PL72 2024 PLAN EUR w KVA & PL73 2025 PLAN EUR w KVA \\ \cline{2-4}
          & QTY                      & QTY                      & QTY                      \\ \hline
      \end{tabular}
   }
\end{table}

Brak jednolitego formatu danych uniemożliwia również stworzenie spójnej bazy, co ogranicza możliwość ich wykorzystania w systemach automatyzacji procesów biznesowych. Dzięki wdrożeniu omawianego rozwiązania, możliwe jest ujednolicenie danych, pozwalając na ich efektywne zarządzanie i automatyczne przetwarzanie.