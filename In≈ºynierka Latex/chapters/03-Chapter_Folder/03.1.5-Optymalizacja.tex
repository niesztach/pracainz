\subsection{Optymalizacja}
Głównym celem implementowanego rozwiązania jest usprawnienie procesu decyzyjnego poprzez zwiększenie efektywności analizy i przetwarzania danych. Dzięki automatyzacji czas potrzebny na podjęcie decyzji zostaje znacząco skrócony, co przekłada się na większą wydajność całego procesu.

Dzięki wprowadzeniu mechanizmów automatyzacji biurowej możliwe jest zmniejszenie liczby osób zaangażowanych w realizację procesu, co może przyczynić się do ograniczenia kosztów operacyjnych i lepszej organizacji personelu w przedsiębiorstwie.

\begin{comment}
Priorytetem implementowanego rozwiązania jest usprawnienie całego procesu decyzyjnego. Głównym elementem jest redukcja czasu wymaganego na realizację poszczególnych zadań, co osiągnięto dzięki wprowadzeniu mechanizmów automatyzacji biurowej. Ważnym aspektem jest również poprawa efektywności analizy oraz przetwarzania danych przez użytkowników, umożliwiając podejmowanie bardziej trafnych decyzji w krótszym czasie.

Zmniejszenie liczby osób zaangażowanych w realizację procesu umożliwia optymalizację wykorzystania zasobów ludzkich. Dzięki temu możliwe jest ograniczenie kosztów operacyjnych i bardziej efektywne zarządzanie personelem, co przyczynia się do zwiększenia ogólnej wydajności przedsiębiorstwa.
\end{comment}