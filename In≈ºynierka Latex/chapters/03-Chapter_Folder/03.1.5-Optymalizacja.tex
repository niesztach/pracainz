\subsection{Optymalizacja}
Priorytetem implementowanego rozwiązania jest usprawnienie całego procesu decyzyjnego. Kluczowym elementem jest redukcja czasu wymaganego na realizację poszczególnych zadań, co osiągnięto dzięki wprowadzeniu mechanizmów automatyzacji. Ważnym aspektem jest również poprawa efektywności analizy oraz przetwarzania danych przez użytkowników, co pozwala na podejmowanie bardziej trafnych decyzji w krótszym czasie.

Zmniejszenie liczby osób zaangażowanych w realizację procesu pozwala na optymalizację wykorzystania zasobów ludzkich. Dzięki temu możliwe jest ograniczenie kosztów operacyjnych i bardziej efektywne zarządzanie personelem, co sprzyja zwiększeniu ogólnej wydajności organizacji.