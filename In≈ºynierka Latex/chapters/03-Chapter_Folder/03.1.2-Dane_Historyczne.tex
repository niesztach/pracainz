\subsection{Archiwizacja danych}
Utworzenie bazy danych gromadzącej informacje o wcześniejszych działaniach realizowanych w ramach projektowanego systemu stanowi istotny element zapewniający ciągłość procesów decyzyjnych. Dzięki systematycznej archiwizacji nowi użytkownicy mogą szybko zapoznać się z przebiegiem procedur i lepiej zrozumieć kontekst dotychczas podejmowanych decyzji. Dostęp do zasobów historycznych nie tylko skraca czas potrzebny na pełne wdrożenie w funkcjonowanie systemu, lecz także usprawnia przetwarzanie danych bieżących.

\begin{comment}
Utworzenie bazy danych zawierającej dane historyczne jest ważne w kontekście projektowanego systemu.

Ma ona za zadanie zapewnić ciągłość procesów decyzyjnych. Archiwizacja danych umożliwia nowym użytkownikom szybkie zapoznanie się z procesem, jego historią oraz podejmowanymi wcześniej decyzjami. Dzięki dostępowi do danych historycznych możliwe jest zrozumienie kontekstu wcześniejszych działań, co znacząco skraca czas potrzebny na wdrożenie się do pracy z systemem.

Zarchiwizowane dane stanowią podstawę do tworzenia raportów dotyczących budżetu oraz kosztów usług IT na przestrzeni lat. Analiza trendów, porównanie wyników oraz identyfikacja obszarów wymagających optymalizacji stają się możliwe dzięki dostępowi do historycznych informacji.

Dane archiwalne pozwalają na obiektywne podejmowanie decyzji, opartych na analizie wcześniejszych wyników. Dzięki temu użytkownicy mogą unikać powtarzania błędów oraz skutecznie przewidywać potencjalne konsekwencje podejmowanych działań.
\end{comment}