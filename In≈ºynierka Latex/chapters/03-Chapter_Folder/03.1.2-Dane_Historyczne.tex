\subsection{Archiwizacja danych}
Utworzenie bazy danych gromadzącej informacje o wcześniejszych działaniach realizowanych w ramach projektowanego systemu stanowi istotny element zapewniający ciągłość procesów decyzyjnych. Dzięki systematycznej archiwizacji nowi użytkownicy mogą szybko zapoznać się z przebiegiem procedur i lepiej zrozumieć kontekst dotychczas podejmowanych decyzji. Dostęp do zasobów historycznych nie tylko skraca czas potrzebny na pełne wdrożenie w funkcjonowanie systemu, lecz także usprawnia przetwarzanie danych bieżących.