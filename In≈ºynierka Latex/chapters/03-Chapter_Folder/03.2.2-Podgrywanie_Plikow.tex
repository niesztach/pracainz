\subsection{Dodawanie informacji do bazy danych}

Po ustaleniu struktury danych wykorzystywanych przez system, kolejnym etapem jest określenie sposobu importu informacji z arkuszy kalkulacyjnych do bazy danych. Postanowiono wykorzystać program Power Automate w celu automatyzacji tego procesu. Jednakże z uwagi na dużą rozbierzność danych wymaga on asysty użytkownika. 

W celu dostosowania danych do struktury bazy oraz ich importu zaplanowano zaimplementowanie formularza, który pozwoli na:
\begin{enumerate}
    \item Walidację nazw kolumn w arkuszu kalkulacyjnym:
    \begin{itemize}
        \item System pobiera nazwy istniejących kolumn z arkusza,
        \item Użytkownik mapuje nazwy kolumn z predefiniowaną listą nagłówków z list SharePoint.
    \end{itemize}
    \item Określenie roku oraz numeru indykacji dla importowanego arkusza.
    \item Przekazanie danych do \emph{flow} w programie \emph{Power Automate}, który przypisze je do odpowiednich list w bazie danych, jednocześnie zapobiegając duplikacji rekordów.
\end{enumerate}

\begin{comment}
W celu dostosowania danych do struktury bazy, zaplanowano zaimplementowanie formularza walidacyjnego dla nazw kolumn. System pobiera nazwy istniejących kolumn z arkusza i umożliwia ich mapowanie z wykorzystaniem predefiniowanej listy nagłówków z list SharePoint.

Po uporządkowaniu struktury, użytkownik określa rok oraz numer indykacji dla importowanego arkusza. Następnie dane przekazywane są do \emph{flow} w programie \emph{Power Automate}, który przypisze je do odpowiednich list w bazie danych, jednocześnie zapobiegając duplikacji rekordów.
\end{comment}
Interfejs jest dodatkowo wyposażony w formularz służący do przypisywania numerów \emph{MPK} nowym serwisom. Jest to kluczowy element, ponieważ numer \emph{MPK} determinuje obszar odpowiedzialny za obsługę danej usługi. Dla serwisów występujących w poprzednich latach, system automatycznie przypisuje istniejące numery \emph{MPK}, redukując ilość danych do wprowadzenia. Jednocześnie zachowana będzie możliwość modyfikacji wcześniej przypisanych numerów.

