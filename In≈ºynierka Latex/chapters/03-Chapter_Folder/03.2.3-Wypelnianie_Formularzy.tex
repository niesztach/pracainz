\subsection{Interfejs procesu decyzyjnego}
Interfejs obsługi procesu decyzyjnego został podzielony na dwa współpracujące ze sobą ekrany. Takie rozwiązanie pozwala na zachowanie przejrzystości prezentowanych informacji przy jednoczesnym zapewnieniu dostępu do wszystkich niezbędnych funkcjonalności.

\subsubsection*{Ekran nawigacyjny}
Pierwszy ekran pełni rolę panelu nawigacyjnego, prezentując najważniejsze informacje dotyczące usługi:
\begin{itemize}
    \item Service Name -- nazwa serwisu,
    \item Service ID -- unikalny identyfikator usługi,
    \item MPK -- numer określający miejsce powstawania kosztów,
    \item Decision -- aktualny status decyzji.
\end{itemize}
Ponadto, użytkownicy będą mieli możliwość filtrowania i wyszukiwania serwisów według następujących kryteriów:
\begin{itemize}
    \item wyszukiwanie serwisów względem ID,
    \item wyszukiwanie serwisów względem nazwy,
    \item filtrowanie według przypisanych numerów MPK,
    \item filtrowanie według statusu decyzji (\emph{Accepted}, \emph{Not Accepted}, \emph{No Status}).
\end{itemize}

\subsubsection*{Ekran szczegółowy}
Po wybraniu serwisu z listy, użytkownik zostaje przekierowany do ekranu szczegółowego, który składa się z trzech głównych sekcji:
\begin{itemize}
    \item \textbf{Podgląd danych historycznych} -- prezentuje on zarówno ogólne informacje o serwisie zbierane na przestrzeni lat, jak i szczegóły dotyczące poszczególnych indykacji.
    \item \textbf{Formularz decyzyjny} -- zestaw pól do wprowadzenia informacji o bieżącej indykacji. Składają się na niego:
          \begin{itemize}
              \item rok (wartość domyślna: bieżący),
              \item numer indykacji (wartość domyślna: kolejny wolny numer),
              \item autor (wartość domyślna: zalogowany użytkownik),
              \item komentarze (wewnętrzny, BSM, K-DES),
              \item decyzja (wartość domyślna: poprzednia decyzja).
          \end{itemize}

\end{itemize}