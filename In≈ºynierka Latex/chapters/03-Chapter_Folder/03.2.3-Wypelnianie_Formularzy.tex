\subsection{Interfejs procesu decyzyjnego}

Interfejs obsługi procesu decyzyjnego został podzielony na dwa współpracujące ze sobą ekrany. Takie rozwiązanie pozwala na zachowanie przejrzystości prezentowanych informacji przy jednoczesnym zapewnieniu dostępu do wszystkich niezbędnych funkcjonalności.

\subsubsection{Ekran listy serwisów}
Pierwszy ekran pełni funkcję panelu nawigacyjnego, prezentując podstawowe informacje o serwisach:
\begin{itemize}
    \item Service Name -- nazwa serwisu,
    \item Service ID -- unikalny identyfikator,
    \item MPK -- numer księgowy obszaru odpowiedzialnego,
    \item Decision -- aktualny status decyzji.
\end{itemize}

Zaimplementowany system filtrowania umożliwia:
\begin{itemize}
    \item wyszukiwanie serwisów po ID lub nazwie,
    \item filtrowanie według przypisanych numerów MPK,
    \item segregację według statusu decyzji (\emph{Accepted}, \emph{Not Accepted}, \emph{No Status}).
\end{itemize}

\subsubsection{Ekran szczegółowy serwisu}
Po wybraniu serwisu z listy, użytkownik zostaje przekierowany do ekranu szczegółowego, który składa się z trzech głównych sekcji:

\paragraph{Sekcja historyczna}
Zawiera dwie uzupełniające się tabele:
\begin{itemize}
    \item tabela przeglądowa -- prezentująca historię decyzji z poprzednich lat,
    \item tabela szczegółowa -- wyświetlająca pełne informacje dla wybranego roku (domyślnie rok poprzedni).
\end{itemize}

\paragraph{Formularz decyzyjny}
Zestaw pól do wprowadzenia informacji o bieżącej indykacji:
\begin{itemize}
    \item rok (wartość domyślna: bieżący),
    \item numer indykacji (wartość domyślna: kolejny wolny numer),
    \item autor (wartość domyślna: zalogowany użytkownik),
    \item komentarze (wewnętrzny, BSM, K-DES),
    \item decyzja (wartość domyślna: poprzednia decyzja).
\end{itemize}

\paragraph{Panel podglądu}
W celu zwiększenia czytelności, długie komentarze (powyżej 30 znaków) są początkowo prezentowane w formie skróconej. Pełna treść komentarza jest dostępna po jego wybraniu i wyświetlana w dedykowanym panelu podglądu.

System automatycznego uzupełniania wartości domyślnych przyspiesza proces decyzyjny, jednocześnie zachowując możliwość modyfikacji każdego z parametrów w razie potrzeby.
