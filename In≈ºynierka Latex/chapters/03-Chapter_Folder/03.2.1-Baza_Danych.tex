\subsection{Baza danych}

Baza danych oparta jest o listy programu Sharepoint. \textcolor{pink}{Jak wcześniej wspomniano, nie jest to dedykowane rozwiązanie do budowy takich struktur. }

Na początku należało określić jakie kolumny będą zawierały omawiane listy. W tym celu przeanalizowano pliki z lat poprzednich i wybrano powtarzające się elementy. Pozwoliło to na uzyskanie jednolitego schematu danych:
\begin{itemize}
    \item Service group,
    \item Service main group,
    \item Service sub group,
    \item Business Service (Service Name),
    \item Instruction link,
    \item ID,
    \item Business Service Manager,
    \item Unit Of Measurement,
    \item Settlement Type,
    \item Current Year Plan EUR,
    \item Quantity Current Year,
    \item Next Year Plan EUR,
    \item Quantity Next Year,
    \item Year,
    \item MPK,
    \item Difference,
    \item Indication Number,
    \item Comment Intern,
    \item Comment Date,
    \item Comment Author,
    \item Comment PZ to WOB,
    \item Comment BSM,
    \item Comment K-DES,
    \item Decision,
    \item Final comment.
\end{itemize} 
Na podstawie wstępnych ustaleń postanowiono pogrupować dane w trzy listy, z których każda charakteryzuje się odmienną strukturą oraz częstotliwością aktualizacji elementów.

Pierwsza, określana jako \emph{Lista usług}, zawiera dane dotyczące usług pozostających niezmiennymi przez wiele lat, takich jak \emph{Business Service} czy \emph{Instruction link}.

Druga, nazwana \emph{Listą kwot}, gromadzi informacje aktualizowane raz do roku, na przykład ceny serwisów (\emph{Current Year Plan EUR, Next Year Plan EUR}) czy liczby licencji (\emph{Quantity Current Year, Quantity Next Year}).

Trzecia, opisana jako \emph{Lista indykacji}, obejmuje rekordy przypisane do poszczególnych indykacji, zawierając zarówno komentarze, jak i decyzje podjęte w trakcie całego procesu.
\vspace{2cm}\par
\textcolor{red}{opisać na jakiej zasadzie listy są między sobą powiązane i dorzucić elegancki schemacik}
\vspace{2cm}\par
Koncepcja zakładała wykorzystanie relacji pomiędzy listami już na poziomie SharePointa, który umożliwia definiowanie kolumn typu \emph{lookup} do powiązywania dwóch list na podstawie wspólnych elementów w określonych kolumnach. Rozwiązanie to ogranicza się jednak do relacji wyłącznie pomiędzy dwiema listami, co w przypadku konieczności integracji trzech list sprawia, że kolumny typu \emph{lookup} nie mają zastosowania.
\newline
\textcolor{Salmon}{Do implementacji dopisać jak finalnie zrobione są relacje między listami(że na poziomie power apps)}






