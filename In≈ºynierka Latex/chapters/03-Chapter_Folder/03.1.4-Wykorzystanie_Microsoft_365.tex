\subsection{Użycie pakietu Microsoft 365}
\begin{comment}


Wykorzystanie platformy Power\footnote{Platforma Power (\english{Power Platform}) -- Składowa pakietu Microsoft 365. Zawiera ona takie programy jak Power Apps, Power Automate czy Power BI.} w połączeniu z Sharepoint, pozwala na utworzenie w pełni funkcjonalnego rozwiązania, zachowując spójność danych dzięki integracji poszczególnych składników pakietu.

Aby korzystanie z aplikacji było możliwe, użytkownicy muszą mieć dostęp do potrzebnych usług oraz licencje. W przypadku omawianego pakietu, każdy z pracowników, ma do niego dostęp. Pozwala to na uniknięcie dodatkowych kosztów.

Niestety użyty pakiet, nie jest dostępny w najbardziej rozbudowanym wariancie. Wprowadza to pewne ograniczenia, ponieważ brakuje w nim oprogramowania do tworzenia i zarządzania rozbudowanymi bazami danych o złożonej strukturze (takie możliwości daje między innymi \emph{Microsoft Azure}).
Sharepoint pozwala jedynie na utworzenie prostej bazy danych opierającej się o wcześniej opisane listy.  Głównym problemem było ograniczenie związane z brakiem możliwości tworzenia relacji między kilkoma listami, co znacząco utrudniało zarządzanie danymi o złożonej strukturze.

\end{comment}

Wykorzystanie platformy Power\footnote{Platforma Power (\english{Power Platform}) -- składowa pakietu Microsoft 365, w skład której wchodzą takie programy jak Power Apps, Power Automate czy Power BI.} w połączeniu z Sharepoint, pozwala na utworzenie w pełni funkcjonalnego rozwiązania, zachowując spójność danych dzięki integracji poszczególnych składników pakietu.

Aby korzystanie z aplikacji było możliwe, użytkownicy muszą mieć dostęp do potrzebnych usług oraz licencje. W przypadku omawianego pakietu, każdy z pracowników, ma do niego dostęp. Pozwala to na uniknięcie dodatkowych kosztów.

Wykorzystany pakiet nie jest dostępny w najbardziej rozbudowanym wariancie, co wprowadza pewne ograniczenia, ponieważ nie zawiera oprogramowania do tworzenia i zarządzania rozbudowanymi bazami danych o złożonej strukturze (takie możliwości daje między innymi \emph{Microsoft Azure}).
Sharepoint pozwala jedynie na utworzenie prostej bazy danych opierającej się o wcześniej opisane listy.


\customnote{ //To chyba do implementacji // Głównym problemem było ograniczenie związane z brakiem możliwości tworzenia relacji między kilkoma listami, co znacząco utrudniało zarządzanie danymi o złożonej strukturze.}
