\section{Struktura procesu}
\sectionauthor{R. Wolniak}
Przedmiotem omawianego procesu jest podjęcie decyzji o zakupie usług IT w zakładzie Volkswagen Poznań. Proces ten polega na wielokrotnej wymianie uwag dotyczących wcześniej używanego lub nowego oprogramowania między oddziałem Volkswagena w Poznaniu a zakładem z siedzibą w Wolfsburgu.

W wyniku wymiany zdań zapada decyzja o zakupie lub rezygnacji z wybranego produktu. Procedura, zazwyczaj podzielona na cztery indykacje (\emph{Indikation} - termin pochodzący od niemieckiego słowa oznaczającego wstępne głosowanie lub wskazanie kierunku działania), rozpoczyna się na początku czerwca i trwa do końca roku.

Efektem podejmowanych działań jest nabycie odpowiedniej liczby potrzebnych uprawnień licencyjnych. Przy
podejmowaniu decyzji kluczowymi aspektami są:
\begin{itemize}
    \item liczba użytkowników danego oprogramowania,
    \item cena zakupu w porównaniu z rokiem poprzednim,
    \item określenie, czy dana usługa zostanie w pełni wykorzystana biorąc pod uwagę poprzednie kryteria.
\end{itemize}
Dotychczas analiza i przetwarzanie danych odbywały się przy użyciu arkuszy kalkulacyjnych programu Excel, a wymiana informacji między jednostkami była realizowana za pomocą wiadomości e-mail.