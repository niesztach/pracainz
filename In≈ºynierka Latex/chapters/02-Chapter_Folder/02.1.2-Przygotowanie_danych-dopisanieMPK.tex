\subsection{Przygotowanie danych}
Otrzymany arkusz kalkulacyjny, zawiera tabelę o strzukturze kolumn podobnej do tabeli \ref{Headers2022}. Brakuje w nim jednak informacji kluczowych do rozpoczęcia cyklu.
Dlatego pierwszym krokiem jest przygotowanie danych przez osobę nadzorującą proces ze strony odziału w Poznaniu.
Jej zadaniem jest manualne przypisanie numeru określającego miejsce powstawania kosztów, wewnętrznie nazywanego \definicja{MPK}. Numer ten definiuje konkretną jednostkę należącą do obszaru IT, która decyduje o zakupie danego produktu. Ponadto, dodawana jest kolumna, w której znajduje się wyliczona różnica cen między rokiem obecnym a poprzednim, w celu określenia czy koszt wzrósł lub zmalał. Tak przetworzony plik zostaje umieszczony we wspólnej przestrzeni dyskowej, co umożliwia pozostałym uczestnikom procesu przystąpienie do analizy oraz dalszego przetwarzania zawartych w nim informacji.

\renewcommand{\arraystretch}{1.1} % Zwiększenie wysokości komórek
\begin{table}[H] % [H] - tabela dokładnie w tym miejscu
    \begin{adjustwidth}{-50pt}{-20pt}
        \centering
        \caption{Nagłówki kolumn z arkusza kalkulacyjnego z roku 2022}
        \label{Headers2022}
        \makebox[\textwidth][c]{%
            \begin{tabular}{*{3}{|m{1.1cm}}|w|m{0.4cm}|m{1.5cm}|m{1.75cm}|w|m{0.7cm}|w|m{0.7cm}|}
                \hline
                Service group & Service main group & Service sub group & Business Service & ID & Business Service Manager & Unit of Measurement & PL70 2022 PLAN EUR w KVA & QTY & PL71 2023 PLAN EUR w KVA & QTY \\ \hline
            \end{tabular}
        }
    \end{adjustwidth}
\end{table}
