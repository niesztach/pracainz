\subsection{Power Automate \cite{v-aangie_official_nodate}}
Power Automate to narzędzie wchodzące w skład pakietu Microsoft 365, które umożliwia automatyzację procesów biznesowych (\akronim{RPA}, \english{Robotic Process Automation}). Pozwala ono na tworzenie przepływów pracy (\english{flows}), automatyzujących powtarzalne zadania i integrujących różne systemy, zwiększając efektywność procesów biznesowych.

Flow w Power Automate jest odpowiednikiem funkcji w standardowych językach programowania. Na przykład, przepływ może automatycznie wysyłać powiadomienia e-mail po aktualizacji rekordu w SharePoint. Różnica polega na tym, że jest ono tworzone w wizualnym środowisku Low-Code i działa na zasadzie logicznego ciągu akcji wyzwalanych po sobie przez określone instrukcje.

Za pomocą flow można tworzyć własne procesy, które przy odpowiedniej implementacji, dorównują tym znanym z pełnych środowisk kodowych pod względem logiki i efektywności. Do dyspozycji są instrukcje warunkowe, pętle, zmienne, operacje na danych czy integracje z API poprzez konektory.
