\subsection{Generowanie raportu}
Ostatnim etapem cyklu obsługi procesu jest generowanie raportu, który jest następnie przesyłany do zakładu w Wolfsburgu w celu dalszych konsultacji. Raport jest tworzony na podstawie danych przechowywanych na listach SharePoint, co zapewnia spójność i aktualność informacji.

W dedykowanym oknie aplikacji użytkownik ma możliwość wyboru odpowiedniego roku oraz etapu (numer indykacji). Na podstawie tych informacji, system przetworzy podane kryteria, aby zgromadzić odpowiednie dane z różnych źródeł.

Kolekcja\footnote{Kolekcja (Power Apps) -- tymczasowy zbiór danych, przechowywanych lokalnie w aplikacji, umożliwiający zarządzanie rekordami podczas jej działania.} danych, utworzona na podstawie wybranych kryteriów, łączy dane z trzech list sharepointowych. Dzięki temu możliwe jest skonsolidowanie danych w jedną, spójną strukturę, która zawierać będzie wszystkie niezbędne informacje do sporządzenia raportu. Utworzone zostaną mechanizmy wyszukiwania, które umożliwią powiązanie identyfikatorów usług z odpowiadającymi im rekordami, co zapewni integralność zgromadzonych danych.

Dodatkowo, w tym samym oknie aplikacji użytkownik ma dostęp do podglądu zgromadzonych danych w formie tabeli. Umożliwi to weryfikację poprawności i kompletności informacji przed wygenerowaniem raportu. Po zatwierdzeniu danych, system przekaże zgromadzoną kolekcję do Power Automate, gdzie zostaną one  zmodyfikowane, w celu do przygotowania raportu w odpowiednim formacie (tj. w formacie arkusza kalkulacyjnego \textit{Excel}) do odesłania.

