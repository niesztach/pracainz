\subsection{Generowanie raportu}
\customnote{\textbf{Niby spoko, ale jakoś do pizdy się niektóre zdania czyta. Trzeba to jakoś poprawić. Jakoś tak zbyt ogólnie}
Ostatnim etapem cyklu obsługi procesu jest generowanie raportu, który będzie gotowy do odesłania w celu kontynuacji konsultacji z zakładem w Wolfsburgu. W ramach tego etapu zaplanowano rozwiązanie wykorzystujące dane przechowywane na listach sharepointowych, co umożliwi wydajne zarządzanie informacjami niezbędnymi do stworzenia raportu.

W dedykowanym oknie aplikacji użytkownik ma możliwość wyboru odpowiedniego roku oraz etapu (numer indykacji). Na podstawie tych informacji, system przetworzy podane kryteria, aby zgromadzić odpowiednie dane z różnych źródeł.

Kolekcja\footnote{Kolekcja (Power Apps) -- tymczasowy zbiór danych, przechowywanych lokalnie w aplikacji, umożliwiający zarządzanie rekordami podczas jej działania.} danych, utworzona na podstawie wybranych kryteriów, łączy dane z trzech list sharepointowych. Dzięki temu możliwe jest skonsolidowanie danych w jedną, spójną strukturę, która zawierać będzie wszystkie niezbędne informacje do sporządzenia raportu. \textbf{TO TRZEBA ROZWINĄĆ W IMPLEMENTACJI BO MOGĄ SIĘ POJAWIĆ JAKIE MECHANIZMY (JA NA PRZYKŁAD NIE WIEM XD):} Mechanizmy wyszukiwania umożliwią powiązanie identyfikatorów usług z odpowiadającymi im rekordami, co zapewni dokładność i kompletność zgromadzonych danych.

Dodatkowo, w tym samym oknie aplikacji użytkownik ma dostęp do podglądu zgromadzonych danych w formie tabeli. Umożliwi to weryfikację poprawności i kompletności informacji przed finalnym wygenerowaniem raportu. Po zatwierdzeniu danych, system przekaże zgromadzoną kolekcję do Power Automate, gdzie zostaną one  zmodyfikowane, w celu do przygotowania raportu w odpowiednim formacie (tj. w formacie arkusza kalkulacyjnego \textit{Excel}) do odesłania.}
