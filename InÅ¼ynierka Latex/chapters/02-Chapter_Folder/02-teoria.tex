
\chapter{Podstawy teoretyczne}

%\newline\textcolor{orange}{
%Można coś dopisać/ poprawić to powyżej. Chciałem tutaj przedstawić zarys jak wygląda proces.
%myśle żeby poniżej w każdym akapicie opisać krok po kroku jak to wyglądało z użyciem exceli.
%Tylko trzeba to napisać sensownie i łądnie żeby nikt się nie zajebał w akcji i żeby uwzględnić
%wszystkie rzeczy. Tutaj chyba nie będziemy się rozwodzili na temat minusów tego rozwiązania.
%no i potem bym dał informacje na temat użytych technologii: typescript, sharepoint, power au-
%tomate i power apps. imo taka kolejność żeby odzwierciedlała trochę jaki jest proces w apce bo
%bedziesz mógł się odwołąć do poprzednich. np że automate zaciaga dane z SP i każdy wie czym
%jest już sharepoint i essa.}
\section{Struktura procesu}
% Przedmiotem omawianego procesu jest podjęcie decyzji na tematu zakupu usług IT w zakładzie
% Volkswagen Poznań. Polega on na wymianie uwag, dotyczących wcześniej używanego bądź nowego
% oprogramowania, między oddziałem Volkswagen w Poznaniu a zakładem z siedzibą w Wolfsburgu.
Przedmiotem omawianego procesu jest podjęcie decyzji dotyczących zakupu usług IT w zakładzie Volkswagen Poznań. Proces ten polega na wielokrotnej wymianie uwag dotyczących wcześniej używanego lub nowego oprogramowania między oddziałem Volkswagena w Poznaniu a zakładem z siedzibą w Wolfsburgu.

W wyniku wymiany zdań zapada decyzja o zakupie lub rezygnacji z wybranego produktu. Procedura, zazwyczaj podzielona na cztery {indykacje}\footnote{\emph{indykacja}  - wstępne głosowanie}, rozpoczyna się wraz z początkiem czerwca i trwa do końca roku.

Efektem podejmowanych działań jest nabycie odpowiedniej ilości potrzebnych uprawnień licencyjnych. Przy
podejmowaniu decyzji kluczowymi aspektami są:
\begin{itemize}
    \item liczba użytkowników danego oprogramowania,
    \item cena zakupu w porównaniu z rokiem poprzednim,
    \item określenie, czy dana usługa zostanie w pełni wykorzystana biorąc pod uwagę poprzednie kryteria.
\end{itemize}
Dotychczas analiza i przetwarzanie danych odbywały się przy użyciu arkuszy kalkulacyjnych programu Excel, a wymiana informacji między jednostkami była realizowana za pomocą wiadomości e-mail.
\subsection{Gromadzenie danych dotyczących ofert usługodawców}
Informacje na temat serwisów są zbierane na początku roku, przed rozpoczęciem cyklu procesu. W tym czasie, prowadzone są rozmowy między menadżerami odpowiedzialnymi za dane rozwiązanie (\akronim{BSM}, \english{Business Service Manager}) a firmami świadczącymi usługi, w celu otrzymania zaaktualizowanych wiadomości związanych z ich produktami. Na podstawie danych od usługodawców oraz menadżerów, powstaje arkusz, który jest przekazywany do zakładu w Poznaniu.
\subsection{Przygotowanie danych}
Otrzymany arkusz kalkulacyjny, zawiera tabelę o strzukturze kolumn podobnej do tabeli \ref{Headers2022}. Brakuje w nim jednak informacji kluczowych do rozpoczęcia cyklu.
Dlatego pierwszym krokiem jest przygotowanie danych przez osobę nadzorującą proces ze strony odziału w Poznaniu.
Jej zadaniem jest manualne przypisanie numeru określającego miejsce powstawania kosztów, wewnętrznie nazywanego \definicja{MPK}. Numer ten definiuje konkretną jednostkę należącą do obszaru IT, która decyduje o zakupie danego produktu. Ponadto, dodawana jest kolumna, w której znajduje się wyliczona różnica cen między rokiem obecnym a poprzednim, w celu określenia czy koszt wzrósł lub zmalał. Tak przetworzony plik zostaje umieszczony we wspólnej przestrzeni dyskowej, co umożliwia pozostałym uczestnikom procesu przystąpienie do analizy oraz dalszego przetwarzania zawartych w nim informacji.

\renewcommand{\arraystretch}{1.1} % Zwiększenie wysokości komórek
\begin{table}[H] % [H] - tabela dokładnie w tym miejscu
    \begin{adjustwidth}{-50pt}{-20pt}
        \centering
        \caption{Nagłówki kolumn z arkusza kalkulacyjnego z roku 2022}
        \label{Headers2022}
        \makebox[\textwidth][c]{%
            \begin{tabular}{*{3}{|m{1.1cm}}|w|m{0.4cm}|m{1.5cm}|m{1.75cm}|w|m{0.7cm}|w|m{0.7cm}|}
                \hline
                Service group & Service main group & Service sub group & Business Service & ID & Business Service Manager & Unit of Measurement & PL70 2022 PLAN EUR w KVA & QTY & PL71 2023 PLAN EUR w KVA & QTY \\ \hline
            \end{tabular}
        }
    \end{adjustwidth}
\end{table}

\subsection{Przebieg Iteracji}
W trakcie trwania iteracji analizowane są kluczowe informacje, takie jak:
\begin{itemize}
    \item jednostka miary (ang. \emph{Unit of Measurement}),
    \item decyzja podjęta w roku poprzednim,
    \item cena oraz liczba użytkowników w roku obecnym,
    \item cena oraz liczba użytkowników w roku przyszłym.
\end{itemize}
Po analizie i porównaniu danych z wcześniejszych lat, w arkuszu powstają kolejne kolumny. Ich struktura nie jest określona przez żaden standard, ale zazwyczaj zawierają one:
\begin{itemize}
    \item Komentarz wewnętrzny,
    \item Status,
    \item Komentarz klienta.
\end{itemize}

\noindent\emph{Komentarz wewnętrzny} nie jest wymagany dla każdego serwisu. Jest on zapisywany w celu skonsultowania decyzji ze współpracownikami.\\ \emph{Status} określa wstępną, wymaganą decyzję (Zaakceptowany/Niezaakceptowany).\\ \emph{Komentarz klienta} zawiera uzasadnienie podjętej decyzji ze strony Volkswagen Poznań.\\Tak uzupełniony arkusz zostaje przekazany pośrednio przez zakład w Wolfsburgu do zarządu firmy. \par
Kolejnym etapem jest analiza tych informacji przez wcześniej wymienione podmioty. Ich zadaniem jest konfrontacja podjętej decyzji. Dodawane są kolejne kolumny:
\begin{itemize}
    \item Komentarz BSM,
    \item Komentarz K-DES.
\end{itemize}

\noindent\emph{Komentarz BSM} to opinia wyrażona przez menedżera usługi, natomiast \emph{Komentarz K-DES} stanowi odpowiedź międzynarodowego zarządu firmy, który odpowiada za kształtowanie strategii IT.\par
Zaaktualizowany plik powraca do Volkswagen Poznań, rozpoczynając tym samym kolejną iterację procesu.







% Podsumowanie przebiegu proceso - nie tworzyłem nowego pliku na dwa zdania.
\vspace{1cm}
Jak wcześniej wspomniano, proces składa się zazwyczaj z czterech iteracji. Etapem zamykający proces jest sporządzenie wymaganych dokumentów oraz faktur.

\section{Wykorzystane technologie}
Aby usprawnić przebieg procesu, zabiezpieczyć go przed błędami i usystematyzować, stworzona została aplikacja do jego obsługi. Głównym kryterium przy doborze technologii była powszechna dostępność do powstałego systemu wśród pracowników. Dlatego też zdecydowano się na wykorzystanie komponentów pakietu \emph{Office 365}. Pakiet ten jest bardzo rozbudowany i jest powszechnie używany w firmie Volkswagen. Zawiera on programy pozwalające na stworzenie kompletnego systemu bez konieczności dostępu do dodatkowych usług.









\renewcommand{\arraystretch}{1.5} % Zwiększenie wysokości komórek
\begin{table}[H] % [t] - tabela bliżej górnej krawędzi strony
    \centering
    \caption{}
    \label{HeaderComparison}
    \makebox[\textwidth][c]{%
        \begin{tabular}{|c|W|W|W|}
        \hline
         & \textbf{2022} & \textbf{2023} & \textbf{2024} \\ \hline
        \multirow{12}{*}{\rotatebox{90}{\parbox{4cm}{\centering \textbf{Nazwy kolumn na\\przestrzeni lat}}}}
        & Service group & Service group & Service group \\ \cline{2-4}
        & Service main group & Service main group & Service main group \\ \cline{2-4}
        & Service sub group & Service sub group & Service sub group \\ \cline{2-4}
        & Business Service & Business Service & Business Service \\ \cline{2-4}
        & ID & ID & ID \\ \cline{2-4}
        & Business Service Manager & Business Service Manager & Business Service Manager \\ \cline{2-4}
        & Unit of Measurement & Unit of Measurement & Resource Unit \\ \cline{2-4}
        &  & Settlementtype & Settlementtype \\ \cline{2-4}
        & PL71 2023 PLAN EUR w KVA & PL71 2023 PLAN EUR w KVA & PL72 2024 PLAN EUR w KVA \\ \cline{2-4}
        & QTY & QTY & QTY \\ \cline{2-4}
        & PL71 2023 PLAN EUR w KVA & PL72 2024 PLAN EUR w KVA & PL73 2025 PLAN EUR w KVA \\ \cline{2-4}
        & QTY & QTY & QTY \\ \hline
        \end{tabular}
    }
\end{table}


    
    
    
