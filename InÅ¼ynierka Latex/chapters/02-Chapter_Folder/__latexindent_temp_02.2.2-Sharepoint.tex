\subsection{SharePoint}
SharePoint to platforma należąca do pakietu Microsoft 365, umożliwiająca tworzenie aplikacji webowych, takich jak witryny i strony internetowe. Jej głównym celem jest usprawnienie współpracy zespołowej poprzez dostarczenie narzędzi do publikowania informacji i raportów, które mogą być skierowane do określonych grup odbiorców. \par
Jednym z kluczowych zastosowań SharePointa jest zarządzanie danymi. Platforma oferuje przestrzeń do przechowywania różnego rodzaju plików, dokumentów i informacji, pełniąc funkcję serwera danych. Dzięki dostępności wbudowanych konektorów\footnote{\emph{Konektor} (\english{connector}) -– moduł umożliwiający integrację aplikacji z usługami lub źródłami danych w celu wymiany informacji i synchronizacji systemów.} (\english{connectors}), umożliwia również wykorzystanie przechowywanych danych w procesie tworzenia aplikacji czy witryn. \par
Istotnym elementem środowiska SharePoint jest możliwość tworzenia list, często nazywanych \emph{listami sharepointowymi}. Listy te mogą być wykorzystywane jako proste bazy danych, które umożliwiają dynamiczne aktualizowanie i synchronizowanie danych w czasie rzeczywistym. \par
SharePoint oferuje zaawansowane zarządzanie uprawnieniami. Administratorzy mogą precyzyjnie definiować dostęp użytkowników do poszczgólnych zasobów witryny co pozwala na skuteczne zabezpieczenie wrażliwych informacji. \par
Platforma jest silnie zintegrowana z innymi usługami pakietu Microsoft 365, takimi jak Teams, Outlook czy OneDrive. Dzięki temu użytkownicy mogą współdzielić dane, pracować nad nimi w czasie rzeczywistym i korzystać z jednego spójnego środowiska pracy. \par
Ważnym aspektem SharePointa jest możliwość dostosowania wyglądu i funkcjonalności witryn do potrzeb użytkowników. Personalizacja obejmuje m.in. konfigurację interfejsu, dodawanie aplikacji webowych czy tworzenie dedykowanych formularzy. \par
W kontekście współczesnych modeli pracy, takich jak praca hybrydowa czy zdalna, SharePoint oferuje wsparcie dla użytkowników korzystających z różnorodnych urządzeń. Dostęp do danych jest możliwy za pośrednictwem przeglądarki internetowej oraz aplikacji mobilnych.

\begin{comment}
SharePoint to platforma wchodząca w skład pakietu Microsoft 365, która służy do zarządzania dokumentami oraz umożliwia efektywną współpracę zespołową. Dzięki niej użytkownicy mogą tworzyć i korzystać z własnych przestrzeni roboczych online, takich jak listy czy  archiwa plików, co znacząco ułatwia organizację oraz szybki dostęp do danych.

Listy sharepointowe mogą pełnić funkcję prostych baz danych, które można łatwo zintegrować z takimi narzędziami jak PowerApps, PowerAutomate, czy Excel. Takie rozwiązanie umożliwia dynamiczne aktualizowanie i synchronizowanie danych w czasie rzeczywistym.
\end{comment}


