\chapter{Wstęp}

    Współczesny świat biznesu stawia coraz większe wymagania wobec przedsiębiorstw, zarówno w zakresie wydajności procesów, jak i precyzji podejmowanych działań. Powszechne metody zarządzania i przetwarzania danych, oparte są na manualnej pracy z wykorzystaniem mało efektywnych narzędzi oraz wymianie informacji w sposób niustandaryzowany. Stają się one niewystarczające w przypadku rosnącej skali operacji oraz wymagań co do szybkości i niezawodności podejmowanych decyzji. W obliczu tych wyzwań coraz większą rolę odgrywają rozwiązania z zakresu automatyzacji biurowej, które umożliwiają oszczędność czasu i zasobów oraz pozwalają na usprawnienie kluczowych procesów organizacyjnych, minimalizując ryzyko błędów ludzkich.
    \par Jednym z obszarów, w którym automatyzacja znajduje zastosowanie, jest zarządzanie usługami IT i powiązanymi kosztami. W dużych organizacjach o rozbudowanej strukturze, konieczność gromadzenia, analizy oraz weryfikacji danych finansowych stanowi poważne wyzwanie. Dzięki wdrożeniu odpowiednich narzędzi, procesy te mogą być prowadzone w sposób uporządkowany i efektywny, umożliwiając jednocześnie bieżącą kontrolę nad wydatkami oraz lepsze planowanie budżetowe. 
    \par   Ustandaryzowany i zautomatyzowany przepływ informacji ogranicza ryzyko powielania błędów i pozwala na skrócenie czasu potrzebnego na wykonanie poszczególnych zadań. Dodatkowo, wdrożenie automatyzacji zapewnia większą przejrzystość i ułatwia dostęp do informacji każdemu uczestnikowi procesu. 
    \par W dobie intensywnej cyfryzacji przedsiębiorstw oraz dynamicznego rozwoju technologii, automatyzacja biurowa staje się konieczna, aby sprostać wymaganiom współczesnego rynku. Odpowiednio zaprojektowane systemy i narzędzia wspierają nie tylko wydajność operacyjną, ale także strategiczne zarządzanie zasobami, umożliwiając rozwój w innych obszarach swojej działalności.
    \vspace{1cm}
    \par Celem pracy jest opracowanie aplikacji, usprawniającej proces podejmowania decyzji dotyczących zakupu \emph{usług IT}\footnote{\emph{Usługi IT}  należy rozumieć jako licencje oraz klucze dostępu do używanych systemów informatycznych.} na najbliższy rok kalendarzowy. Praca została wykonana z wykorzystaniem \emph{Power Platform} oraz \emph{SharePoint}, które są integralną cześcią pakietu \emph{Microsoft 365}. Zdecydowano się na wybór tego rozwiązania, ponieważ pozwala ono na prostą integrację między programami wchodzącymi w skład pakietu. Ponadto, każdy z uczestników procesu ma dostęp do wspomnianych serwisów, co pozwala uniknąć dodatkowych kosztów.

    \par\textbf{\textcolor{red}{DOPISAĆ:} \newline 
    \begin{quote}
        \color{red}
        Struktura pracy jest następująca. W rozdziale 2  \newline przedstawiono przegląd literatury na temat \ldots  \newline 
        Rozdział 3 jest poświęcony \ldots (kilka zdań).  \newline 
        Rozdział 4 zawiera \ldots (kilka zdań) \ldots itd.  \newline 
        Rozdział X stanowi podsumowanie pracy. 
        \end{quote}}
       \textbf{\textcolor{red}{ 
W przypadku prac inżynierskich zespołowych lub magisterskich 2-osobowych, po tych dwóch w/w akapitach 
musi w pracy znaleźć się akapit, w którym będzie opisany udział w pracy poszczególnych członków zespołu. Na przykład:}
\begin{quote}
\color{red}
    Jan Kowalski w ramach niniejszej pracy wykonał projekt tego i tego, opracował \ldots
    Grzegorz Brzęczyszczykiewicz wykonał \ldots, itd. 
\end{quote}}

