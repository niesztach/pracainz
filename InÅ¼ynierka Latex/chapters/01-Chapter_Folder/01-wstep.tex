\chapter{Wstęp}
Współczesny świat biznesu stawia coraz większe wymagania wobec przedsiębiorstw, zarówno w zakresie wydajności procesów, jak i precyzji podejmowanych decyzji. Tradycyjne metody zarządzania i przetwarzania danych, oparte na pracy manualnej i mało efektywnych narzędziach, stają się niewystarczające w obliczu rosnącej skali operacji oraz konieczności szybkiego i niezawodnego podejmowania decyzji. W odpowiedzi na te wyzwania coraz większą rolę odgrywają rozwiązania z zakresu automatyzacji biurowej, które pozwalają na oszczędność czasu i zasobów, usprawnienie kluczowych procesów organizacyjnych oraz minimalizację ryzyka błędów ludzkich.
\par Jednym z obszarów, w którym automatyzacja znajduje zastosowanie, jest zarządzanie usługami IT i powiązanymi kosztami. W dużych organizacjach o rozbudowanej strukturze, konieczność gromadzenia, analizy oraz weryfikacji danych finansowych stanowi poważne wyzwanie. Dzięki wdrożeniu odpowiednich narzędzi, procesy te mogą być prowadzone w sposób uporządkowany i efektywny, umożliwiając jednocześnie bieżącą kontrolę nad wydatkami oraz lepsze planowanie budżetowe.
\par   Ustandaryzowany i zautomatyzowany przepływ informacji ogranicza ryzyko powielania błędów i pozwala na skrócenie czasu potrzebnego na wykonanie poszczególnych zadań. Dodatkowo, wdrożenie automatyzacji zapewnia większą przejrzystość i ułatwia dostęp do informacji każdemu uczestnikowi procesu.
\par W dobie intensywnej cyfryzacji przedsiębiorstw oraz dynamicznego rozwoju technologii, automatyzacja biurowa staje się konieczna, aby sprostać wymaganiom współczesnego rynku. Odpowiednio zaprojektowane systemy i narzędzia wspierają nie tylko wydajność operacyjną, ale także strategiczne zarządzanie zasobami, umożliwiając rozwój w innych obszarach swojej działalności.
\vspace{1cm}
\par Celem pracy jest opracowanie aplikacji usprawniającej proces podejmowania decyzji dotyczących zakupu \emph{usług IT}\footnote{\emph{Usługi IT}  należy rozumieć jako licencje oraz klucze dostępu do używanych systemów informatycznych.} na najbliższy rok kalendarzowy. Praca została wykonana z wykorzystaniem \emph{Power Platform} oraz \emph{SharePoint}, które są integralną częścią pakietu \emph{Microsoft 365}. Zdecydowano się na wybór tego rozwiązania, ponieważ pozwala ono na prostą integrację między programami wchodzącymi w skład pakietu. Ponadto, każdy z uczestników procesu ma dostęp do wspomnianych serwisów, co pozwala uniknąć dodatkowych kosztów.

\pagebreak
Struktura pracy jest następująca:
\begin{itemize}
\item Rozdział 2  przedstawia dotychczasowy przebieg procesu oraz opisuje wykorzystane narzędzia.

\item Rozdział 3 jest poświęcony architekturze rozwiązania. Omówiono w nim założenia projektowe, których należało się trzymać. Ponadto zawiera on pierwotną koncepcje rozwiązania, która opisuje jakie funkcjonalności powinny zostać zaimplementowane.

\item Rozdział 4 zawiera końcową implementację poszczególnych komponentów. Opisano w nim każdy element aplikacji oraz omówiono integrację między poszczególnymi komponentami.

\item Rozdział 5 stanowi przedstawienie działania systemu. Zawiera on opis przebiegu procesu z wykorzystaniem stworzonego rozwiązania.

\item Rozdział 6 opisuje problemy napotkane podczas implementacji oraz ich rozwiązania, zastosowane przy implementacji.

\item Rozdział 7 stanowi podsumowanie pracy. Przedstawiono w nim wnioski oraz możliwości dalszego rozwoju aplikacji.
\end{itemize}

\vspace{0.5cm}
\noindent Michał Gajdzis w ramach niniejszej pracy zaimplementował:
\begin{itemize}
    \item strukturę bazy danych SharePoint,
    \item przepływ pozwalający na zapis i formatowanie pliku,
    \item mechanizm zmiany nazw kolumn w arkuszu,
    \item mechanizm zapisu danych wprowadzanych przez użytkownika w bazie danych SharePoint,
    \item ekran generowania raportu wraz z potrzebnymi przepływami.
\end{itemize}

\vspace{0.5cm}
\noindent Remigiusz Wolniak wykonał:
\begin{itemize}
    \item przepływ służący do importu danych z pliku do bazy danych,
    \item skrypt formatujący plik,
    \item interfejs oraz logikę ekranów: głównego, dodawania danych od systemu, edycji danych oraz samouczka,
    \item mechanizmy zarządzania danymi wewnątrz aplikacji.
\end{itemize}