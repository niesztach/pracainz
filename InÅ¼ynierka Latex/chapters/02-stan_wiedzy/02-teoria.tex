
\chapter{Stan wiedzy}

Praca koncentruje się na automatyzacji biurowej procesu wymiany wycen usług oraz kosztów IT, który do tej pory opierał się na ręcznym zarządzaniu za pomocą plików programu Excel. Tego rodzaju podejście było czasochłonne, mało efektywne i podatne na błędy, szczególnie w przypadku dużych zbiorów danych oraz konieczności współpracy między działami firmy.

W celu usprawnienia tego procesu zaprojektowano oraz wdrożono aplikację opartą na plat ormie Microsoft Power Apps. Nowe rozwiązanie oferuje intuicyjny interfejs, automatyzację kluczowych etapów oraz możliwość pełnego śledzenia przepływu danych. Dzięki temu aplikacja znacząco przyczynia się do zwiększenia efektywności pracy biurowej, minimalizując ryzyko błędów i usprawniając współpracę w organizacji.


%\chapter{Podstawy teoretyczne}

Rozdział teoretyczny --- przegląd literatury naświetlający stan wiedzy na dany temat. 

Przegląd literatury naświetlający stan wiedzy na dany temat obejmuje rozdziały pisane na podstawie
literatury, której wykaz zamieszczany jest w części pracy pt.~\emph{Literatura} (lub inaczej \emph{Bibliografia},
\emph{Piśmiennictwo}). W tekście pracy muszą wystąpić odwołania do wszystkich pozycji zamieszczonych w
wykazie literatury. \textbf{Nie należy odnośników do literatury umieszczać w stopce strony.} Student jest
bezwzględnie zobowiązany do wskazywania źródeł pochodzenia informacji przedstawianych w pracy,
dotyczy to również rysunków, tabel, fragmentów kodu źródłowego programów itd. Należy także podać
adresy stron internetowych w przypadku źródeł pochodzących z Internetu.


