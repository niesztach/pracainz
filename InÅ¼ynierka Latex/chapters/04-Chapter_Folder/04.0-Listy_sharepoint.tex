\section{Utworzenie bazy danych na platformie Sharepoint}

Zgodnie z koncepcją, baza danych utworzona została w środowisku SharePoint.
Pierwszym krokiem jest stworzenie strony dedykowanej temu procesowi. W ekranie startowym programu należy wybrać opcję \emph{Utwórz witrynę}. Następnie należy wybrać typ witryny \emph{Witryna zespołu} oraz szablon określający jej wygląd. Na koniec należy określić nazwę tworzonej strony.

Kolejnym krokiem jest utworzenie struktury list oraz plików. Na potrzeby procesu utworzono dwa foldery -- \emph{TempFiles} przechowujący arkusze przed zapisaniem ich danych na listach oraz \emph{ArchivedFiles} przechowujący pliki po zakończeniu procesu.
Ponadto utworzono trzy listy:
\begin{itemize}
    \item \emph{Lista\_Uslug},
    \item \emph{Lista\_Kwot},
    \item \emph{Lista\_Indykacji}.
\end{itemize}
Ich struktura jest taka sama jak opisana w podsekcji \ref{Subsec: StrukturaBazyDanych}. 
W celu dodania tych elementów należy wejść w utworzoną witrynę, następnie wybrać \emph{Zawartość witryny} z bocznego paska nawigacji. Dalej, należy wcisnąć przycisk \emph{Nowy} i wybrać typ elementu -- \emph{Lista} oraz \emph{Biblioteka dokumentów}. Na końcu trzeba podać nazwę oraz opcjonalnie opis elementu.
Po wykonaniu tych kroków, folder jest gotowy do użycia. Jednakże w przypadku list należy jeszcze zdefiniować kolumny. \\W tym celu należy wejść w utworzoną listę i wybrać przycisk \emph{Dodaj kolumnę} znajdujący się po prawej stronie domyślnie utworzonych kolumn. Dalej pojawi się okno z możliwością wyboru typu kolumny oraz jej nazwy. Po zatwierdzeniu, kolumna zostanie dodana do listy.
Na koniec należy skonfigurować niektóre kolumny. Aby to zrobić należy wybrać ustawienia strony i nacisnąć pozycję \emph{Ustawienia listy}. Tam wybieramy nazwę kolumny, którą chcemy skofigurować.\\
Tabela \ref{tab:SharepointList} przedstawia konfigurację kolumn listy:
\begin{table}[t]
    \centering
    \caption{Konfiguracja listy na platformie Sharepoint}
    \label{tab:SharepointList}
\begin{tabular}{|l|l|}
    \hline
    \textbf{Nazwa kolumny} & \textbf{Typ} \\ \hline
    Service\_ID & Liczba \\ \hline
    Service\_Name & Pojedynczy wiersz tekstu \\ \hline
    Service\_Group & Pojedynczy wiersz tekstu \\ \hline
    Service\_Sub\_Group & Pojedynczy wiersz tekstu \\ \hline
    Service\_Main\_Group & Pojedynczy wiersz tekstu \\ \hline
    Instruction\_Link & Pojedynczy wiersz tekstu \\ \hline
    Unit\_Of\_Measurement & Pojedynczy wiersz tekstu \\ \hline
    Settlement\_Type & Pojedynczy wiersz tekstu \\ \hline
    Business\_Service\_Manager & Pojedynczy wiersz tekstu \\ \hline
    Current\_Year\_Plan\_EUR & Liczba \\ \hline
    QTY\_Current\_Year & Liczba \\ \hline
    Next\_Year\_Plan\_EUR & Liczba \\ \hline
    QTY\_Next\_Year & Liczba \\ \hline
    Difference & Obliczeniowa \\ \hline
    MPK & Pojedynczy wiersz tekstu \\ \hline
    Year & Liczba \\ \hline
    IndicationNo & Liczba \\ \hline
    Comment\_Date & Pojedynczy wiersz tekstu \\ \hline
    Comment\_Author & Wiele wierszy tekstu \\ \hline
    Comment\_PZ\_to\_WOB & Wiele wierszy tekstu \\ \hline
    Comment\_BSM & Wiele wierszy tekstu \\ \hline
    Comment\_K-DES & Wiele wierszy tekstu \\ \hline
    Comment\_Intern & Wiele wierszy tekstu \\ \hline
    Decision & Liczba \\ \hline
    Final\_comment & Wiele wierszy tekstu \\ \hline
    \end{tabular}
\end{table}

    \noindent Kolumna \emph{Difference} jest kolumną obliczeniową co oznacza, że jej wartość jest obliczana względem podanej formuły. W tym przypadku oblicza ona różnicę cen między rokiem następnym a bieżącym. \\
    Kolumna \emph{MPK} pomimo wartości liczbowych, jest typu tekstowego gdyż znacznie upraszcza to filtrowanie oraz przepisywanie istniejących numerów do nowych danych. \\
    Kolumna \emph{Decision} jest kolumną liczbową ponieważ status decyzji konwertowany jest na kod liczbowy aby przyspieszyć proces filtrowania:
    \begin{itemize}
        \item \emph{Accepted} $\rightarrow$ 1,
        \item \emph{Not Accepted} $\rightarrow$ -1,
        \item \emph{No Status} $\rightarrow$ 0.
    \end{itemize}

    Początkowo planowano wykorzystać wbudowany w SharePoint mechanizm \emph{lookup} do implementacji relacji między listami. Jednak ze względu na ograniczenie tego mechanizmu do relacji wyłącznie między dwiema listami, zdecydowano się na realizację powiązań na poziomie logiki aplikacji. Szczegóły tej implementacji zostały opisane w rozdziale dotyczącym realizacji systemu.
    

